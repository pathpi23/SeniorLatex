\appendix{Database Dictionary}
\noindent{\large\bf ตารางแสดงรายละเอียดข้อมูลในระบบฐานข้อมูล} \\

\begin{table}[H]
\caption{ตารางอธิบายความหมายตาราง term\_word}\label{tbl:termword}
\begin{tabular}{|l|l|l|}
\hline
\multicolumn{3}{|c|}{term\_word}                                                                                                                            \\ \hline
\multicolumn{1}{|c|}{ชื่อคอลัมน์} & \multicolumn{1}{c|}{ความหมาย}            & \multicolumn{1}{c|}{ประเภท}                                                  \\ \hline
term\_word\_id                    & id สำหรับบ่งบอกคำศัพท์                   & \begin{tabular}[c]{@{}l@{}}INT (10) PK \\ Auto\_Increment\end{tabular}       \\ \hline
term                              & คำศัพท์                                  & VARCHAR (191)                                                                \\ \hline
frequency                         & จำนวนความถี่ของหนังสือที่มีคำศัพท์นี้อยู่ & INT (191)                                                                    \\ \hline
score\_idf                        & คะแนน idf ของคำศัพท์นี้                  & FLOAT (255,4)                                                                \\ \hline
rec\_create\_at                   & วันเวลาของการเพิ่มคำศัพท์นี้เข้าสู่ระบบ  & \begin{tabular}[c]{@{}l@{}}DATETIME (6) \\ current\_timestamp\end{tabular}   \\ \hline
rec\_modified\_at                 & วันเวลาที่อัปเดทข้อมูลของคำศัพท์         & \begin{tabular}[c]{@{}l@{}}DATETIME   (6) \\ current\_timestamp\end{tabular} \\ \hline
\end{tabular}
\end{table}

\begin{table}[H]
\caption{ตารางอธิบายความหมายตาราง user}\label{tbl:user}
\begin{tabular}{|l|l|l|}
\hline
\multicolumn{3}{|c|}{user}                                                                                                                                   \\ \hline
\multicolumn{1}{|c|}{ชื่อคอลัมน์} & \multicolumn{1}{c|}{ความหมาย}             & \multicolumn{1}{c|}{ประเภท}                                                  \\ \hline
user\_id                          & id สำหรับบ่งบอกผู้ใช้งาน                  & \begin{tabular}[c]{@{}l@{}}INT   (10) PK \\ Auto\_Increment\end{tabular}     \\ \hline
name                              & ชื่อของผู้ใช้งาน                          & VARCHAR (50)                                                                 \\ \hline
surname                           & นามสกุลของผู้ใช้งาน                       & VARCHAR (191)                                                                \\ \hline
role                              & ตำแหน่งของผู้ใช้งาน                       & VARCHAR (191)                                                                \\ \hline
username                          & ชื่อผู้ใช้งานสำหรับทำการ login            & VARCHAR (191)                                                                \\ \hline
password                          & รหัสผ่านผู้ใช้งานสำหรับทำการ login        & VARCHAR (191)                                                                \\ \hline
create\_at                        & วันเวลาของผู้ใช้งานของการเพิ่มเข้าสู่ระบบ & \begin{tabular}[c]{@{}l@{}}DATETIME   (6) \\ current\_timestamp\end{tabular} \\ \hline
active                            & สถานะการระงับบัญชีผู้ใช้งาน               & \begin{tabular}[c]{@{}l@{}}INT   (11) \\ Default 1\end{tabular}              \\ \hline
\end{tabular}
\end{table}

\begin{table}[H]
\caption{ตารางอธิบายความหมายตาราง score}\label{tbl:score}        
\begin{tabular}{|l|l|l|}
\hline
\multicolumn{3}{|c|}{score}                                                                                                                       \\ \hline
\multicolumn{1}{|c|}{ชื่อคอลัมน์} & \multicolumn{1}{c|}{ความหมาย}  & \multicolumn{1}{c|}{ประเภท}                                                  \\ \hline
score\_id                         & id สำหรับบ่งบอกคะแนนของคำศัพท์ & \begin{tabular}[c]{@{}l@{}}INT (10) PK \\ Auto\_Increment\end{tabular}       \\ \hline
score\_tf                         & คะแนน tf ของคำศัพท์            & FLOAT (255,4)                                                                \\ \hline
score\_tf\_idf                    & คะแนน tf-idf ของคำศัพท์        & FLOAT   (255,4)                                                              \\ \hline
index\_term\_word\_id             & id สำหรับบ่งบอกคำศัพท์         & INT   (10)                                                                   \\ \hline
index\_document\_id               & id สำหรับบ่งบอกหนังสือ          & INT   (10)                                                                   \\ \hline
generate\_by                      & คะแนนถูกคำนวณโดยใคร            & \begin{tabular}[c]{@{}l@{}}VARCHAR   (191) \\ Default ‘default’\end{tabular} \\ \hline
rec\_status                       & สถานะการใช้คะแนนนี้            & \begin{tabular}[c]{@{}l@{}}INT   (191) \\ Default 1\end{tabular}             \\ \hline
\end{tabular}
\end{table}

\begin{table}[H]
\caption{ตารางอธิบายความหมายตาราง pre\_term\_in\_page}\label{tbl:preterminpage}        
\begin{tabular}{|l|l|l|}
\hline
\multicolumn{1}{|c|}{ชื่อคอลัมน์} & \multicolumn{1}{c|}{ความหมาย}                                      & \multicolumn{1}{c|}{ประเภท}                                                   \\ \hline
pre\_term\_in\_page\_id           & id   สำหรับบ่งบอกคำศัพท์ชั่วคราวที่รอให้ผู้ใช้งานตรวจสอบ           & \makecell[l]{INT   (10) PK\\Auto\_Increment} \\ \hline
pre\_term                         & คำศัพท์ชั่วคราวที่รอให้ผู้ใช้ตรวจสอบ                               & VARCHAR   (191)                                                               \\ \hline
index\_page\_in\_document\_id     & id   สำหรับบ่งบอกที่อยู่ของคำศัพท์ชั่วคราวที่รอให้ผู้ใช้งานตรวจสอบ & INT (10) FK                                                                   \\ \hline
\end{tabular}
\end{table}

\begin{table}[H]
\caption{ตารางอธิบายความหมายตาราง page\_in\_document}\label{tbl:pageindocument}        
\begin{tabular}{|l|l|l|}
\hline
\multicolumn{3}{|c|}{page\_in\_document}                                                                                             \\ \hline
\multicolumn{1}{|c|}{ชื่อคอลัมน์} & \multicolumn{1}{c|}{ความหมาย}                                    & \multicolumn{1}{c|}{ประเภท}   \\ \hline
page\_in\_document\_id            & id สำหรับบ่งบอกที่อยู่ของคำศัพท์ชั่วคราวที่รอให้ผู้ใช้งานตรวจสอบ & \begin{tabular}[c]{@{}l@{}}INT (10) PK\\ Auto\_Increment\end{tabular} \\ \hline
page\_index                       & หน้าของหนังสือ                                                    & INT (191)                     \\ \hline
name                              & ชื่อ File ของข้อมูล                                              & VARCHAR (191)                 \\ \hline
rec\_status\_confirm              & สถานะการยืนยันโดยผู้ใช้งาน                                       & \begin{tabular}[c]{@{}l@{}}INT (2) PK\\ Default 2\end{tabular}            \\ \hline
index\_document\_id               & id สำหรับบ่งบอกหนังสือ                                            & INT (10) FK                   \\ \hline
\end{tabular}
\end{table}

\begin{table}[H]
\caption{ตารางอธิบายความหมายตาราง nodejs\_log}\label{tbl:nodejslog}        
\begin{tabular}{|l|l|l|}
\hline
\multicolumn{3}{|c|}{nodejs\_log}                                                                                                                                  \\ \hline
\multicolumn{1}{|c|}{ชื่อคอลัมน์} & \multicolumn{1}{c|}{ความหมาย}                   & \multicolumn{1}{c|}{ประเภท}                                                  \\ \hline
nodejs\_log\_id                   & id สำหรับการจัดเก็บประวัติการทำงานฝั่ง nodejs   & \begin{tabular}[c]{@{}l@{}}INT   (10) PK \\ Auto\_Increment\end{tabular}     \\ \hline
status\_code                      & เก็บสถานะ HTTP หลังจากที่ส่งไปแล้วว่าได้สถานะใด & INT (191)                                                                    \\ \hline
header\_date                      & เก็บข้อมูล header ของ HTTP ที่ส่งไป             & VARCHAR (191)                                                                \\ \hline
server                            & ชื่อรูปแบบของเซิฟเวอร์ที่ส่งไป                  & VARCHAR (191)                                                                \\ \hline
url                               & ตำแหน่งโดเมนหรือ IP ที่ส่งไป                    & INT (10) FK                                                                  \\ \hline
content\_type                     & รูปแบบเนื้อหาที่ส่งไป                           & VARCHAR (191)                                                                \\ \hline
rec\_status                       & สถานะที่บอกว่าการส่งเกิดข้อผิดพลาดระหว่างทาง    & INT (191)                                                                    \\ \hline
rec\_create\_date                 & วันเวลาที่ทำการส่ง   ณ ตอนนั้น                  & \begin{tabular}[c]{@{}l@{}}DATETIME   (6) \\ current\_timestamp\end{tabular} \\ \hline
\end{tabular}
\end{table}

\begin{table}[H]
\caption{ตารางอธิบายความหมายตาราง knex\_migrations\_lock}\label{tbl:knexmigrationslock}        
\begin{tabular}{|l|l|l|}
\hline
\multicolumn{3}{|c|}{knex\_migrations\_lock}                                                                                                          \\ \hline
\multicolumn{1}{|c|}{ชื่อคอลัมน์} & \multicolumn{1}{c|}{ความหมาย}            & \multicolumn{1}{c|}{ประเภท}                                            \\ \hline
index                             & id บ่งบอกลำดับของไฟล์ migration ของ knex & \begin{tabular}[c]{@{}l@{}}INT (10) PK \\ Auto\_Increment\end{tabular} \\ \hline
is\_locked                        & สถานะของไฟล์ migration                   & INT (11)                                                               \\ \hline
\end{tabular}
\end{table}

\begin{table}[H]
\caption{ตารางอธิบายความหมายตาราง knex\_migrations}\label{tbl:knexmigrations}        
\begin{tabular}{|l|l|l|}
\hline
\multicolumn{3}{|c|}{knex\_migrations}                                                                                                                        \\ \hline
\multicolumn{1}{|c|}{ชื่อคอลัมน์} & \multicolumn{1}{c|}{ความหมาย}                    & \multicolumn{1}{c|}{ประเภท}                                            \\ \hline
id                                & id บ่งบอกลำดับการทำงานของไฟล์ migration ของ knex & \begin{tabular}[c]{@{}l@{}}INT (10) PK \\ Auto\_Increment\end{tabular} \\ \hline
name                              & ชื่อไฟล์ migration ที่ถูกทำงานเรียบร้อย          & VARCHAR (255)                                                          \\ \hline
batch                             & ลำดับที่                                         & INT (11)                                                               \\ \hline
migration\_time                   & เวลาที่ถุกสั่งให้ทำงาน                           & \begin{tabular}[c]{@{}l@{}}TIMESTAMP\\ current\_timestamp\end{tabular} \\ \hline
\end{tabular}
\end{table}

\begin{table}[H]
\caption{ตารางอธิบายความหมายตาราง indexing\_publisher\_document}\label{tbl:indexingpublisherdocument}        
\begin{tabular}{|l|l|l|}
\hline
\multicolumn{3}{|c|}{indexing\_publisher\_document}                                                                                             \\ \hline
\multicolumn{1}{|c|}{ชื่อคอลัมน์} & \multicolumn{1}{c|}{ความหมาย}      & \multicolumn{1}{c|}{ประเภท}                                            \\ \hline
indexing\_publisher\_id           & id สำหรับบ่งบอกสำนักพิมพ์          & \begin{tabular}[c]{@{}l@{}}INT (10) PK \\ Auto\_Increment\end{tabular} \\ \hline
publisher                         & ชื่อสำนักพิมพ์                     & VARCHAR (191)                                                          \\ \hline
frequency                         & จำนวนของสำนักพิมพ์นี้ที่ถูกอ้างอิง & INT (191)                                                              \\ \hline
\end{tabular}
\end{table}


\begin{table}[H]
\caption{ตารางอธิบายความหมายตาราง indexing\_publisher\_email\_document}\label{tbl:indexingpublisherdocument}            
\begin{tabular}{|l|l|l|}
\hline
\multicolumn{3}{|c|}{indexing\_publisher\_email\_document}                                                                                      \\ \hline
\multicolumn{1}{|c|}{ชื่อคอลัมน์} & \multicolumn{1}{c|}{ความหมาย}      & \multicolumn{1}{c|}{ประเภท}                                            \\ \hline
indexing\_publisher\_email\_id    & id สำหรับบ่งบอกสำนักพิมพ์          & \begin{tabular}[c]{@{}l@{}}INT (10) PK \\ Auto\_Increment\end{tabular} \\ \hline
publisher\_email                  & e-mail ของสำนักพิมพ์               & VARCHAR (191)                                                          \\ \hline
frequency                         & จำนวนของสำนักพิมพ์นี้ที่ถูกอ้างอิง & INT (191)                                                              \\ \hline
\end{tabular}
\end{table}

\begin{table}[H]
\caption{ตารางอธิบายความหมายตาราง indexing\_issued\_date\_document}\label{tbl:indexingissueddatedocument}        
\begin{tabular}{|l|l|l|}
\hline
\multicolumn{3}{|c|}{indexing\_issued\_date\_document}                                                                                                 \\ \hline
\multicolumn{1}{|c|}{ชื่อคอลัมน์} & \multicolumn{1}{c|}{ความหมาย}             & \multicolumn{1}{c|}{ประเภท}                                            \\ \hline
indexing\_issued\_date\_id        & id สำหรับบ่งบอกปีที่เขียน                 & \begin{tabular}[c]{@{}l@{}}INT (10) PK \\ Auto\_Increment\end{tabular} \\ \hline
issued\_date                      & วันเวลาของปีที่เขียนหนังสือ                & DATE                                                                   \\ \hline
frequency                         & จำนวนของวันเวลาของปีที่เขียนที่ถูกอ้างอิง & INT (191)                                                              \\ \hline
\end{tabular}
\end{table}

\begin{table}[H]
\caption{ตารางอธิบายความหมายตาราง indexing\_creator\_orgname\_document}\label{tbl:indexingcreatororgnamedocument}        
\begin{tabular}{|l|l|l|}
\hline
\multicolumn{3}{|c|}{indexing\_creator\_orgname\_document}                                                                                                \\ \hline
\multicolumn{1}{|c|}{ชื่อคอลัมน์} & \multicolumn{1}{c|}{ความหมาย}                & \multicolumn{1}{c|}{ประเภท}                                            \\ \hline
indexing\_creator\_orgname\_id    & id สำหรับบ่งบอกชื่อหน่วยงานรับผิดชอบสังกัด   & \begin{tabular}[c]{@{}l@{}}INT (10) PK \\ Auto\_Increment\end{tabular} \\ \hline
creator\_orgname                  & ชื่อหน่วยงานรับผิดชอบสังกัด                  & VARCHAR (191)                                                          \\ \hline
frequency                         & จำนวนของหน่วยงานรับผิดชอบสังกัดที่ถูกอ้างอิง & INT (191)                                                              \\ \hline
\end{tabular}
\end{table}

\begin{table}[H]
\caption{ตารางอธิบายความหมายตาราง indexing\_creator\_document}\label{tbl:indexingcreatordocument}
\begin{tabular}{|l|l|l|}
\hline
\multicolumn{3}{|c|}{indexing\_creator\_document}                                                                                                \\ \hline
\multicolumn{1}{|c|}{ชื่อคอลัมน์} & \multicolumn{1}{c|}{ความหมาย}       & \multicolumn{1}{c|}{ประเภท}                                            \\ \hline
indexing\_creator\_id             & id สำหรับบ่งบอกชื่อผู้เขียนหนังสือ   & \begin{tabular}[c]{@{}l@{}}INT (10) PK \\ Auto\_Increment\end{tabular} \\ \hline
creator                           & ชื่อของผู้เขียนหนังสือ               & VARCHAR (191)                                                          \\ \hline
frequency                         & จำนวนของผู้เขียนหนังสือที่ถูกอ้างอิง & INT (191)                                                              \\ \hline
\end{tabular}
\end{table}


\begin{table}[H]
\caption{ตารางอธิบายความหมายตาราง dc\_contributors}\label{tbl:dccontributor}
\begin{tabular}{|l|l|l|}
\hline
\multicolumn{3}{|c|}{dc\_contributors}                                                                                                      \\ \hline
\multicolumn{1}{|c|}{ชื่อคอลัมน์} & \multicolumn{1}{c|}{ความหมาย}                                             & \multicolumn{1}{c|}{ประเภท} \\ \hline
dc\_contributors\_id              & id สำหรับบ่งบอกข้อมูลความสัมพันธ์ของชื่อหน่วยข้อมูลผู้ร่วมงาน   กับหนังสือ & \begin{tabular}[c]{@{}l@{}}INT (10) PK\\ Auto\_Increment\end{tabular} \\ \hline
index\_contributor\_id            & id สำหรับบ่งบอกชื่อผู้เขียนหนังสือ                                         & INT(10) FK                  \\ \hline
index\_document\_id               & id สำหรับบ่งบอกชื่อหน่วยข้อมูลผู้ร่วมงาน                                  & INT(10) FK                  \\ \hline
\end{tabular}
\end{table}

\begin{table}[H]
\caption{ตารางอธิบายความหมายตาราง indexing\_contributor\_document}\label{tbl:indexingcontributordocument}
\begin{tabular}{|l|l|l|}
\hline
\multicolumn{3}{|c|}{indexing\_contributor\_document}                                                        \\ \hline
\multicolumn{1}{|c|}{ชื่อคอลัมน์} & \multicolumn{1}{c|}{ความหมาย}              & \multicolumn{1}{c|}{ประเภท} \\ \hline
indexing\_contributor\_id         & id   สำหรับบ่งบอกชื่อหน่วยข้อมูลผู้ร่วมงาน & \begin{tabular}[c]{@{}l@{}}INT (10) PK\\ Auto\_Increment\end{tabular}   \\ \hline
contributor                       & ชื่อหน่วยข้อมูลผู้ร่วมงาน                  & VARCHAR (191)               \\ \hline
frequency                         & จำนวนของหน่วยข้อมูลผู้ร่วมงานที่ถูกอ้างอิง & INT (191)                   \\ \hline
\end{tabular}
\end{table}

\begin{table}[H]
\caption{ตารางอธิบายความหมายตาราง indexing\_contributor\_role\_document}\label{tbl:indexingcontributorroledocument}
\begin{tabular}{|l|l|l|}
\hline
\multicolumn{3}{|c|}{indexing\_contributor\_role\_document}                                                                                        \\ \hline
\multicolumn{1}{|c|}{ชื่อคอลัมน์} & \multicolumn{1}{c|}{ความหมาย}              & \multicolumn{1}{c|}{ประเภท}                                       \\ \hline
indexing\_contributor\_role\_id   & id   สำหรับบ่งบอกตำแหน่งของผู้ร่วมงาน      & \begin{tabular}[c]{@{}l@{}}INT(10)\\ Auto\_Increment\end{tabular} \\ \hline
contributor\_role                 & ชื่อตำแหน่งของผู้ร่วมงาน                   & VARCHAR (191)                                                     \\ \hline
index\_contributor                & id   สำหรับบ่งบอกชื่อหน่วยข้อมูลผู้ร่วมงาน & INT (191)                                                         \\ \hline
\end{tabular}
\end{table}


\begin{table}[H]
\caption{ตารางอธิบายความหมายตาราง dc\_type}\label{tbl:dctype}
\begin{tabular}{|l|l|l|}
\hline
\multicolumn{3}{|c|}{dc\_type}                                                                                                               \\ \hline
\multicolumn{1}{|c|}{ชื่อคอลัมน์} & \multicolumn{1}{c|}{ความหมาย}   & \multicolumn{1}{c|}{ประเภท}                                              \\ \hline
DC\_type\_id                    & id สำหรับบ่งบอกประเภทของหนังสือ & \begin{tabular}[c]{@{}l@{}}INT (10) PK  \\ Auto\_Increment\end{tabular} \\ \hline
DC\_type                        & ประเภทของหนังสือ                & VARCHAR   (191)                                                         \\ \hline
index\_document\_id             & id สำหรับบ่งบอกหนังสือ          & INT   (10)                                                             \\ \hline
\end{tabular}
\end{table}

\begin{table}[H]
\caption{ตารางอธิบายความหมายตาราง dc\_relation}\label{tbl:dcrelation}
\begin{tabular}{|l|l|l|}
\hline
\multicolumn{3}{|c|}{dc\_relation}                                                                      \\ \hline
\multicolumn{1}{|c|}{ชื่อคอลัมน์} & \multicolumn{1}{c|}{ความหมาย}        & \multicolumn{1}{c|}{ประเภท}  \\ \hline
DC\_relation\_id                  & id   สำหรับบ่งบอกหนังสือที่เกี่ยวข้อง & \begin{tabular}[c]{@{}l@{}}INT (10) PK\\ Auto\_Increment\end{tabular} \\ \hline
DC\_relation                      & ชื่อหนังสือที่เกี่ยวข้อง              & VARCHAR   (191)              \\ \hline
index\_document\_id               & id สำหรับบ่งบอกหนังสือ                & INT   (10)                   \\ \hline
\end{tabular}
\end{table}

\begin{table}[H]
\caption{ตารางอธิบายความหมายตาราง dc\_keyword}\label{tbl:dckeyword}
\begin{tabular}{|l|l|l|}
\hline
\multicolumn{3}{|c|}{dc\_keyword}                                                                \\ \hline
\multicolumn{1}{|c|}{ชื่อคอลัมน์} & \multicolumn{1}{c|}{ความหมาย} & \multicolumn{1}{c|}{ประเภท}  \\ \hline
DC\_keyword\_id                   & id   สำหรับบ่งบอกคำสัญ         & \begin{tabular}[c]{@{}l@{}}INT (10) PK\\ Auto\_Increment\end{tabular} \\ \hline
DC\_keyword                       & คำศัพท์                       & VARCHAR   (191)              \\ \hline
index\_document\_id               & id สำหรับบ่งบอกหนังสือ         & INT   (10)                   \\ \hline
\end{tabular}
\end{table}

% \begin{table}[H]
% \caption{ตารางอธิบายความหมายตาราง document}\label{tbl:document}
% \vspace*{-1.25em}
% \end{table}
\begin{longtable}[l]{|l|l|l|}
\caption{ตารางอธิบายความหมายตาราง document}
\label{tbl:document}
\endfirsthead
\endhead
\hline
\multicolumn{3}{|c|}{document}                                                                \\ \hline
\multicolumn{1}{|c|}{ชื่อคอลัมน์}      & \multicolumn{1}{c|}{ความหมาย}                    & \multicolumn{1}{c|}{ประเภท}                                                       \\ \hline 
document\_id                         & id สำหรับบ่งบอกหนังสือ                                        & \begin{tabular}[c]{@{}l@{}}INT(10)   \\ Auto\_Increment\end{tabular} \\ \hline
status\_process\_document            & สถานะการทำงานของหนังสือ                                       & INT   (2)                                                            \\ \hline
name                                 & ชื่อไฟล์ PDF หนังสือ                                          & VARCHAR   (191)                                                      \\ \hline
version                              & ครั้งที่ตีพิมพ์                                              & INT   (191)                                                          \\ \hline
page\_start                          & หน้าหนังสือที่กำหนดเป็นหน้าเริ่ม                              & INT   (191)                                                          \\ \hline
amount\_page                         & จำนวนหน้าทั้งหมดของหนังสือ                                    & INT   (191)                                                          \\ \hline
path                                 & ตำแหน่งไฟล์ PDF ที่ผู้ใช้งานอัปโหลดเข้าสู่ระบบ               & TEXT                                                                 \\ \hline
path\_image                          & ตำแหน่งไฟล์รูปภาพของหนังสือ                                   & TEXT                                                                 \\ \hline
name                                 & ชื่อไฟล์ PDF หนังสือ                                          & VARCHAR (191)                                                        \\ \hline
DC\_title                            & ชื่อหนังสือ                                                   & VARCHAR (191)                                                        \\ \hline
DC\_title\_alternative               & ชื่อรองของหนังสือ                                             & VARCHAR (191)                                                        \\ \hline
DC\_description\_table\_of\_contents & สาระสำคัญที่มาจากสารบัญ                                      & TEXT                                                                 \\ \hline
DC\_description\_note                & รายละเอียดทั่วไปของหนังสือ                                    & TEXT                                                                 \\ \hline
DC\_description\_summary             & สาระสำคัญของข้อมูลสารสนเทศที่ผ่านการค้นหา   รวบรวม วิเคราะห์ & TEXT                                                                 \\ \hline
DC\_description\_abstract            & ข้อมูลสรุปจากบทคัดย่อ   วิทยานิพนธ์ และเนื้อหา               & TEXT                                                                 \\ \hline
DC\_format                           & รูปแบบข้อมูลที่ถูกจัดเก็บในระบบ                              & VARCHAR (191)                                                        \\ \hline
DC\_format\_extent                   & ขนาดของไฟล์หนังสือ                                            & VARCHAR (191)                                                        \\ \hline
DC\_identifier\_URL                  & แหล่งที่มาของหนังสือ                                          & VARCHAR   (191)                                                      \\ \hline
DC\_identifier\_ISBN                 & เลขมาตราฐานสากลของหนังสือ                                     & VARCHAR   (191)                                                      \\ \hline
DC\_source                           & หน่วยข้อมูลต้นฉบับ                                           & VARCHAR   (191)                                                      \\ \hline
DC\_language                         & ภาษาของหนังสือ                                                & VARCHAR   (191)                                                      \\ \hline
DC\_coverage\_spatial                & สถานที่ของหนังสือที่เป็นเจ้าของ                               & VARCHAR   (191)                                                      \\ \hline
DC\_coverage\_temporal               & ช่วงเวลาในหน่วยปีของหนังสือ                                   & VARCHAR   (191)                                                      \\ \hline
DC\_rights                           & ระดับการเข้าถึงของข้อมูล                                     & VARCHAR   (191)                                                      \\ \hline
DC\_rights\_access                   & ตำแหน่งที่มีสิทธิ์ในการเข้าถึงข้อมูล                         & VARCHAR   (191)                                                      \\ \hline
thesis\_degree\_name                 & ชื่อเต็มของปริญญา                                            & VARCHAR   (191)                                                      \\ \hline
thesis\_degree\_level                & ระดับของปริญญา                                               & VARCHAR   (191)                                                      \\ \hline
thesis\_degree\_discipline           & สาขาวิชา                                                     & VARCHAR   (191)                                                      \\ \hline
thesis\_degree\_grantor              & มหาวิทยาลัย                                                  & VARCHAR   (191)                                                      \\ \hline
index\_creator                       & id สำหรับบ่งบอกชื่อผู้เขียนหนังสือ                            & INT   (10) FK                                                        \\ \hline
index\_creator\_orgname              & id สำหรับบ่งบอกชื่อหน่วยงานรับผิดชอบสังกัด                   & INT   (10) FK                                                        \\ \hline
index\_publisher                     & id สำหรับบ่งบอกสำนักพิมพ์                                    & INT   (10) FK                                                        \\ \hline
index\_publisher\_email              & id สำหรับบ่งบอกอีเมล์สำนักพิมพ์                              & INT   (10) FK                                                        \\ \hline
index\_issued\_date                  & id สำหรับบ่งบอกปีที่เขียน                                    & INT   (10) FK                                                        \\ \hline
rec\_create\_at                      & วันเวลาของหนังสือที่ถูกนำเข้าสู่ระบบ         & \begin{tabular}[c]{@{}l@{}}DATETIME   (6)   \\ current\_timestamp \end{tabular}             \\ \hline
rec\_create\_by                      & id สำหรับบ่งบอกผู้ใช้งานที่นำหนังสือเข้าสู่ระบบ               & INT   (10) FK                                                        \\ \hline
rec\_modified\_at                    & วันเวลาของหนังสือที่ถูกแก้ไขข้อมูล           & \begin{tabular}[c]{@{}l@{}}DATETIME   (6)   \\ current\_timestamp \end{tabular}            \\ \hline
rec\_modified\_by                    & id สำหรับบ่งบอกผู้ใช้งานที่แก้ไขหนังสือในระบบ                 & INT   (10) FK                                                        \\ \hline
rec\_status                          & ค่าสถานะของหนังสือสำหรับการใช้งาน                            & INT   (2)                                                            \\ \hline
\end{longtable}