\chapter{บทนำ}

\section{คำสำคัญ}

Natural language processing, RESTful Service, Optical character recognition, Image Processing, Information retrieval, Term Frequency-Inverse Document Frequency, Word2Vec, Word Embedded 

\section{ความสำคัญของปัญหา}

นับตั้งแต่การก่อตั้งหอสมุดมหาวิทยาลัยเทคโนโลยีพระจอมเกล้าธนบุรี ได้มีการเก็บรวบรวมองค์ความรู้จากประสบการณ์การทำงานของคณะอาจารย์ ผู้เชี่ยวชาญในทางด้านศาสตร์ต่าง ๆ ในรูปแบบลายมือและสื่อสิ่งพิมพ์ไม่ว่าจะเป็น หนังสือ เอกสาร รวมถึงบันทึกเหตุการณ์ในอดีตในรูปของจดหมายเหตุเพื่อส่งต่อประวัติศาสตร์ความรู้ไปยังคนรุ่นหลังโดยมีการจัดเก็บอยู่ภายในหอจดหมายเหตุที่มีเจ้าหน้าที่บรรณารักษ์เป็นผู้ดูแล และเนื่องจากการที่ เอกสาร หนังสือยังไม่ได้มีการจัดเก็บในรูปแบบดิจิตอลทำให้เมื่อมีบุคคลภายนอกที่ต้องการข้อมูลเพื่อนำไปทำกิจกรรมต่าง ๆ ไม่ว่าจะเป็นการทำวิจัย รายงาน หรือหาข้อมูลเพื่อประกอบการประชุมก็ตามแต่ ก็จำเป็นที่จะต้องมาติดต่อเจ้าหน้าที่บรรณารักษ์ผู้ดูแลเพื่อที่จะให้เจ้าหน้าที่บรรณารักษ์ทำการค้นหาหนังสือที่มีเนื้อหาตามที่เราต้องการ ซึ่งการค้นหาข้อมูลที่ต้องการนั้นเจ้าหน้าที่จะต้องทำการค้นหาด้วยระบบมือทำให้การค้นหาข้อมูลดำเนินการไปอย่างล่าช้า นอกจากนั้นวิธีการหาข้อมูลของเจ้าหน้าที่บรรณารักษ์จะเลือกตรวจสอบข้อมูลของหนังสือจากการดูสารบัญทำให้ข้อมูลที่ได้รับมาอาจจะตกหล่นจากข้อมูลเล่มอื่นได้ 

เพื่ออำนวยความสะดวกให้กับบรรณารักษ์ในการสืบค้นข้อมูลและทำให้การบริการในการสืบค้นเอกสารต่าง ๆ และให้บุคคลภายนอกสามารถทำการค้นหาข้อมูลได้ด้วยตนเองครบถ้วนทางคณะผู้จัดทำโครงการจึงได้พัฒนาระบบการจัดเก็บเอกสารและระบบการค้นหาโดยการใช้เครื่องมือในการทำ OCR เพื่อแปลงเอกสารให้อยู่ในรูปแบบของเอกสาร digital และหาคำสำคัญในการสร้าง tag ด้วยวิธี Term Frequency - Inverse Document Frequency เพื่อเพิ่มประสิทธิ์ภาพให้กับการค้นหา 

\section{ประเภทของโครงงาน}

นำเสนอความต้องการของผู้มีส่วนได้ส่วนเสียเฉพาะกลุ่ม 

\section{วิธีการที่นำเสนอ}

ระบบการค้นหาเอกสาร มีขั้นตอนการทำงานดังนี้

\begin{enumerate}
    \item นำเอกสารมาแปลงเป็นรูปภาพในรูปแบบสแกน
    \item นำรูปภาพเข้าสู่ระบบโดยใช้การรับส่งข้อมูลแบบ RESTful API ในระบุประเภทของการใช้งาน
    \item นำรูปภาพผ่านกระบวนการ Image Processing โดยใช้ OpenCV ในการลบส่วนอื่น ๆที่ไม่ใช่ข้อความออกและตัดเฉพาะข้อความเพื่อนำไปใช้ในขั้นต่อไป
    \item นำรูปที่ผ่านการทำ Image Processing มาเข้าสู่ระบบ OCR เพื่อแปลงข้อมูลจากรูปภาพมาเป็นข้อความในระบบดิจิตอล
    \item นำข้อมูลที่เก็บไว้มาทำการตัดแบ่งคำภาษาไทยและแก้คำผิด
    \item ค้นหาคำสำคัญโดยใช้วิธี TF-IDF เพื่อนำมาใช้ในการสร้าง Tag 
    \item นำข้อมูลที่ถูกแปลงเก็บและข้อมูลเกี่ยวกับ Tag ลงในดาต้าเบส 
    \item ทำระบบค้นหาในรูปแบบ Cosine Similarity 
    \item ทำระบบหาคำใกล้เคียงโดยใช้วิธี Word2Vec
    \item ทำแพลตฟอร์มเว็ปไซต์เพื่อเป็น User Interface ให้กับผู้ใช้งานได้ใช้งานสำหรับการใช้งานในการค้นหาข้อมูลและเพิ่มข้อมูลหนังสือลงไปในฐานข้อมูลเพิ่ม
\end{enumerate}
\section{วัตถุประสงค์}
\begin{enumerate}
    \item สร้างระบบแปลงข้อมูลเอกสารให้อยู่ในรูปแบบดิจิตอล
    \item สร้าง web platform เพื่อทำการค้นหาเอกสารจากคำค้น และพัฒนาเครื่องมือสนับสนุนการทำงานของบรรณารักษ์ประจำหอบรรณสารสนเทศ
    \item สร้างระบบการค้นหาโดยการใช้วิธีการ อินฟอเมชันรีทีฟวอล ซึ่งวัดความใกล้เคียงกันระหว่างคำค้นและข้อมูลในฐานข้อมูลโดยวิธี โคซาย ซิมิลาริตี้
    \item เพิ่มประสิทธิ์ภาพในการเข้าถึงข้อมูลในรูปแบบดิจิตอล
    \item เรียนรู้เรื่องการทำ Image processing
\end{enumerate}
\section{ขอบเขตของงานวิจัย}
\begin{enumerate}
    \item ระบบแปลงข้อมูลจากเอกสารและหนังสือเก่า รองรับเฉพาะเอกสารที่เป็นตัวอักษรแบบพิมพ์ และรองรับไฟล์เอกสารเฉพาะ PDF เท่านั้น
    \item ทำระบบตัดคำ Stop word ภาษาไทยโดยอ้างอิงมาจาก pythainip และภาษาอังกฤษจาก nltk
    \item ทำระบบค้นหาแบบ Cosine Similarity ในระบบ Information retrieval
    \item ข้อมูลหนังสือที่นำมาใช้คือหนังสือจำพวก งานแสดงกตเวทิตาจิต เอกสารรายงานประจำปี ตั้งแต่ปีพุทธศักราช 2527 ถึง 2560 รวมประมาณ 44 เล่ม จากหอจดหมายเหตุมหาวิทยาลัยเทคโนโลยีพระจอมเกล้าธนบุรี
    \item ทำ platform เว็บไซต์ในรูปแบบ responsive แต่ไม่รองรับขนาดมือถือ รองรับเฉพาะคอมพิวเตอร์หรือโน๊ตบุ๊ค
    \item การแปลงสิ่งพิมพ์เป็นดิจิตอลใช้ Tesseract ในการแปลงเอกสารและหนังสือเป็นรูปแบบดิจิตอล
    \item การตัดคำภาษาไทยทางคณะผู้จัดทำ จะใช้ freeware เช่น DeepCut มาใช้ในส่วนของการตัดคำภาษาไทย
\end{enumerate}
\section{เนื้อหาทางวิศวกรรมที่เป็นต้นฉบับ}
\begin{itemize}
    \item การทำ Image processing สำหรับการเตรียมภาพก่อนนำไปทำ OCR 
    
    โปรเจคของเราทำเกี่ยวกับการทำ OCR เพื่ออ่านภาพให้กลายเป็น text แต่ถึงแม้ว่าภาพที่ได้มาจะจากการสแกนหรือการถ่ายรูป แต่ถึงอย่างนั้น OCR ที่ใช้ก็ยังคงมีข้อจำกัดในเรื่องของคุณภาพของภาพที่ใช้ ถ้าเกิดว่าภาพที่ใช้เอียง หรือมี noise จะทำให้การอ่านมีประสิทธิภาพน้อยลง นอกจากนี้การตัดภาพแยกย่อหน้าแต่ละย่อหน้าทำให้การอ่านมีความถูกต้องมากยิ่งขึ้น

    \item การพัฒนาเว็บไซต์สำหรับการค้นหาหนังสือในหอจดหมายเหตุ
    
    เว็บไซต์ของเราจะใช้ ReactJS, NodeJs, python  ในการพัฒนาเว็บไซต์เป็น Interface ให้กับ user สำหรับการใช้งานระบบการค้นหาหนังสือ รวมถึงการอัปโหลดเอกสารเพื่อแปลงเอกสารเข้าสู่ระบบดิจิตอลและ API ต่าง ๆ
    
    \item คัดเลือกคำสำคัญออกมาเพื่อสร้าง tag 
    
    สำหรับแบ่งแยกหมวดหมู่ของหนังสือโดยใช้ หลักการของ TF-IDF ในการค้นหาคำสำคัญของหนังสือเพื่อนำมาสร้าง tag และใช้สำหรับการค้นหาข้อมูล
    
    \item ทำระบบค้นหาโดยใช้คำที่มีความหมายใกล้เคียง
    
    สำหรับการค้นหาเราจะนำคะแนน TF-IDF มาใช้เป็นคะแนนเพื่อใช้ในการค้นหาแบบ Cosine similarity และค้นหาคำใกล้เคียง (Query Expansion) เพื่อทำให้การค้นหาเจอผลลัพธ์ที่ต้องการเพิ่มมากขึ้น
    
\end{itemize}
\section{การแยกย่อยงาน และร่างแผนการดำเนินงาน}
\begin{enumerate}
    \item 	ศึกษาและค้นคว้าปัญหาของโครงการ
    \item 	เสนอหัวข้อโปรเจค 
    \item 	ค้นหาข้อมูลเกี่ยวกับเทคโนโลยีที่ใช้ในโปรเจค
    \item 	ประเมินความเป็นไปได้และกำหนดขอบเขตของโปรเจค 
    \item 	จัดเก็บ requirement จากกลุ่มผู้ใช้งาน
    \begin{enumerate}[label*=\arabic*.]
        \item   ติดต่อเจ้าหน้าที่ของหอสมุด
        \item   เก็บข้อมูลที่ต้องการแปลงเข้าสู่ระบบดิจิตอล
    \end{enumerate}
    \item 	นำเสนอโครงการครั้งที่ 1 
    \item 	ออกแบบ UX/UI
    \item 	แปลงรูปภาพเป็น Full-text
    \begin{enumerate}[label*=\arabic*.]
        \item 	นำเอกสารมาแปลงเป็นรูปภาพในรูปแบบสแกน 
        \item  ศึกษาการใช้งาน OpenCV
        \item  สร้างระบบ Image processing เพื่อทำการปรับแต่งรูปภาพและทำการปรับแต่งจนได้ระบบที่รองรับกับ Data ที่มี
        \item  	นำรูปที่ผ่านการทำ Image Processing มาเข้าสู่ระบบ OCR เพื่อแปลงข้อมูลจากรูปภาพมาเป็นข้อความในระบบดิจิตอล 
    \end{enumerate}
    \item 	นำข้อมูลที่เก็บไว้มาทำการตัดแบ่งคำภาษาไทยและหาคำสำคัญโดยใช้ TF-IDF 
    \begin{enumerate}[label*=\arabic*.]
        \item	ทำการตัดแบ่งคำ (Tokenization)
        \item	ลบ stop word ออกจากข้อมูล 
    \end{enumerate}
    \item	ทำระบบค้นหา 
    \begin{enumerate}[label*=\arabic*.]
       \item	ทำระบบค้นหาโดยใช้หลักการ Cosine Similarity
       \item	ทำการค้นหาด้วยคำใกล้เคียงโดยใช้ Word2Vec 
    \end{enumerate}    
    \item   จัดทำเว็บไซต์แพลตฟอร์ม
    \item   ทดสอบระบบ
    \item   ปรับปรุงแก้ไข
    \item   นำเสนอโปรเจค
    \end{enumerate}

\section{ตารางการดำเนินงาน}

\renewcommand{\arraystretch}{1.5}
\begin{table}[H]
    \caption{ตารางการดำเนินงาน ภาคการศึกษาที่ 1/2563}\label{tbl:work1}
    \begin{tabular}{|l|p{0.20\linewidth}|l|l|l|l|l|l|l|l|l|l|l|l|l|l|l|l|l|l|l|l|}
    \hline
    \multicolumn{22}{|c|}{ตารางการดำเนินงาน ภาคการศึกษาที่ 1/2563}                                                                                                                                                                                                                                                                                                                                                                                                                                                                                                                                                                                                                                                \\ \hline
                       &                 & \multicolumn{4}{c|}{สิงหาคม}                                                                                                                                   & \multicolumn{4}{c|}{กันยายน}                                                                                     & \multicolumn{4}{c|}{ตุลาคม}                                                                                     & \multicolumn{4}{c|}{พฤศจิกายน}                                                                                     & \multicolumn{4}{c|}{ธันวาคม}                                                                                     \\ \cline{3-22} 
    \multirow{-2}{*}{ที่} & \multicolumn{1}{c|}{\multirow{-2}{*}{หัวข้อ}} & \multicolumn{1}{c|}{1}   & \multicolumn{1}{c|}{2}                          & \multicolumn{1}{c|}{3}                          & \multicolumn{1}{c|}{4}   & \multicolumn{1}{c|}{1}   & \multicolumn{1}{c|}{2}   & \multicolumn{1}{c|}{3}   & \multicolumn{1}{c|}{4}   & \multicolumn{1}{c|}{1}   & \multicolumn{1}{c|}{2}   & \multicolumn{1}{c|}{3}   & \multicolumn{1}{c|}{4}   & \multicolumn{1}{c|}{1}   & \multicolumn{1}{c|}{2}   & \multicolumn{1}{c|}{3}   & \multicolumn{1}{c|}{4}   & \multicolumn{1}{c|}{1}   & \multicolumn{1}{c|}{2}   & \multicolumn{1}{c|}{3}   & \multicolumn{1}{c|}{4}   \\ \hline
    1                  & ศึกษาค้นคว้าหาปัญหาของโครงการ                                        & \cellcolor[HTML]{656565} & \cellcolor[HTML]{656565}                        & \cellcolor[HTML]{656565}                        &                          &                          &                          &                          &                          &                          &                          &                          &                          &                          &                          &                          &                          &                          &                          &                          &                          \\ \hline
    2                  & เสนอหัวข้อโปรเจค                                        &                          & \cellcolor[HTML]{656565}{\color[HTML]{656565} } & \cellcolor[HTML]{656565}{\color[HTML]{656565} } &                          &                          &                          &                          &                          &                          &                          &                          &                          &                          &                          &                          &                          &                          &                          &                          &                          \\ \hline
    3                  & ศึกษาและหาข้อมูลเกี่ยวกับเทคโนโลยีที่ใช้ในโปรเจค                                        &                          &                                                 & \cellcolor[HTML]{656565}                        & \cellcolor[HTML]{656565} &                          &                          &                          &                          &                          &                          &                          &                          &                          &                          &                          &                          &                          &                          &                          &                          \\ \hline
    4                  & ประเมินความเป็นไปได้และกำหนดขอบเขตของโปรเจค                                       &                          &                                                 & \cellcolor[HTML]{656565}                        & \cellcolor[HTML]{656565} &                          &                          &                          &                          &                          &                          &                          &                          &                          &                          &                          &                          &                          &                          &                          &                          \\ \hline
    5                  & จัดเก็บ requirement จากกลุ่มผู้ใช้งาน                                        &                          &                                                 &                                                 & \cellcolor[HTML]{656565} & \cellcolor[HTML]{656565} &                          &                          &                          &                          &                          &                          &                          &                          &                          &                          &                          &                          &                          &                          &                          \\ \hline
    6                  & นำเสนอโครงงานครั้งที่ 1                                        &                          &                                                 &                                                 &                          & \cellcolor[HTML]{656565} &                          &                          &                          &                          &                          &                          &                          &                          &                          &                          &                          &                          &                          &                          &                          \\ \hline
    7                  & ออกแบบ UX/UI                                        &                          &                                                 &                                                 &                          &                          & \cellcolor[HTML]{656565} & \cellcolor[HTML]{656565} &                          &                          &                          &                          &                          &                          &                          &                          &                          &                          &                          &                          &                          \\ \hline
    8                  & แปลงรูปภาพเป็น Full-text                                        &                          &                                                 &                                                 &                          &                          & \cellcolor[HTML]{656565} & \cellcolor[HTML]{656565} & \cellcolor[HTML]{656565} & \cellcolor[HTML]{656565} & \cellcolor[HTML]{656565} & \cellcolor[HTML]{656565} & \cellcolor[HTML]{656565} & \cellcolor[HTML]{656565} & \cellcolor[HTML]{656565} & \cellcolor[HTML]{656565} &                          &                          &                          &                          &                          \\ \hline
    9                  & นำข้อมูลที่เก็บไว้มาทำการตัดแบ่งคำภาษาไทยและทำการสร้าง tag โดยใช้หลักการของ TF-IDF                                        &                          &                                                 &                                                 &                          &                          &                          & \cellcolor[HTML]{656565} & \cellcolor[HTML]{656565} & \cellcolor[HTML]{656565} & \cellcolor[HTML]{656565} & \cellcolor[HTML]{656565} & \cellcolor[HTML]{656565} & \cellcolor[HTML]{656565} & \cellcolor[HTML]{656565} & \cellcolor[HTML]{656565} & \cellcolor[HTML]{656565} & \cellcolor[HTML]{656565} &                          &                          &                          \\ \hline
    10                 & นำเสนอโปรเจค                                        &                          &                                                 &                                                 &                          &                          &                          &                          &                          &                          &                          &                          &                          &                          &                          &                          &                          & \cellcolor[HTML]{656565} &                          &                          &                          \\ \hline
    11                 & จัดทำระบบการค้นหา                                        &                          &                                                 &                                                 &                          &                          &                          &                          &                          &                          &                          &                          &                          &                          &                          &                          &                          &                          & \cellcolor[HTML]{656565} & \cellcolor[HTML]{656565} & \cellcolor[HTML]{656565} \\ \hline
    \end{tabular}
    \end{table}

\begin{table}[H]
\caption{ตารางการดำเนินงาน ภาคการศึกษาที่ 2/2563}\label{tbl:work2}
\begin{tabular}{|l|p{0.35\linewidth}|l|l|l|l|l|l|l|l|l|l|l|l|l|l|l|l|}
\hline
\multicolumn{18}{|c|}{ตารางการดำเนินงาน ภาคการศึกษาที่ 2/2563}                                                                                                                                                                                                                                                                                                                                                                                                                                                                 \\ \hline
                   &                    & \multicolumn{4}{c|}{มกราคม}                                                                                     & \multicolumn{4}{c|}{กุมภาพันธ์}                                                                                     & \multicolumn{4}{c|}{มีนาคม}                                                                                     & \multicolumn{4}{c|}{เมษายน}                                                                                     \\ \cline{3-18} 
\multirow{-2}{*}{ที่} & \multicolumn{1}{c|}{\multirow{-2}{*}{หัวข้อ}} & 1                        & 2                        & 3                        & 4                        & 1                        & 2                        & 3                        & 4                        & 1                        & 2                        & 3                        & 4                        & 1                        & 2                        & 3                        & 4                        \\ \hline
1                  & จัดทำระบบการค้นหา                   & \cellcolor[HTML]{656565} & \cellcolor[HTML]{656565} & \cellcolor[HTML]{656565} & \cellcolor[HTML]{656565} & \cellcolor[HTML]{656565} & \cellcolor[HTML]{656565} & \cellcolor[HTML]{656565} &                          &                          &                          &                          &                          &                          &                          &                          &                          \\ \hline
2                  & จัดทำเว็บไซต์แพลตฟอร์ม                   & \cellcolor[HTML]{656565} & \cellcolor[HTML]{656565} & \cellcolor[HTML]{656565} & \cellcolor[HTML]{656565} & \cellcolor[HTML]{656565} & \cellcolor[HTML]{656565} & \cellcolor[HTML]{656565} & \cellcolor[HTML]{656565} &                          &                          &                          &                          &                          &                          &                          &                          \\ \hline
3                  & ทดสอบระบบ                   &                          &                          &                          &                          & \cellcolor[HTML]{656565} & \cellcolor[HTML]{656565} & \cellcolor[HTML]{656565} & \cellcolor[HTML]{656565} & \cellcolor[HTML]{656565} & \cellcolor[HTML]{656565} & \cellcolor[HTML]{656565} & \cellcolor[HTML]{656565} & \cellcolor[HTML]{656565} & \cellcolor[HTML]{656565} &                          &                          \\ \hline
4                  & ปรับปรุงแก้ไข                   &                          &                          &                          &                          & \cellcolor[HTML]{656565} & \cellcolor[HTML]{656565} & \cellcolor[HTML]{656565} & \cellcolor[HTML]{656565} & \cellcolor[HTML]{656565} & \cellcolor[HTML]{656565} & \cellcolor[HTML]{656565} & \cellcolor[HTML]{656565} & \cellcolor[HTML]{656565} & \cellcolor[HTML]{656565} &                          &                          \\ \hline
5                  & นำเสนอโปรเจค                   &                          &                          &                          &                          &                          &                          &                          &                          &                          &                          &                          &                          &                          &                          & \cellcolor[HTML]{656565} & \cellcolor[HTML]{656565} \\ \hline
\end{tabular}
\end{table}

\subsection{ผลการดำเนินงานในภาคการศึกษาที่ 1}
\begin{itemize}
    \item ทำระบบ Image processing สำหรับการเตรียมรูปภาพสำหรับการแปลงข้อมูลเป็นดิจิตอล
    \item ทำ API ในการตัดคำและจัดการ stop word สำหรับการเตรียมการ text processing
    \item ทำระบบ Term Frequency-Inverse Document Frequency สำหรับการค้นหาคำสำคัญเพื่อสร้าง tag 
    \item ทำส่วนของการทำการค้นหาข้อมูลเบื้องต้น
\end{itemize}

\subsection{ผลการดำเนินงานในภาคการศึกษาที่ 2}
\begin{itemize}
    \item ทำระบบค้นหาให้เสร็จสิ้น
    \item ปรับปรุงระบบค้นหาให้ตอบโจทย์มากยิ่งขึ้น
    \item ทำเว็บไซต์ platform ทั้งฝั่ง frontend และ backend
\end{itemize}
