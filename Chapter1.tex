\chapter{บทนำ}

\section{คำสำคัญ}

Natural language processing, RESTful Service, Optical character recognition, Image Processing, Information retrieval, Term Frequency-Inverse Document Frequency, Word2Vec, Word Embedded 

\section{ความสำคัญของปัญหา}

นับตั้งแต่การก่อตั้งหอสมุดมหาวิทยาลัยเทคโนโลยีพระจอมเกล้าธนบุรี ได้มีการเก็บรวบรวมองค์ความรู้จากประสบการณ์การทำงานของคณะอาจารย์ ผู้เชี่ยวชาญในทางด้านศาสตร์ต่าง ๆ ในรูปแบบลายมือและสื่อสิ่งพิมพ์ไม่ว่าจะเป็น หนังสือ เอกสาร รวมถึงบันทึกเหตุการณ์ในอดีตในรูปของจดหมายเหตุเพื่อส่งต่อประวัติศาสตร์ความรู้ไปยังคนรุ่นหลังโดยมีการจัดเก็บอยู่ภายในหอจดหมายเหตุที่มีเจ้าหน้าที่บรรณารักษ์เป็นผู้ดูแล และเนื่องจากการที่ เอกสาร หนังสือยังไม่ได้มีการจัดเก็บในรูปแบบดิจิตอลทำให้เมื่อมีบุคคลภายนอกที่ต้องการข้อมูลเพื่อนำไปทำกิจกรรมต่าง ๆ ไม่ว่าจะเป็นการทำวิจัย รายงาน หรือหาข้อมูลเพื่อประกอบการประชุมก็ตามแต่ ก็จำเป็นที่จะต้องมาติดต่อเจ้าหน้าที่บรรณารักษ์ผู้ดูแลเพื่อที่จะให้เจ้าหน้าที่บรรณารักษ์ทำการค้นหาหนังสือที่มีเนื้อหาตามที่เราต้องการ ซึ่งการค้นหาข้อมูลที่ต้องการนั้นเจ้าหน้าที่จะต้องทำการค้นหาด้วยระบบมือทำให้การค้นหาข้อมูลดำเนินการไปอย่างล่าช้า นอกจากนั้นวิธีการหาข้อมูลของเจ้าหน้าที่บรรณารักษ์จะเลือกตรวจสอบข้อมูลของหนังสือจากการดูสารบัญทำให้ข้อมูลที่ได้รับมาอาจจะตกหล่นจากข้อมูลเล่มอื่นได้ 

เพื่ออำนวยความสะดวกให้กับบรรณารักษ์ในการสืบค้นข้อมูลและทำให้การบริการในการสืบค้นเอกสารต่าง ๆ และให้บุคคลภายนอกสามารถทำการค้นหาข้อมูลได้ด้วยตนเองครบถ้วนทางคณะผู้จัดทำโครงการจึงได้พัฒนาระบบการจัดเก็บเอกสารและระบบการค้นหาโดยการใช้เครื่องมือในการทำ OCR เพื่อแปลงเอกสารให้อยู่ในรูปแบบของเอกสาร digital และหาคำสำคัญในการสร้าง tag ด้วยวิธี Term Frequency - Inverse Document Frequency เพื่อเพิ่มประสิทธิ์ภาพให้กับการค้นหา 

\section{ประเภทของโครงงาน}

นำเสนอความต้องการของผู้มีส่วนได้ส่วนเสียเฉพาะกลุ่ม 

\section{วิธีการที่นำเสนอ}
\section{วัตถุประสงค์}
\section{ขอบเขตของงานวิจัย}
\section{เนื้อหาทางวิศวกรรมที่เป็นต้นฉบับ}
\section{การแยกย่อยงาน และร่างแผนการดำเนินงาน}
\section{ตารางการดำเนินงาน}
\subsection{ผลการดำเนินงานในภาคการศึกษาที่ 1}
\subsection{ผลการดำเนินงานในภาคการศึกษาที่ 2}
