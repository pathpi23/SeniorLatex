\chapter{สรุปผล}

\section{ผลการดำเนินงาน}
ในภาคการเรียนที่ 1/2563 ทางคณะผู้จัดทำได้ทำการนำเสนอหัวข้อโปรเจค ศึกษาและรวบรวมข้อมูลต่างๆ เก็บ Requirment จากบรรณารักษ์ และได้ออกแบบเว็บไซต์ User Interface, โครงสร้างฐานข้อมูลและวิธีต่างๆในการสร้างเว็บไซต์ที่จะทำการแปลงรูปภาพให้อยู่ในรูปแบบดิจิทัล 
และทำการศึกษา เรียนรู้และออกแบบวิธีการเตรียมข้อมูลรูปภาพก่อนที่จะนำไปทำการ OCR สร้างระบบเตรียมข้อมูลตัวหนังสือ อย่างการตัดคำ และการแก้ไขคำผิด เพื่อเตรียมข้อมูลที่ได้จากการทำ OCR ให้อยู่ในรูปแบบที่เหมาะสมสำหรับระบบการค้นหา และสร้างระบบค้นหาคำสำคัญด้วยหลักการของ TF-IDF เพื่อใช้สร้างแท็ก ของตัวหนังสือ

ณ เวลาปัจจุบันในภาคเรียนที่ 2/2563 ทางคณะผู้จัดทำได้วางแผนที่จะสร้างเว็บไซต์ จัดทำระบบการค้นหา และโมเดล Word2Vec ทำการประเมินระบบการออกแบบ User Interface การเตรียมข้อมูลรูปภาพ ระบบการแก้คำผิด และระบบการค้นหา 

\begin{table}[H]
\caption{ตารางสรุปผลลัพธ์การดำเนินงาน}\label{tbl:milestone}
\begin{tabular}{|p{0.5\linewidth}|l|l|l|}
\hline
\multicolumn{1}{|c|}{แผนการดำเนินการ}                                                & \multicolumn{1}{c|}{ยังไม่ดำเนินการ} & \multicolumn{1}{c|}{กำลังดำเนินการ} & \multicolumn{1}{c|}{เสร็จสิ้น} \\ \hline
ศึกษาค้นคว้าหาปัญหาของโครงการ                                                        &                                      &                                     & \cellcolor[HTML]{92D050}       \\ \hline
เสนอหัวข้อโปรเจค                                                                     &                                      &                                     & \cellcolor[HTML]{92D050}       \\ \hline
ศึกษาและหาข้อมูลเกี่ยวกับเทคโนโลยีที่ใช้ในโปรเจค                                     &                                      &                                     & \cellcolor[HTML]{92D050}       \\ \hline
ประเมินความเป็นไปได้และกำหนดขอบเขตของโปรเจค                                          &                                      &                                     & \cellcolor[HTML]{92D050}       \\ \hline
จัดเก็บ requirement จากกลุ่มผู้ใช้งาน                                                &                                      &                                     & \cellcolor[HTML]{92D050}       \\ \hline
นำเสนอโครงงานครั้งที่ 1                                                              &                                      &                                     & \cellcolor[HTML]{92D050}       \\ \hline
ออกแบบ UX/UI                                                                         &                                      &                                     & \cellcolor[HTML]{92D050}       \\ \hline
การแปลงรูปภาพให้อยู่ในรูปแบบดิจิทัล                                                        &                                      &                                     & \cellcolor[HTML]{92D050}       \\ \hline
นำข้อมูลที่เก็บไว้มาทำการตัดแบ่งคำภาษาไทยและทำการสร้างแท็ก โดยใช้หลักการของ TF-IDF &                                      &                                     & \cellcolor[HTML]{92D050}       \\ \hline
จัดทำระบบการค้นหา                                                                    &                                      & \cellcolor[HTML]{FFC000}            &                                \\ \hline
จัดทำเว็บไซต์แพลตฟอร์ม                                                               &                                      &                                     & \cellcolor[HTML]{92D050}       \\ \hline
ทดสอบระบบ                                                                            &                                      & \cellcolor[HTML]{FFC000}            &                                \\ \hline
ปรับปรุงแก้ไข                                                                        &                                      & \cellcolor[HTML]{FFC000}            &                                \\ \hline
นำเสนอโปรเจค                                                                         & \cellcolor[HTML]{FF0000}                & \cellcolor[HTML]{FFFFFF}               &                                \\ \hline
\end{tabular}
\end{table}
\section{สรุปผลการดำเนินงาน}
จากผลลัพธ์จากบทที่ 4 ในส่วนของการแปลงข้อมูลในหนังสือ (Digitization) ให้อยู่ในรูปแบบของหนังสือดิจิทัลนั้นเราเลือกใช้ชุดข้อมูลชุดที่่ 1 ที่่ผ่านการเตรียมข้อมูลรูปภาพด้วยวิธีที่ 2 (การลบพื้นหลัง) และการแก้ไขคำผิด (Correction) ซึ่งมีเปอร์เซ็นต์ความถูกต้องสูงสุดอยู่ที่ 76.61 \%
ซึ่งเปอร์เซ็นต์ผลลัพธ์ที่ได้มานี้เกิดขึ้นมาจากการสุ่มจากข้อมูลที่หลากหลาย ซึ่งในครั้งนี้การที่เปอร์เซ็นต์ต่ำกว่าเมื่อช่วงภาคเรียนที่ 1/2563 ที่ใช้เช็คคำส่วนใหญ่เช็คจากเล่มรายงานที่สุ่มยังไม่หลากหลายและไม่ได้รวมกับเปอร์เซ็นต์คำที่ไม่สามารถแปลงเป็นดิจิทัล และเนื่องจากข้อมูลที่นำมาใช้สำหรับการทดสอบนั้นเป็นหน้าที่ติดรูปภาพเยอะทำให้จำนวนคำที่ได้มาทดสอบบางหน้านั้นมีไม่เยอะ
รวมถึงการแก้ไขคำผิดที่ทำให้คำเฉพาะหรือชื่อคนที่บางครั้งตรวจไม่พบว่าเป็นคำเฉพาะและทำให้การแก้ไขนั้นผิดพลาด และในส่วนการค้นหานั้นยังอยู่ในขั้นตอนการดำเนินการ ซึ่งยังไม่ได้นำมาทดสอบ
\section{ปัญหาที่พบและการแก้ไข}
\subsection{ปัญหาหน้าสีอ่านยาก}
เนื่องจากหนังสือแต่ละเล่มมีลักษณะที่แตกต่างกันในเรื่องของสีของกระดาษและตัวอักษร ลักษณะการสแกนรูปภาพจึงทำให้การนำประโยคข้อความที่ถูกตัดออกมาทำ OCR แล้วเกิดความผิดผลาดเยอะ

\underline{การแก้ไข}

	ทำการแก้ไขกระบวนการการเตรียมข้อมูลรูปภาพ จากการพยายามข้ามหน้าสีเป็นพัฒนาการลบพื้นหลังในรูปแบบใหม่
\subsection{ปัญหาการหมุนไม่ตรง}
เนื่องจากภาษาไทยมีสระวรรณยุกต์ด้านบนต่อกันสูงสุดต่อกันถึง 2 ชั้นนอกจากนั้นยังมีสระวรรณยุกต์ด้านล่างทำให้บางทีไม่สามารถแยกข้อความแต่ละบรรทัดออกมาได้อย่างสมบูรณ์จึงทำให้มีการหมุนที่ผิดพลาดเกิดขึ้น

\underline{การแก้ไข}

	ทำการแก้ไขกระบวนการการเตรียมข้อมูลรูปภาพ ปรับเปลี่ยนวิธีการหมุนเป็นการหา Arctan ที่จุดขอบด้านบนที่ทำให้การหมุนมีข้อผิดพลาดน้อยลง
\subsection{ปัญหาการแก้ไขคำผิด}
เนื่องจากการแก้ไขคำผิดยังไม่สามารถแก้คำผิดรูปแบบคำเฉพาะได้อย่างเช่นชื่อคน หรือชื่อสถานที่ หรืออาจจะคิดว่าคำเฉพาะนั้นผิดพลาดและทำการแก้ไขให้อัตโนมัติทำให้คำที่ได้รับออกมาเกิดข้อผิดพลาดขึ้น

\underline{การแก้ไข}

	ให้ผู้ใช้งานได้ตรวจสอบและแก้ไขได้เองก่อจะเพิ่มข้อมูลเข้าสู่ระบบและเพิ่มข้อมูลคำเฉพาะบางส่วนลงไป
\subsection{ปัญหาเรื่องระยะเวลาในการเพิ่มข้อมูลหนังสือ}
เนื่องจากการเพิ่มข้อมูลมีขั้นตอนจำนวนมากและใช้เวลานานผู้จัดทำจึงออกแบบโครงสร้างเป็นระบบ Thread เพื่อที่จะให้เซิฟเวอร์ตอบกลับไปยังผู้ใช้งานและเปิด Thread ในการทำงานไม่ว่าจะเป็น OCR หรือการทำการเตรียมข้อมูลตัวหนังสือ เพื่อเพิ่มความเร็วในการทำงานแต่ก็ไม่สามารถลดเวลาการทำงานลงได้

\underline{การแก้ไข}

	ทำแก้ไขโดยการ Spawn Process แทนขึ้นมาเนื่องจาก python นั้นเวลาเปิด Thread ยังคงใช้ core เดียวในการประมวลลัพธ์จึงต้องเปลี่ยนมาเป็นการ Spawn Process ที่แยกการใช้ core ของหน่วยประมวลผลทำให้สามารถลดเวลาในการเพิ่มข้อมูลหนังสือได้
\section{ข้อจำกัดและข้อเสนอแนะ}
\begin{enumerate}
    \item การแก้ไขคำเฉพาะยังไม่สามารถทำได้ถึงแม้จะมีการเพิ่มคำศัพท์เฉพาะลงไปก็ตามแต่ก็ยังต้องให้มนุษย์เป็นผู้ตรวจสอบอีกรอบเพื่อความถูกต้อง
    \item เนื่องจากลักษณะหนังสือแต่และเล่มแตกต่างกันทำให้การเตรียมข้อมูลรูปภาพที่อาจจะไม่ได้ส่งผลลัพธ์ที่ดีที่สุดให้กับหนังสือทุกประเภททำให้ประสิทธิ์ภาพในหนังสือบางเล่มน้อยกว่าหรือมากกว่าอีกเล่มได้
    \item ลักษณะแสงของการสแกนหนังสือเนื่องจากไฟล์ที่ได้รับมาไม่มีการควบคุมในการสแกนหนังสือเข้ามาจึงทำให้ไฟล์มีความสว่างที่ไม่เท่ากันขนาดและการจัดวางที่ไม่เหมือนกันทำให้ประสิทธิ์ภาพของแต่ละเล่มอาจจะไม่เท่าเดิม
\end{enumerate}
