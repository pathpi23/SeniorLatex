%%%%%% Run at command line, run
%%%%%% xelatex grad-sample.tex 
%%%%%% for a few times to generate the output pdf file
\documentclass[12pt,oneside,openright,a4paper]{cpe-thai-project}
\usepackage{enumitem}
\usepackage{multirow}
\usepackage[table,xcdraw]{xcolor}
\usepackage{float}
\usepackage{url}
\usepackage{graphicx}
\usepackage{makecell}
\usepackage{longtable}
\usepackage{enumitem}
\usepackage{caption}
\XeTeXlinebreaklocale "th_TH"
\XeTeXlinebreakskip = 0pt plus 1pt
\usepackage{polyglossia}

\graphicspath{ {./picture/} }

\captionsetup[longtable]{labelfont=bf,labelsep=space,justification=raggedright,singlelinecheck=false}


\setdefaultlanguage{thai}
\setotherlanguage{english}
\newfontfamily\thaifont[Script=Thai,Scale=1.23]{TH Sarabun New}
\defaultfontfeatures{Mapping=tex-text,Scale=1.23,LetterSpace=0.0}
\setmainfont[Scale=1.23,LetterSpace=0,WordSpace=1.0,FakeStretch=1.0]{TH Sarabun New}
% \defaultfontfeatures{Mapping=tex-text,Scale=1.23,LetterSpace=0.0}
% \setmainfont[Scale=1.23,LetterSpace=0,WordSpace=1.0,FakeStretch=1.0]{TH Sarabun New}
%\setmathfont(Digits)[Scale=1.0,LetterSpace=0,FakeStretch=1.0]{Times New Roman}

%%%%%%%%%%%%%%%%%%%%%%%%%%%%%%%%%%%%%%%%%%%%%%%%%%%%%%%%%%%%%%%%%%%
% Customize below to suit your needs 
% The ones that are optional can be left blank. 
%%%%%%%%%%%%%%%%%%%%%%%%%%%%%%%%%%%%%%%%%%%%%%%%%%%%%%%%%%%%%%%%%%%
% First line of title
\def\disstitleone{Project No. 67}   
% Second line of title
\def\disstitletwo{ระบบจัดเก็บและจัดการเอกสารภายในหอบรรณสารสนเทศ }  
% Three line of title
\def\disstitlethree{(KMUTT Archives Management Platform) }   

% Your first name and lastname
\def\dissauthor{Mr.Akarapon Boonsermsakul}   % 1st member
%%% Put other group member names here ..
\def\dissauthortwo{Ms.Thanaporn Pitianusorn}   % 2nd member (optional)
\def\dissauthorthree{Mr.Annop Kongsombatcharoen}   % 3rd member (optional)


% The degree that you're persuing..
\def\dissdegree{Bachelor of Engineering} % Name of the degree
\def\dissdegreeabrev{B.Eng} % Abbreviation of the degree
\def\dissyear{2020}                   % Year of submission
\def\thaidissyear{2563}               % Year of submission (B.E.)

%%%%%%%%%%%%%%%%%%%%%%%%%%%%%%%%%%%%%%%%%%%%
% Your project and independent study committee..
%%%%%%%%%%%%%%%%%%%%%%%%%%%%%%%%%%%%%%%%%%%%
\def\dissadvisor{Asst.Prof. Suthathip Manee, Ph.D.}  % Advisor
%%% Leave it empty if you have no Co-advisor
\def\disscoadvisor{}  % Co-advisor
\def\disscommitteetwo{Dr.Prapong Prechaprapranwong, Ph.D.}  % 3rd committee member (optional)
\def\disscommitteethree{Asst.Prof.Sanan Srakaew}   % 4th committee member (optional) 
\def\disscommitteefour{Asst.Prof.Surapont Toomnark}    % 5th committee member (optional) 

\def\worktype{Project} %%  Project or Independent study
\def\disscredit{3}   %% 3 credits or 6 credits


\def\fieldofstudy{Computer Engineering} 
\def\department{Computer Engineering} 
\def\faculty{Engineering}

\def\thaifieldofstudy{วิศวกรรมคอมพิวเตอร์} 
\def\thaidepartment{วิศวกรรมคอมพิวเตอร์} 
\def\thaifaculty{วิศวกรรมศาสตร์}
 
\def\appendixnames{Appendix} %%% Appendices or Appendix

\def\thaiworktype{ปริญญานิพนธ์} %  Project or research project % 
\def\thaidisstitleone{ระบบจัดเก็บและจัดการเอกสารภายในหอบรรณสารสนเทศ}
\def\thaidisstitletwo{KMUTT Archives Management Platform}
\def\thaidissauthor{นายอัครพล บุญเสริมศักดิ์กุล}
\def\thaidissauthortwo{นางสาวธนพร ปิติอนุสรณ์} %Optional
\def\thaidissauthorthree{นายอรรณพ กองสมบัติเจริญ} %Optional

\def\thaidissadvisor{ผศ.ดร.สุธาทิพย์ มณีวงศ์วัฒนา}
%% Leave this empty if you have no co-advisor
\def\thaidissdegree{วิศวกรรมศาสตรบัณฑิต}

% Change the line spacing here...
\linespread{1.15}

%%%%%%%%%%%%%%%%%%%%%%%%%%%%%%%%%%%%%%%%%%%%%%%%%%%%%%%%%%%%%%%%
% End of personal customization.  Do not modify from this part 
% to \begin{document} unless you know what you are doing...
%%%%%%%%%%%%%%%%%%%%%%%%%%%%%%%%%%%%%%%%%%%%%%%%%%%%%%%%%%%%%%%%


%%%%%%%%%%%% Dissertation style %%%%%%%%%%%
%\linespread{1.6} % Double-spaced  
%%\oddsidemargin    0.5in
%%\evensidemargin   0.5in
%%%%%%%%%%%%%%%%%%%%%%%%%%%%%%%%%%%%%%%%%%%
%\renewcommand{\subfigtopskip}{10pt}
%\renewcommand{\subfigbottomskip}{-5pt} 
%\renewcommand{\subfigcapskip}{-6pt} %vertical space between caption
%                                    %and figure.
%\renewcommand{\subfigcapmargin}{0pt}

\renewcommand{\topfraction}{0.85}
\renewcommand{\textfraction}{0.1}

\newtheorem{theorem}{Theorem}
\newtheorem{lemma}{Lemma}
\newtheorem{corollary}{Corollary}
\setlength{\parskip}{1em}

\def\QED{\mbox{\rule[0pt]{1.5ex}{1.5ex}}}
\def\proof{\noindent\hspace{2em}{\itshape Proof: }}
\def\endproof{\hspace*{\fill}~\QED\par\endtrivlist\unskip}
%\newenvironment{proof}{{\sc Proof:}}{~\hfill \blacksquare}
%% The hyperref package redefines the \appendix. This one 
%% is from the dissertation.cls
%\def\appendix#1{\iffirstappendix \appendixcover \firstappendixfalse \fi \chapter{#1}}
%\renewcommand{\arraystretch}{0.8}
%%%%%%%%%%%%%%%%%%%%%%%%%%%%%%%%%%%%%%%%%%%%%%%%%%%%%%%%%%%%%%%%
%%%%%%%%%%%%%%%%%%%%%%%%%%%%%%%%%%%%%%%%%%%%%%%%%%%%%%%%%%%%%%%%
\begin{document}
\pdfstringdefDisableCommands{%
\let\MakeUppercase\relax
}
\begin{center}
\includegraphics[width=2.8cm]{logo02.jpg}
\end{center}
\vspace*{-1cm}
\maketitlepage
\makesignaturepage 

%%%%%%%%%%%%%%%%%%%%%%%%%%%%%%%%%%%%%%%%%%%%%%%%%%%%%%%%%%%%%%
%%%%%%%%%%%%%%%%%%%%%% English abstract %%%%%%%%%%%%%%%%%%%%%%%
%%%%%%%%%%%%%%%%%%%%%%%%%%%%%%%%%%%%%%%%%%%%%%%%%%%%%%%%%%%%%%
\abstract

KMUTT's library have collected the archive of valued documents. 
Because these document have not transformed into digital form, 
there is vital problem in searching for information in these 
document for librarian and patrons. In this project, we developed 
web platform to digitize these document into digital format 
and implement the search function that facilitate the librarian 
and patron to search for information. The platform consists of 
2 components. The first part is importing documents and 
digitization. In this step, we applied image processing 
techniques such as Morphology Transformation to preprocess
the images of documents and transform the images to full text 
data by using Tesseract. After getting the text files, we tokenize 
the text into words by using the Deepcut library and find the 
significant words of the document by using the TF-IDF algorithm. 
In the second part, we start by getting the input from the user 
and use the word2Vec model to find a similar word. And take 
input and similar words to get the TF-IDF score that we 
generate at first to find the best document for the input word.  
For comparison correct of  OCR is 74.75\% , use OCR and correction is 76.61\%.
\begin{flushleft}
\begin{tabular*}{\textwidth}{@{}lp{0.8\textwidth}}
\textbf{Keywords}: & Natural language processing / RESTful Service / Optical character recognition / Image Processing / Information retrieval / Term Frequency-Inverse Document Frequency / Word2Vec / Word Embedded 
\end{tabular*}
\end{flushleft}
\endabstract

%%%%%%%%%%%%%%%%%%%%%%%%%%%%%%%%%%%%%%%%%%%%%%%%%%%%%%%%%%%%%%
%%%%%%%%%% Thai abstract here %%%%%%%%%%%%%%%%%%%%%%%%%%%%%%%%%
%%%%%%%%%%%%%%%%%%%%%%%%%%%%%%%%%%%%%%%%%%%%%%%%%%%%%%%%%%%%%%
% {\newfontfamily\thaifont{TH Sarabun New:script=thai}[Scale=1.3]
% \XeTeXlinebreaklocale "th_TH"	
% \thaifont
\thaiabstract

การจะสืบค้นข้อมูลจากเอกสารหรือชั้นหนังสือที่มีการรวบรวมข้อมูลไว้ตั้งแต่อดีตนั้นเป็น
ปัญหาอย่างหนึ่งของเจ้าหน้าที่บรรณารักษ์ที่ต้องทำการดูแลเอกสารเหล่านี้ เนื่องจาก
การที่ยังไม่มีการเก็บหนังสือและเอกสารให้อยู่ในรูปแบบของข้อมูลดิจิทัลทำให้ต้อง
สืบค้นโดยการค้นหาเอกสารและหนังสือแต่ละเล่มโดยการดูจากเนื้อหาสารบัญเพื่อให้
ได้หนังสือที่ตรงกับข้อมูลที่ต้องการมากที่สุด ซึ่งการที่ค้นหาจากหน้าสารบัญของ
หนังสือแต่ละเล่มก็จะทำให้การค้นหาเป็นไปอย่างล่าช้า และบางครั้งการดูเพียง
แค่สารบัญก็อาจจะทำให้ได้หนังสือที่ไม่ตรงกับความต้องการของผู้ที่เข้ามายืมหนังสือ 
ในโครงการนี้เราได้ทำการพัฒนาการระบบจัดเก็บและค้นหาเอกสารอิเล็กทรินิกส์ 
โดยแบ่งออกเป็น 2 ขั้นตอนคือ การนำเข้าข้อมูล  และการสร้างระบบค้นหา 
โดยขั้นตอนการนำเข้าข้อมูล เราจะเริ่มจากการเตรียมข้อมูลรูปภาพ
เพื่อเตรียมข้อมูลรูปภาพที่ได้มา ก่อนจะนำไปผ่านกระบวนการ OCR 
เพื่อแปลงรูปภาพเหล่านี้ให้อยู่ในรูปของข้อมูลดิจิทัล โดยการเก็บข้อมูลในรูปแบบของ 
Information Retrieval เพื่อช่วยให้ความเร็วการค้นหามีประสิทธิภาพมากยิ่งขึ้น 
และนำข้อมูลมาทำการตัดคำ และเช็คคำผิด จากนั้นจะนำมาหาคำสำคัญของหนังสือหรือเอกสารนั้น ๆ
โดยการใช้การหาคะแนนแบบ TF-IDF ส่วนการสร้างระบบการค้นหาจะเริ่มจากรับคำค้นหามาจากผู้
ใช้และทำการนำคำที่ได้ไปเข้าโมเดล word2Vec เพื่อหาคำที่ใกล้เคียง 
จากนั้นนำคำใกล้เคียงและคำค้นหาไปดึงคะแนน TF-IDF ที่เก็บไว้เพื่อค้นหาว่า
มีเอกสารหรือหนังสือเล่มไหนที่มีคะแนนที่ตรงและใกล้เคียงกับคำค้นหามากที่สุด
โดยผลลัพธ์จากการทำ OCR ถูกต้อง 74.75 \%และเมื่อนำมาผ่านกระบวนการแก้คำผิดได้ความถูกต้องอยู่ที่ 76.61\%
\begin{flushleft}
\begin{tabular*}{\textwidth}{@{}lp{0.8\textwidth}}
 & \\

\textbf{คำสำคัญ}: & Natural language processing / RESTful Service / Optical character recognition / Image Processing / Information retrieval / Term Frequency-Inverse Document Frequency / Word2Vec / Word Embedded 
\end{tabular*}
\end{flushleft}
\endabstract
% }

%%%%%%%%%%%%%%%%%%%%%%%%%%%%%%%%%%%%%%%%%%%%%%%%%%%%%%%%%%%%
%%%%%%%%%%%%%%%%%%%%%%% Acknowledgments %%%%%%%%%%%%%%%%%%%%
%%%%%%%%%%%%%%%%%%%%%%%%%%%%%%%%%%%%%%%%%%%%%%%%%%%%%%%%%%%%
\preface
ขอขอบคุณนางสาวอารยา ศรีบัวบาน เจ้าหน้าที่หอบรรณสารสนเทศและ ผศ.ดร.สุธาทิพย์ มณีวงศ์วัฒนา อาจารย์ที่ปรึกษารวมทั้งเจ้าหน้าที่ภายในหอสมุดมหาวิทยาลัยเทคโนโลยีพระจอมเกล้าธนบุรีที่เสียสละเวลาให้ความรู้ความเข้าใจ ทั้งในเรื่องการเก็บข้อมูลและคอยแนะนำวิธีการจัดการกับปัญหาต่างๆที่เกิดขึ้น นำมาสู่การทำหัวข้อปริญญานิพนธ์ฉบับนี้ให้สำเร็จตามที่ต้องการ 

%%%%%%%%%%%%%%%%%%%%%%%%%%%%%%%%%%%%%%%%%%%%%%%%%%%%%%%%%%%%%
%%%%%%%%%%%%%%%% ToC, List of figures/tables %%%%%%%%%%%%%%%%
%%%%%%%%%%%%%%%%%%%%%%%%%%%%%%%%%%%%%%%%%%%%%%%%%%%%%%%%%%%%%
% The three commands below automatically generate the table 
% of content, list of tables and list of figures
\tableofcontents                    
\listoftables
\listoffigures                      

%%%%%%%%%%%%%%%%%%%%%%%%%%%%%%%%%%%%%%%%%%%%%%%%%%%%%%%%%%%%%%
%%%%%%%%%%%%%%%%%%%%% List of symbols page %%%%%%%%%%%%%%%%%%%
%%%%%%%%%%%%%%%%%%%%%%%%%%%%%%%%%%%%%%%%%%%%%%%%%%%%%%%%%%%%%%
% You have to add this manually..
% \listofsymbols
% \begin{flushleft}
% \begin{tabular}{@{}p{0.07\textwidth}p{0.7\textwidth}p{0.1\textwidth}}
% \textbf{SYMBOL}  & & \textbf{UNIT} \\[0.2cm]
% $\alpha$ & Test variable\hfill & m$^2$ \\
% $\lambda$ & Interarival rate\hfill &  jobs/second\\
% $\mu$ & Service rate\hfill & jobs/second\\
% \end{tabular}
% \end{flushleft}
%%%%%%%%%%%%%%%%%%%%%%%%%%%%%%%%%%%%%%%%%%%%%%%%%%%%%%%%%%%%%%
%%%%%%%%%%%%%%%%%%%%% List of vocabs & terms %%%%%%%%%%%%%%%%%
%%%%%%%%%%%%%%%%%%%%%%%%%%%%%%%%%%%%%%%%%%%%%%%%%%%%%%%%%%%%%%
% You also have to add this manually..

% \listofvocab
% \begin{flushleft}
% \begin{tabular}{@{}p{1in}@{=\extracolsep{0.5in}}l}
% API &   \\
% CNN & Convolutional Neural Network \\
% ER & Entity Relationship \\
% FK & Foreign Key \\
% IDF & Inverse Document Frequency \\
% INT & Interger \\
% IR & Information Retrieval \\
% JWT & JSON Web Token \\
% LSTM & Long Short Term Memory \\
% MVC & Model View Controller \\
% NLP & Natural Language Processing \\
% NLTK & \\ 
% OCR & Optical Character Recognization \\
% OpenCV & \\
% PDF & \\
% PK & Primary Key \\
% TF & Term Frequency \\
% UI & User Interface \\
% UX & User Experience \\

% \end{tabular}
% \end{flushleft}

%\setlength{\parskip}{1.2mm}

%%%%%%%%%%%%%%%%%%%%%%%%%%%%%%%%%%%%%%%%%%%%%%%%%%%%%%%%%%%%%%%
%%%%%%%%%%%%%%%%%%%%%%%% Main body %%%%%%%%%%%%%%%%%%%%%%%%%%%%
%%%%%%%%%%%%%%%%%%%%%%%%%%%%%%%%%%%%%%%%%%%%%%%%%%%%%%%%%%%%%%%
\chapter{บทนำ}

\section{คำสำคัญ}

Natural language processing, RESTful Service, Optical character recognition, Image Processing, Information retrieval, Term Frequency-Inverse Document Frequency, Word2Vec, Word Embedded 

\section{ความสำคัญของปัญหา}

นับตั้งแต่การก่อตั้งหอสมุดมหาวิทยาลัยเทคโนโลยีพระจอมเกล้าธนบุรี ได้มีการเก็บรวบรวมองค์ความรู้จากประสบการณ์การทำงานของคณะอาจารย์ ผู้เชี่ยวชาญในทางด้านศาสตร์ต่าง ๆ ในรูปแบบลายมือและสื่อสิ่งพิมพ์ไม่ว่าจะเป็น หนังสือ เอกสาร รวมถึงบันทึกเหตุการณ์ในอดีตในรูปของจดหมายเหตุเพื่อส่งต่อประวัติศาสตร์ความรู้ไปยังคนรุ่นหลังโดยมีการจัดเก็บอยู่ภายในหอจดหมายเหตุที่มีเจ้าหน้าที่บรรณารักษ์เป็นผู้ดูแล และเนื่องจากการที่ เอกสาร หนังสือยังไม่ได้มีการจัดเก็บในรูปแบบดิจิตอลทำให้เมื่อมีบุคคลภายนอกที่ต้องการข้อมูลเพื่อนำไปทำกิจกรรมต่าง ๆ ไม่ว่าจะเป็นการทำวิจัย รายงาน หรือหาข้อมูลเพื่อประกอบการประชุมก็ตามแต่ ก็จำเป็นที่จะต้องมาติดต่อเจ้าหน้าที่บรรณารักษ์ผู้ดูแลเพื่อที่จะให้เจ้าหน้าที่บรรณารักษ์ทำการค้นหาหนังสือที่มีเนื้อหาตามที่เราต้องการ ซึ่งการค้นหาข้อมูลที่ต้องการนั้นเจ้าหน้าที่จะต้องทำการค้นหาด้วยระบบมือทำให้การค้นหาข้อมูลดำเนินการไปอย่างล่าช้า นอกจากนั้นวิธีการหาข้อมูลของเจ้าหน้าที่บรรณารักษ์จะเลือกตรวจสอบข้อมูลของหนังสือจากการดูสารบัญทำให้ข้อมูลที่ได้รับมาอาจจะตกหล่นจากข้อมูลเล่มอื่นได้ 

เพื่ออำนวยความสะดวกให้กับบรรณารักษ์ในการสืบค้นข้อมูลและทำให้การบริการในการสืบค้นเอกสารต่าง ๆ และให้บุคคลภายนอกสามารถทำการค้นหาข้อมูลได้ด้วยตนเองครบถ้วนทางคณะผู้จัดทำโครงการจึงได้พัฒนาระบบการจัดเก็บเอกสารและระบบการค้นหาโดยการใช้เครื่องมือในการทำ OCR เพื่อแปลงเอกสารให้อยู่ในรูปแบบของเอกสาร digital และหาคำสำคัญในการสร้าง tag ด้วยวิธี Term Frequency - Inverse Document Frequency เพื่อเพิ่มประสิทธิ์ภาพให้กับการค้นหา 

\section{ประเภทของโครงงาน}

นำเสนอความต้องการของผู้มีส่วนได้ส่วนเสียเฉพาะกลุ่ม 

\section{วิธีการที่นำเสนอ}

ระบบการค้นหาเอกสาร มีขั้นตอนการทำงานดังนี้

\begin{enumerate}
    \item นำเอกสารมาแปลงเป็นรูปภาพในรูปแบบสแกน
    \item นำรูปภาพเข้าสู่ระบบโดยใช้การรับส่งข้อมูลแบบ RESTful API ในระบุประเภทของการใช้งาน
    \item นำรูปภาพผ่านกระบวนการ Image Processing โดยใช้ OpenCV ในการลบส่วนอื่น ๆที่ไม่ใช่ข้อความออกและตัดเฉพาะข้อความเพื่อนำไปใช้ในขั้นต่อไป
    \item นำรูปที่ผ่านการทำ Image Processing มาเข้าสู่ระบบ OCR เพื่อแปลงข้อมูลจากรูปภาพมาเป็นข้อความในระบบดิจิตอล
    \item นำข้อมูลที่เก็บไว้มาทำการตัดแบ่งคำภาษาไทยและแก้คำผิด
    \item ค้นหาคำสำคัญโดยใช้วิธี TF-IDF เพื่อนำมาใช้ในการสร้าง Tag 
    \item นำข้อมูลที่ถูกแปลงเก็บและข้อมูลเกี่ยวกับ Tag ลงในดาต้าเบส 
    \item ทำระบบค้นหาในรูปแบบ Cosine Similarity 
    \item ทำระบบหาคำใกล้เคียงโดยใช้วิธี Word2Vec
    \item ทำแพลตฟอร์มเว็ปไซต์เพื่อเป็น User Interface ให้กับผู้ใช้งานได้ใช้งานสำหรับการใช้งานในการค้นหาข้อมูลและเพิ่มข้อมูลหนังสือลงไปในฐานข้อมูลเพิ่ม
\end{enumerate}
\section{วัตถุประสงค์}
\begin{enumerate}
    \item สร้างระบบแปลงข้อมูลเอกสารให้อยู่ในรูปแบบดิจิตอล
    \item สร้าง web platform เพื่อทำการค้นหาเอกสารจากคำค้น และพัฒนาเครื่องมือสนับสนุนการทำงานของบรรณารักษ์ประจำหอบรรณสารสนเทศ
    \item สร้างระบบการค้นหาโดยการใช้วิธีการ อินฟอเมชันรีทีฟวอล ซึ่งวัดความใกล้เคียงกันระหว่างคำค้นและข้อมูลในฐานข้อมูลโดยวิธี โคซาย ซิมิลาริตี้
    \item เพิ่มประสิทธิ์ภาพในการเข้าถึงข้อมูลในรูปแบบดิจิตอล
    \item เรียนรู้เรื่องการทำ Image processing
\end{enumerate}
\section{ขอบเขตของงานวิจัย}
\begin{enumerate}
    \item ระบบแปลงข้อมูลจากเอกสารและหนังสือเก่า รองรับเฉพาะเอกสารที่เป็นตัวอักษรแบบพิมพ์ และรองรับไฟล์เอกสารเฉพาะ PDF เท่านั้น
    \item ทำระบบตัดคำ Stop word ภาษาไทยโดยอ้างอิงมาจาก pythainip และภาษาอังกฤษจาก nltk
    \item ทำระบบค้นหาแบบ Cosine Similarity ในระบบ Information retrieval
    \item ข้อมูลหนังสือที่นำมาใช้คือหนังสือจำพวก งานแสดงกตเวทิตาจิต เอกสารรายงานประจำปี ตั้งแต่ปีพุทธศักราช 2527 ถึง 2560 รวมประมาณ 44 เล่ม จากหอจดหมายเหตุมหาวิทยาลัยเทคโนโลยีพระจอมเกล้าธนบุรี
    \item ทำ platform เว็บไซต์ในรูปแบบ responsive แต่ไม่รองรับขนาดมือถือ รองรับเฉพาะคอมพิวเตอร์หรือโน๊ตบุ๊ค
    \item การแปลงสิ่งพิมพ์เป็นดิจิตอลใช้ Tesseract ในการแปลงเอกสารและหนังสือเป็นรูปแบบดิจิตอล
    \item การตัดคำภาษาไทยทางคณะผู้จัดทำ จะใช้ freeware เช่น DeepCut มาใช้ในส่วนของการตัดคำภาษาไทย
\end{enumerate}
\section{เนื้อหาทางวิศวกรรมที่เป็นต้นฉบับ}
\begin{itemize}
    \item การทำ Image processing สำหรับการเตรียมภาพก่อนนำไปทำ OCR 
    
    โปรเจคของเราทำเกี่ยวกับการทำ OCR เพื่ออ่านภาพให้กลายเป็น text แต่ถึงแม้ว่าภาพที่ได้มาจะจากการสแกนหรือการถ่ายรูป แต่ถึงอย่างนั้น OCR ที่ใช้ก็ยังคงมีข้อจำกัดในเรื่องของคุณภาพของภาพที่ใช้ ถ้าเกิดว่าภาพที่ใช้เอียง หรือมี noise จะทำให้การอ่านมีประสิทธิภาพน้อยลง นอกจากนี้การตัดภาพแยกย่อหน้าแต่ละย่อหน้าทำให้การอ่านมีความถูกต้องมากยิ่งขึ้น

    \item การพัฒนาเว็บไซต์สำหรับการค้นหาหนังสือในหอจดหมายเหตุ
    
    เว็บไซต์ของเราจะใช้ ReactJS, NodeJs, python  ในการพัฒนาเว็บไซต์เป็น Interface ให้กับ user สำหรับการใช้งานระบบการค้นหาหนังสือ รวมถึงการอัปโหลดเอกสารเพื่อแปลงเอกสารเข้าสู่ระบบดิจิตอลและ API ต่าง ๆ
    
    \item คัดเลือกคำสำคัญออกมาเพื่อสร้าง tag 
    
    สำหรับแบ่งแยกหมวดหมู่ของหนังสือโดยใช้ หลักการของ TF-IDF ในการค้นหาคำสำคัญของหนังสือเพื่อนำมาสร้าง tag และใช้สำหรับการค้นหาข้อมูล
    
    \item ทำระบบค้นหาโดยใช้คำที่มีความหมายใกล้เคียง
    
    สำหรับการค้นหาเราจะนำคะแนน TF-IDF มาใช้เป็นคะแนนเพื่อใช้ในการค้นหาแบบ Cosine similarity และค้นหาคำใกล้เคียง (Query Expansion) เพื่อทำให้การค้นหาเจอผลลัพธ์ที่ต้องการเพิ่มมากขึ้น
    
\end{itemize}
\section{การแยกย่อยงาน และร่างแผนการดำเนินงาน}
\begin{enumerate}
    \item 	ศึกษาและค้นคว้าปัญหาของโครงการ
    \item 	เสนอหัวข้อโปรเจค 
    \item 	ค้นหาข้อมูลเกี่ยวกับเทคโนโลยีที่ใช้ในโปรเจค
    \item 	ประเมินความเป็นไปได้และกำหนดขอบเขตของโปรเจค 
    \item 	จัดเก็บ requirement จากกลุ่มผู้ใช้งาน
    \begin{enumerate}[label*=\arabic*.]
        \item   ติดต่อเจ้าหน้าที่ของหอสมุด
        \item   เก็บข้อมูลที่ต้องการแปลงเข้าสู่ระบบดิจิตอล
    \end{enumerate}
    \item 	นำเสนอโครงการครั้งที่ 1 
    \item 	ออกแบบ UX/UI
    \item 	แปลงรูปภาพเป็น Full-text
    \begin{enumerate}[label*=\arabic*.]
        \item 	นำเอกสารมาแปลงเป็นรูปภาพในรูปแบบสแกน 
        \item  ศึกษาการใช้งาน OpenCV
        \item  สร้างระบบ Image processing เพื่อทำการปรับแต่งรูปภาพและทำการปรับแต่งจนได้ระบบที่รองรับกับ Data ที่มี
        \item  	นำรูปที่ผ่านการทำ Image Processing มาเข้าสู่ระบบ OCR เพื่อแปลงข้อมูลจากรูปภาพมาเป็นข้อความในระบบดิจิตอล 
    \end{enumerate}
    \item 	นำข้อมูลที่เก็บไว้มาทำการตัดแบ่งคำภาษาไทยและหาคำสำคัญโดยใช้ TF-IDF 
    \begin{enumerate}[label*=\arabic*.]
        \item	ทำการตัดแบ่งคำ (Tokenization)
        \item	ลบ stop word ออกจากข้อมูล 
    \end{enumerate}
    \item	ทำระบบค้นหา 
    \begin{enumerate}[label*=\arabic*.]
       \item	ทำระบบค้นหาโดยใช้หลักการ Cosine Similarity
       \item	ทำการค้นหาด้วยคำใกล้เคียงโดยใช้ Word2Vec 
    \end{enumerate}    
    \item   จัดทำเว็บไซต์แพลตฟอร์ม
    \item   ทดสอบระบบ
    \item   ปรับปรุงแก้ไข
    \item   นำเสนอโปรเจค
    \end{enumerate}

\section{ตารางการดำเนินงาน}

\renewcommand{\arraystretch}{1.5}
\begin{table}[H]
    \caption{ตารางการดำเนินงาน ภาคการศึกษาที่ 1/2563}\label{tbl:work1}
    \begin{tabular}{|l|p{0.20\linewidth}|l|l|l|l|l|l|l|l|l|l|l|l|l|l|l|l|l|l|l|l|}
    \hline
    \multicolumn{22}{|c|}{ตารางการดำเนินงาน ภาคการศึกษาที่ 1/2563}                                                                                                                                                                                                                                                                                                                                                                                                                                                                                                                                                                                                                                                \\ \hline
                       &                 & \multicolumn{4}{c|}{สิงหาคม}                                                                                                                                   & \multicolumn{4}{c|}{กันยายน}                                                                                     & \multicolumn{4}{c|}{ตุลาคม}                                                                                     & \multicolumn{4}{c|}{พฤศจิกายน}                                                                                     & \multicolumn{4}{c|}{ธันวาคม}                                                                                     \\ \cline{3-22} 
    \multirow{-2}{*}{ที่} & \multicolumn{1}{c|}{\multirow{-2}{*}{หัวข้อ}} & \multicolumn{1}{c|}{1}   & \multicolumn{1}{c|}{2}                          & \multicolumn{1}{c|}{3}                          & \multicolumn{1}{c|}{4}   & \multicolumn{1}{c|}{1}   & \multicolumn{1}{c|}{2}   & \multicolumn{1}{c|}{3}   & \multicolumn{1}{c|}{4}   & \multicolumn{1}{c|}{1}   & \multicolumn{1}{c|}{2}   & \multicolumn{1}{c|}{3}   & \multicolumn{1}{c|}{4}   & \multicolumn{1}{c|}{1}   & \multicolumn{1}{c|}{2}   & \multicolumn{1}{c|}{3}   & \multicolumn{1}{c|}{4}   & \multicolumn{1}{c|}{1}   & \multicolumn{1}{c|}{2}   & \multicolumn{1}{c|}{3}   & \multicolumn{1}{c|}{4}   \\ \hline
    1                  & ศึกษาค้นคว้าหาปัญหาของโครงการ                                        & \cellcolor[HTML]{656565} & \cellcolor[HTML]{656565}                        & \cellcolor[HTML]{656565}                        &                          &                          &                          &                          &                          &                          &                          &                          &                          &                          &                          &                          &                          &                          &                          &                          &                          \\ \hline
    2                  & เสนอหัวข้อโปรเจค                                        &                          & \cellcolor[HTML]{656565}{\color[HTML]{656565} } & \cellcolor[HTML]{656565}{\color[HTML]{656565} } &                          &                          &                          &                          &                          &                          &                          &                          &                          &                          &                          &                          &                          &                          &                          &                          &                          \\ \hline
    3                  & ศึกษาและหาข้อมูลเกี่ยวกับเทคโนโลยีที่ใช้ในโปรเจค                                        &                          &                                                 & \cellcolor[HTML]{656565}                        & \cellcolor[HTML]{656565} &                          &                          &                          &                          &                          &                          &                          &                          &                          &                          &                          &                          &                          &                          &                          &                          \\ \hline
    4                  & ประเมินความเป็นไปได้และกำหนดขอบเขตของโปรเจค                                       &                          &                                                 & \cellcolor[HTML]{656565}                        & \cellcolor[HTML]{656565} &                          &                          &                          &                          &                          &                          &                          &                          &                          &                          &                          &                          &                          &                          &                          &                          \\ \hline
    5                  & จัดเก็บ requirement จากกลุ่มผู้ใช้งาน                                        &                          &                                                 &                                                 & \cellcolor[HTML]{656565} & \cellcolor[HTML]{656565} &                          &                          &                          &                          &                          &                          &                          &                          &                          &                          &                          &                          &                          &                          &                          \\ \hline
    6                  & นำเสนอโครงงานครั้งที่ 1                                        &                          &                                                 &                                                 &                          & \cellcolor[HTML]{656565} &                          &                          &                          &                          &                          &                          &                          &                          &                          &                          &                          &                          &                          &                          &                          \\ \hline
    7                  & ออกแบบ UX/UI                                        &                          &                                                 &                                                 &                          &                          & \cellcolor[HTML]{656565} & \cellcolor[HTML]{656565} &                          &                          &                          &                          &                          &                          &                          &                          &                          &                          &                          &                          &                          \\ \hline
    8                  & แปลงรูปภาพเป็น Full-text                                        &                          &                                                 &                                                 &                          &                          & \cellcolor[HTML]{656565} & \cellcolor[HTML]{656565} & \cellcolor[HTML]{656565} & \cellcolor[HTML]{656565} & \cellcolor[HTML]{656565} & \cellcolor[HTML]{656565} & \cellcolor[HTML]{656565} & \cellcolor[HTML]{656565} & \cellcolor[HTML]{656565} & \cellcolor[HTML]{656565} &                          &                          &                          &                          &                          \\ \hline
    9                  & นำข้อมูลที่เก็บไว้มาทำการตัดแบ่งคำภาษาไทยและทำการสร้าง tag โดยใช้หลักการของ TF-IDF                                        &                          &                                                 &                                                 &                          &                          &                          & \cellcolor[HTML]{656565} & \cellcolor[HTML]{656565} & \cellcolor[HTML]{656565} & \cellcolor[HTML]{656565} & \cellcolor[HTML]{656565} & \cellcolor[HTML]{656565} & \cellcolor[HTML]{656565} & \cellcolor[HTML]{656565} & \cellcolor[HTML]{656565} & \cellcolor[HTML]{656565} & \cellcolor[HTML]{656565} &                          &                          &                          \\ \hline
    10                 & นำเสนอโปรเจค                                        &                          &                                                 &                                                 &                          &                          &                          &                          &                          &                          &                          &                          &                          &                          &                          &                          &                          & \cellcolor[HTML]{656565} &                          &                          &                          \\ \hline
    11                 & จัดทำระบบการค้นหา                                        &                          &                                                 &                                                 &                          &                          &                          &                          &                          &                          &                          &                          &                          &                          &                          &                          &                          &                          & \cellcolor[HTML]{656565} & \cellcolor[HTML]{656565} & \cellcolor[HTML]{656565} \\ \hline
    \end{tabular}
    \end{table}

\begin{table}[H]
\caption{ตารางการดำเนินงาน ภาคการศึกษาที่ 2/2563}\label{tbl:work2}
\begin{tabular}{|l|p{0.35\linewidth}|l|l|l|l|l|l|l|l|l|l|l|l|l|l|l|l|}
\hline
\multicolumn{18}{|c|}{ตารางการดำเนินงาน ภาคการศึกษาที่ 2/2563}                                                                                                                                                                                                                                                                                                                                                                                                                                                                 \\ \hline
                   &                    & \multicolumn{4}{c|}{มกราคม}                                                                                     & \multicolumn{4}{c|}{กุมภาพันธ์}                                                                                     & \multicolumn{4}{c|}{มีนาคม}                                                                                     & \multicolumn{4}{c|}{เมษายน}                                                                                     \\ \cline{3-18} 
\multirow{-2}{*}{ที่} & \multicolumn{1}{c|}{\multirow{-2}{*}{หัวข้อ}} & 1                        & 2                        & 3                        & 4                        & 1                        & 2                        & 3                        & 4                        & 1                        & 2                        & 3                        & 4                        & 1                        & 2                        & 3                        & 4                        \\ \hline
1                  & จัดทำระบบการค้นหา                   & \cellcolor[HTML]{656565} & \cellcolor[HTML]{656565} & \cellcolor[HTML]{656565} & \cellcolor[HTML]{656565} & \cellcolor[HTML]{656565} & \cellcolor[HTML]{656565} & \cellcolor[HTML]{656565} &                          &                          &                          &                          &                          &                          &                          &                          &                          \\ \hline
2                  & จัดทำเว็บไซต์แพลตฟอร์ม                   & \cellcolor[HTML]{656565} & \cellcolor[HTML]{656565} & \cellcolor[HTML]{656565} & \cellcolor[HTML]{656565} & \cellcolor[HTML]{656565} & \cellcolor[HTML]{656565} & \cellcolor[HTML]{656565} & \cellcolor[HTML]{656565} &                          &                          &                          &                          &                          &                          &                          &                          \\ \hline
3                  & ทดสอบระบบ                   &                          &                          &                          &                          & \cellcolor[HTML]{656565} & \cellcolor[HTML]{656565} & \cellcolor[HTML]{656565} & \cellcolor[HTML]{656565} & \cellcolor[HTML]{656565} & \cellcolor[HTML]{656565} & \cellcolor[HTML]{656565} & \cellcolor[HTML]{656565} & \cellcolor[HTML]{656565} & \cellcolor[HTML]{656565} &                          &                          \\ \hline
4                  & ปรับปรุงแก้ไข                   &                          &                          &                          &                          & \cellcolor[HTML]{656565} & \cellcolor[HTML]{656565} & \cellcolor[HTML]{656565} & \cellcolor[HTML]{656565} & \cellcolor[HTML]{656565} & \cellcolor[HTML]{656565} & \cellcolor[HTML]{656565} & \cellcolor[HTML]{656565} & \cellcolor[HTML]{656565} & \cellcolor[HTML]{656565} &                          &                          \\ \hline
5                  & นำเสนอโปรเจค                   &                          &                          &                          &                          &                          &                          &                          &                          &                          &                          &                          &                          &                          &                          & \cellcolor[HTML]{656565} & \cellcolor[HTML]{656565} \\ \hline
\end{tabular}
\end{table}

\subsection{ผลการดำเนินงานในภาคการศึกษาที่ 1}
\begin{itemize}
    \item ทำระบบ Image processing สำหรับการเตรียมรูปภาพสำหรับการแปลงข้อมูลเป็นดิจิตอล
    \item ทำ API ในการตัดคำและจัดการ stop word สำหรับการเตรียมการ text processing
    \item ทำระบบ Term Frequency-Inverse Document Frequency สำหรับการค้นหาคำสำคัญเพื่อสร้าง tag 
    \item ทำส่วนของการทำการค้นหาข้อมูลเบื้องต้น
\end{itemize}

\subsection{ผลการดำเนินงานในภาคการศึกษาที่ 2}
\begin{itemize}
    \item ทำระบบค้นหาให้เสร็จสิ้น
    \item ปรับปรุงระบบค้นหาให้ตอบโจทย์มากยิ่งขึ้น
    \item ทำเว็บไซต์ platform ทั้งฝั่ง frontend และ backend
\end{itemize}

\chapter{ที่มา ทฤษฎีและงานวิจัยที่เกี่ยวข้อง}
\section{บทนำ}

โดยทฤษฎีที่เกี่ยวข้องกับโปรเจคนี้มีหลากหลายสาขาด้วยกันโดยจะแบ่งเป็นส่วนของการเตรียมข้อมูลรูปภาพ โดยการใช้ Open source Computer Vision (OpenCV) เพื่อนำไปใช้กับส่วนของการทำ Optical Character Recognition (OCR), Tesseract OCR และส่วนของการทำ Natural language processing (NLP) โดยการใช้ Team Frequency Inverse Document Frequency (TF-IDF), Minimum Edit Distance,  Deep Cut ส่วนต่อไปคือสร้างระบบการค้นหาได้ใช้ Cosine Similarity และในส่วนการสร้างเว็บไซต์โดยใช้ RESTful API และส่วนสุดท้ายการทำ Word Embedding 

\section{แนวความคิดทางทฤษฎี}

\subsection{การเตรียมข้อมูลรูปภาพ}

เป็นการประมวลผลรูปภาพที่แปลงภาพให้เป็นข้อมูลทางดิจิทัลเพื่อใช้สำหรับปรับคุณภาพของภาพให้ตรงตามความต้องการ อย่างการตัดสิ่งรบกวน การลบกรอบ การหมุนรูป หรือการปรับให้ภาพมีความคมชัดมากยิ่งขึ้น ในโปรเจคของเรานั้นเอามาใช้ในการปรับคุณภาพของรูปภาพเพื่อช่วยให้การทำ OCR แม่นยำมากยิ่งขึ้น 

\subsubsection{คอนทัว (Contour) }

คอนทัว (Contour) \cite{doxygen} คือเส้นเค้าโครงของรูปภาพ ที่ไว้หาขอบเขตพื้นที่ที่มีค่าสีต่อเนื่องกัน หรือค่าเดียวกัน โดยใช้การเปลี่ยนให้รูปภาพอยู่ในรูปของ matrix และเช็คดูว่าค่าสีที่มีความแตกต่างอย่างชัดเจนเริ่มที่ตรงไหนและสร้างเป็นเส้นเค้าโครงขึ้นมาดังรูป \ref{fig:contour} ซึ่งการหาเส้นเค้าโครงจะทำงานได้ดีก็ต่อเมื่อเป็นรูปภาพแบบ Binary 

\begin{figure}[!h]
    \centering
    \includegraphics{contour}
    \caption{แสดงการหาเค้าโครงภายในรูป}\label{fig:contour}
\end{figure}

\subsubsection{การเปลี่ยนแปลงทางสัณฐานวิทยา (Morphology Transformation)}

เป็นกระบวนการเตรียมข้อมูลรูปภาพ  ที่จะทำการนำรูปภาพมาทำการเปลี่ยนแปลงลักษณะ รูปร่างของวัตถุภายในภาพ ปกติแล้วจะใช้ภาพที่เป็น Binary ซึ่งส่วนใหญ่จะใช้สำหรับการกำจัด 
noise การซ่อมแซมรูปร่างของภาพ หรือการเพิ่มขนาดให้กับวัตถุนั้นๆ โดยการทำการเปลี่ยนแปลงทางสัณฐานวิทยา (Morphology Transformation) นั้นจะมีวิธีการดำเนินการพื้นฐานอยู่ 2 วิธีคือการขยายภาพ และการกร่อนภาพ

Dilation คือการเพิ่มพื้นที่สีขาวของรูปเพิ่มพื้นที่สีไปตามขอบพื้นที่สีขาวและจะเปลี่ยนพื้นที่สีดำให้กลายเป็นสีขาวทำให้พื้นที่สีขาวมีความหนามากขึ้นดังรูป 

\begin{figure}[H]
    \centering
    \includegraphics{dilation}
    \caption{แสดงการทำการขยายภาพ (Dilation) เพื่อเพิ่มพื้นที่สีขาว}\label{fig:Dilation}
\end{figure}

Erosion คือการกร่อนภาพ หรือก็คือจะลดพื้นที่สีขาวของภาพออกไปซึ่งวิธีการนี้ส่วนใหญ่จะใช้สำหรับการแยกสิ่งที่ของที่อยู่ติดกัน หรือลบ pepper noise ที่เป็น noise เล็กๆได้ โดยจะใช้หลักการเดียวกับการขยายภาพ  (Dilation) เพียงแต่จะเปลี่ยนจากพื้นที่สีขาวให้กลายเป็นพื้นที่สีดำดังรูป

\begin{figure}[H]
    \centering
    \includegraphics{erosion}
    \caption{แสดงการทำกร่อนภาพ (Erosion) เพื่อกร่อนพื้นที่สีขาว}\label{fig:Erosion}
\end{figure}

\subsection{Optical Character Recognition (OCR)}

OCR เป็นกระบวนการของการแปลงอักษรบนสื่อสิ่งพิมพ์ให้เป็นข้อความที่สามารถค้นหา เปลี่ยนแปลงและแก้ไขได้โดยที่ไม่ต้องพิมพ์ขึ้นมาใหม่ ด้วยการทำ Deep learning ในการเรียนรู้ภาพเพื่อแปลงออกมาเป็นตัวอักษร ซึ่งในโปรเจคของทางผู้จัดทำต้องทำระบบเกี่ยวกับค้นหาที่จะต้องคัดคำออกมาจากสิ่งพิมพ์เหล่านั้น จึงจำเป็นที่จะต้องใช้ OCR ในการแปลงภาพต้นแบบออกมาให้เป็นตัวอักษรก่อนที่จะนำไปใช้งานต่อ

จากการศึกษาพบว่าการทำ OCR ภาษาไทยนั้นมีอยู่มากมายในปัจจุบัน หนึ่งในนั้นมี T - OCR ซึ่งเป็น Library ของ AI For Thai \cite{nectec} และ Tesseract ของ Google \cite{google} ที่ใช้สำหรับแปลงภาพเป็นตัวอักษร โดยกลุ่มของเราเลือกที่จะใช้ Tesseract ในการทำ OCR เนื่องจากไม่เสียค่าใช้จ่ายเมื่อเทียบกับการใช้ OCR ของ AI For Thai นอกจากนั้นเรื่องของการเรียกใช้งานอย่างต่อเนื่อง Tesseract สามารถทำได้ดีกว่าเนื่องจากไม่จำเป็นต้องเรียกใช้งาน AI For Thai จากภายนอก

\subsection{Natural language processing}

Natural language processing คือกระบวนการที่ใช้ในทางปัญญาประดิษฐ์ซึ่ง เป็นกระบวนการที่ทำการวิเคราะห์ทางด้านภาษาซึ่งเอาไปประยุกต์ทำให้ปัญญาประดิษฐ์ (AI) สามารถทำให้คอมพิวเตอร์เข้าใจภาษาและตอบกลับได้ใกล้เคียงกับมนุษย์มากขึ้น โดยในโปรเจคนี้จะใช้มาช่วยในการหาคำสำคัญของหนังสือ และบทความต่าง ๆ เพื่อช่วยให้การค้นหาบทความมีประสิทธิภาพมากขึ้น

\subsubsection{Information retrieval}

Information retrieval คือ เทคโนโลยีการเก็บข้อมูลอย่างนึงโดยจะมีทั้งหมด 2 ลักษณะ ลักษณะที่ 1 คือ Boolean Retrieval เป็นการสร้างโครงสร้างข้อมูลในรูปแบบ Matrix ที่มีค่าเพียงแค่ 0,1 โดยที่ 0 คือไม่มีคำ (Term) ในหนังสือนั้น และ 1 คือมีคำ (Term) อยู่ภายในหนังสือนั้นหรือเรียกได้ว่าเป็น Term-Document Incidence  Matrix ดัง ตารางที่ 2.1 โดยที่ถ้าเราพิจารณาในรูปแบบแถวเราจะได้ Vector ของ Term นั้นที่ปรากฏอยู่ในหนังสือ ไหนบ้าง แต่การเก็บในรูปแบบ Boolean Retrieval เมื่อมีหนังสือ เยอะขึ้นจะทำให้เกิดค่า 0 ที่ไม่มีประโยชน์มากขึ้นจึงมีลักษณะที่ 2 คือโครงสร้างแบบ Inverted index เป็นการเก็บเพียง Term นั้นอยู่ภายในหนังสือ ไหนบ้างเพื่อจะเก็บแต่เพียงข้อมูลสำคัญเอาไว้ดัง ตารางที่ 2.2 โดย คำ (Term) จะผ่านกระบวนการเตรียมข้อมูลตัวอักษร ประกอบไปด้วย Tokenization (การตัดคำจากประโยชน์), Normalization (การจัดการคำย่อ), Stemming (การแปลงคำให้อยู่รูปแบบเดียวกัน), Stop words (จัดการคำที่ไม่มีความหมาย) เพื่อเป็นการจัดรูปของคำให้อยู่ในรูปแบบเดียวกันก่อนที่จะนำไปใช้งาน ซึ่งการเก็บข้อมูลแบบ Information retrieval (IR) จะทำให้การค้นหาข้อมูลภายในฐานข้อมูลได้อย่างรวดเร็วและมีประสิทธิ์ภาพ
\begin{table}[H]
\caption{Information retrieval ในลักษณะ Boolean Retrieval}\label{tbl:ir}
    \begin{tabular}{|l|c|c|c|c|c|}
        \hline
                  & Antony \& Cleopatra & Julius Ceasar & The Tempest & Hamlet & Othello \\ \hline
        Antony    & 1                   & 1             & 0           & 0      & 0       \\ \hline
        Brutus    & 1                   & 1             & 0           & 1      & 0       \\ \hline
        Ceasar    & 1                   & 1             & 0           & 1      & 1       \\ \hline
        Calpurnia & 0                   & 0             & 1           & 0      & 0       \\ \hline
        Cleopatra & 1                   & 0             & 1           & 1      & 1       \\ \hline
        Mercy     & 1                   & 0             & 1           & 1      & 1       \\ \hline
        \end{tabular}
\end{table}

    \begin{figure}[H]
        \centering
        \includegraphics{ir}
        \caption{Information retrieval ในลักษณะ Index Retrieval}\label{fig:ir}
    \end{figure}

\subsubsection{TF-IDF}

เป็นเทคนิคในการคัดแยกคำตามความสำคัญผ่านการให้น้ำหนักในแต่ละคำ โดยแบ่งเป็นสองส่วนนั้นก็คือ TF (Team Frequency) เป็นการดูว่าคำนี้ หรือว่า Term นี้ปรากฏขึ้นภายใน document มากน้อยเพียงไหน และ IDF (Inverse  Document Frequency) คือการหาความผกผันในความถี่ของหนังสือโดยคะแนนความผกผันที่ทำให้รู้ว่าคำนี้เป็นคำที่มีความสำคัญเฉพาะภายในหนังสือนี้ แต่เนื่องจากการดูคะแนน IDF เพียงอย่างเดียวไม่สามารถบอกได้ว่า Term นั้นเป็นคำสำคัญ จึงจำเป็นต้องนำค่า TF มาคูณกับ IDF เป็นค่า TF-IDF เพื่อดูความสำคัญของ Term นั้น ในส่วนการคำนวณนี้เพื่อนำไปใช้ในการค้นหาแบบ Cosine Similarity ต่อไป โดยที่ TF จะใช้เป็น Log normalization โดยคำนวณได้จากสมการ 2.1 ซึ่ง $f_{t,d}$ คือความถี่ของคำ (Term) ที่ปรากฏขึ้นภายใน Document ส่วน IDF จะคำนวณจากสมการ 2.2 ซึ่ง N คือจำนวน Document ที่มีภายในระบบ และ $n_{t}$ คือ จำนวนของ document ที่มีคำ (term) นี้อยู่ เมื่อหาค่าทั้ง TF และ IDF ได้แล้วก็จะหาค่าของ TF-IDF ได้จากสมการ 2.3

\begin{equation}
    tf=\log{(1+f_{t,d})}
    \end{equation}

\begin{equation}
    idf=\log{\frac{N}{n_{t}}}
\end{equation}

\begin{equation}
    TF-IDF=tf*idf
    \end{equation}    

\subsubsection{Cosine Similarity}

เป็นหน่วยวัดความคล้ายคลึงกันระหว่างข้อมูลสอง Vector โดยวัดจากมุม cosine ของจาก Vector ทั้งสองโดยคำนวณได้จากสมการ 2.4 โดยที่ ||x||,||y|| คือ สมการของ Euclidean norm ของ Vertor x, y ดังสมการ 2.5 โดยในโปรเจคนี้เราได้นำค่าน้ำหนักของ TF-IDF มาเป็นน้ำหนักในการคิดค่า Cosine Similarity โดยนำประโยคที่จะค้นหามาผ่านกระบวนการเตรียมข้อมูลตัวอักษร ก่อนที่จะนำมาค้นหาว่า document ไหนมีค่า relevance score (คะแนนความสัมพันธ์) เพื่อนำมาเรียงค่าคะแนนสูงสุดแสดงเป็นผลลัพธ์การค้นหา


\begin{equation}
    \sin(x,y)=\frac{x*y}{\|x\|\|y\|}
\end{equation}    

\begin{equation}
    \|x\|=\sqrt{x_{1}^2+x_{2}^2+\cdots+x_{n}^2}
\end{equation}    
\subsubsection{Minimum Edit Distance}

เป็นหลักการที่หาระยะห่างที่สั้นที่สุดจากคำนึงไปสู่อีกคำนึงจะมีความแตกต่างกันเท่าไหร่ซึ่งจะหลักการเช็คความห่างของคำทั้งหมดสามรูปแบบ
\begin{itemize}
    \item รูปแบบ Insert(I) จะเป็นการเพิ่มตัวอักษรลงไปในคำนั้น เพื่อคำดั้งเดิมของเราจะเปลี่ยนแปลงเป็นคำที่เราต้องการ
    \item รูปแบบ Delete(D) จะเป็นการลบตัวอักษรออกไปในคำนั้น เพื่อคำดั้งเดิมของเราจะเปลี่ยนแปลงเป็นคำที่เราต้องการ
    \item รูปแบบ Replace(R) จะเป็นการเปลี่ยนตัวอักษรนั้นให้เป็นตัวอักษรใหม่ เพื่อคำดั้งเดิมของเราจะเปลี่ยนแปลงเป็นคำที่เราต้องการ
\end{itemize}
\begin{figure}[H]
    \centering
    \includegraphics{editdistance}
    \caption{หลักการการเช็ค edit distance \cite{ritambhara}}\label{fig:editdistance}
\end{figure}
หลังจากมีรูปแบบการวัดระยะห่างของคำดังรูปภาพที่ \ref{fig:editdistance} แล้ว จะต้องทำการหาคำที่สั้นที่สุดผ่านรูปแบบของตารางดังรูปภาพที่ \ref{fig:minimumtable} ซึ่งการคำนวณผ่านตารางจะเป็นการนำการกระทำก่อนหน้ามาคำนวนเรื่อย ๆ จนได้รูปการเปลี่ยนเป็นคำใหม่ที่ใช้การเปลี่ยนน้อยที่สุด
\begin{figure}[H]
    \centering
    \includegraphics{minimumtable}
    \caption{ตัวอย่างตารางการทำ Minimum edit distance \cite{ritambhara}}\label{fig:minimumtable}
\end{figure}

ซึ่งในโปรเจคของเราได้ดึงหลักการ Minimum edit distance มาใช้ในการตรวจสอบหาคำที่สะกดไม่ถูกต้องโดยมีเกณฑ์ตั้งไว้ว่าถ้าเกินที่กำหนดไว้จะถือว่าคำ ๆ นั้นสะกดไม่ถูกต้องแล้วถูกแก้ให้เป็นคำที่สะกดถูกต้อง

\subsubsection{Spelling Corrector by Peter Norvig}

หลักการการแก้คำผิดของ Peter Norvig\cite{norvig} เป็นการนำคำที่ถูกพิมพ์เข้ามาสร้างมาเป็นคำใหม่โดยการแบ่งคำ การลบคำ การกลับคำ การแทนที่คำ การเพิ่มคำ 
และนำคำที่สร้างใหม่ไปเช็คในพจนานุกรมภาษาว่ามีคำที่สร้างใหม่คำนั้นหรือไม่ ถ้ามีให้นำคำที่ได้ไปหาค่าความน่าจะเป็นของคำที่มีโอกาสเกิดขึ้นมากที่สุด ซึ่งใน Library 
นี้ Norvig ได้นำไปหาโดยใช้ข้อมูลจาก Corpus ที่สร้างขึ้นจากคำที่ได้จาก Project Gutenberg\cite{guten}, Wikitionary\cite{wikitionary} และ British 
National Corpus\cite{kilgarriff} ที่จะรวมคำต่างเอาไว้ จากนั้นคำนวนความน่าจะเป็นของคำที่เกิดขึ้นโดยใส่คำที่ต้องการหาแล้วนับว่าใน Corpus นั้นมีคำนั้นปรากฏขึ้นเท่าไร
หารด้วยจำนวนคำของ Corpus ทั้งหมด ก็จะได้ความน่าจะเป็นออกมา ซึ่งคำที่จะแก้ก็จะเลือกจากคำที่มีค่าความน่าจะเป็นมากที่สุด โดย library PyThaiNLP\cite{pythainlp} และ Pyspellchecker\cite{pypi} 
นั้นก็ใช้หลักการเดียวกันกับการแก้คำผิดของ Peter Norvig 

\subsection{RESTful Service}
เป็นการสร้าง web service โดยเรียกใช้ผ่านทาง HTTP Method ทั้ง 4 ประเภท GET/POST/PUT/DELETE ส่งข้อมูลออกมาเป็นรูปของ XML ทำให้ปริมาณข้อมูลที่ส่งมาน้อยกว่าการใช้ Protocol SOAP  โดยโครงสร้างของ 
HTTP Request ดังรูปภาพที่ \ref{fig:httpreq} ประกอบด้วย 

\begin{enumerate}
 \item VERB: แสดง method ของ HTTP
 \item URI: ตำแหน่งของข้อมูลที่ต้องการ
 \item HTTP Version: เวอร์ชั่นของ HTTP
 \item Request Header: Metadata ที่เก็บข้อมูลในรูปแบบ Key-Value ของ header
 \item Request Body: ส่วนเก็บข้อมูลของเนื้อหา
\end{enumerate}

\begin{figure}[H]
    \centering
    \includegraphics{httpreq}
    \caption{แสดงถึงโครงสร้างของ HTTP Request \cite{Saixiii}}\label{fig:httpreq}
\end{figure}

HTTP Response ดังรูปภาพที่ \ref{fig:httpres} ประกอบด้วย


\begin{enumerate}
	\item HTTP Version: เวอร์ชั่นของ HTTP
	\item Response Code: รหัสผลลัพธ์ของการทำงานในระดับ HTTP เป็นเลข 3 หลัก
	\item Response Header: Metadata ที่เก็บข้อมูลในรูปแบบ Key-Value ของ header
	\item Response Body: ส่วนเก็บข้อมูลของเนื้อหา

   \end{enumerate}
   
\begin{figure}[H]
    \centering
    \includegraphics{httpres}
    \caption{แสดงถึงโครงสร้างของ HTTP Response \cite{Saixiii}}\label{fig:httpres}
\end{figure}

\subsection{Word Embedding}

เป็นวิธีการที่จะเปลี่ยนคำปกติเป็น vector ให้อยู่ในหลากหลายมิติเพื่อให้สามารถเปรียบเทียบคำต่าง ๆ 
 โดยมีการทำ word embedding มากมายไม่ว่าจะเป็น Word2Vec \cite{xin} \cite{Goldberg} ที่ถูกสร้างโดยทีมวิจัยของ 
Google FastText \cite{fasttext} เป็น word embedding อีกหนึ่งตัวที่สร้างขึ้นจากทีมวิจัยของ facebook หรือจะเป็น ELMo \cite{matthew} 
ที่เป็นรูปแบบการ word embedding ที่ดูรูปคำโดยรอบเป็นต้น 



\subsubsection{Word2Vec}
ทฤษฎี Word2Vec\cite{shortStoryForWord2Vec} เป็นทฤษฎีที่ช่วยจัดการกับคำศัพท์ที่มีความหมายใกล้เคียงกัน หรือคำศัพท์ที่มีความสัมพันธ์อย่างคำตรงข้ามกัน 
อย่างคำว่า พระราชา กับ กษัตริย์ ที่เป็นคำที่ความหมายใกล้เคียงกัน แต่ตรงข้ามกับคำว่า ราชินี โดยทั้งหมดนี้จะถูกจัดเก็บในรูปแบบของเวกเตอร์
แน่นอนว่าการที่จะรู้ว่าแต่ละคำมีความสัมพันธ์ได้นั้น เราจำเป็นต้องสร้างประเภทของคำจากรูปประโยคอย่างเช่น “การ เลี้ยง หมา เป็น สัตว์เลี้ยง ถือ เป็น การ ผ่อนคลาย” 
กับ “แมว ถือ เป็น สัตว์เลี้ยง ที่ สุขุม” จากทั้งสองประโยคจะเห็นว่าคำที่เหมือนกันจากทั้งสองประโยค จะเป็นคำว่า สัตว์เลี้ยง ซึ่งการทำ Word2Vec 
จะเป็นการดึงประเภทโดยเช็คจากคำใกล้เคียงเพื่อจัดกลุ่ม ดังนั้นจากคำว่า สัตว์เลี้ยง ที่มีคำว่า หมา และ แมว เป็นสัตว์เลี้ยงเช่นเดียวกัน จะแสดงว่าคำว่า 
หมา และ แมว มีความหมายใกล้เคียงกันจากความสัมพันธ์กัน ทั้งนี้การที่แต่ละคำจะมีสัมพันธ์ที่เชื่อมโยงกันได้มาก หรือน้อยขึ้นอยู่ระยะของคำ (Window Size) 
ยกตัวอย่างถ้ามีค่ามาก คำว่าสัตว์เลี้ยงอาจจะมีคำที่ไม่เกี่ยวข้องเกิดขึ้นหากข้อมูลไม่เพียงพอ เช่น ตู้เย็น หมา แมว เป็นสัตว์เลี้ยง แต่ถ้าหากค่าน้อยเกินไป หมา 
อาจจะไม่ถูกนับเป็นประเภทสัตว์เลี้ยงแทน

นอกจากส่วนการใช้คำข้างเคียงมาช่วยในการทำ Word2Vec แล้วยังมีส่วนของรูปแบบการสร้างโมเดลเพื่อหาคำที่มีความสัมพันธ์กันอีกด้วย 
โดยจะมีสองรูปแบบคือ CBOW (continuous bag of words) จะเป็นรูปแบบของการที่ใช้คำหลาย ๆ คำเผื่อมาหาคำเพียงคำเดียว 
และ Skip-Gram ที่จะตรงข้ามกับรูปแบบที่แล้วคือ การที่ใช้คำหนึ่งคำเพื่อหาคำหลายคำออกมา
\begin{figure}[H]
    \centering
    \includegraphics[scale=0.5]{cbowskip}
    \caption{ภาพแสดงโมเดล Skip-gram และ CBOW \cite{ichi}}\label{fig:cbowskip}
\end{figure}

ซึ่งผลลัพธ์ที่ได้จากการทำ Word2Vec ในรูปแบบ CBOW และ Skip-gram จะเปลี่ยนคำให้อยู่ในรูปแบบของ vector 
ก็จะเป็นเสมือนจุดบนกราฟที่สามารถหาค่าได้โดยใช้ vector ดังภาพที่ \ref{fig:embedgraph}
ว่ามีความสัมพันธ์ใกล้เคียงกับคำไหนบ้างในระบบโดยมาจากการ dot กันระหว่างเวกเตอร์ของคำ 
ซึ่งมิติใน Word2Vec นั้นจะเป็นการแสดงถึงการจำแนกประเภทของคำเหล่านี้ โดยที่เราจะสามารถกำหนดตัวเลขขึ้นมาและโมเดลจะทำการสร้างมิติ  

\begin{figure}[H]
    \centering
    \includegraphics[scale=0.5]{embedgraph}
    \caption{ภาพแสดงการหาความสัมพันธ์ระหว่างคำโดยใช้วิธีการทาง Word embedded แบบ Word2Vec \cite{lukkid}}\label{fig:embedgraph}
\end{figure}

จากภาพที่ \ref{fig:7Dto2D} จะแสดงให้เห็นว่าแต่ละคำใกล้เคียงกับอะไรบ้าง อย่างเช่น cat และ kitten
จะมีค่าในแต่ละมิติที่ใกล้เคียงกัน ดูจากภาพมิติก็คือ living being, feline, human เป็นต้น ซึ่งจะถูกเรียนรู้ขึ้นมาโดยโมเดล ในที่นี้กำหนด ไว้ 7 มิติ
จะเห็นได้จากภาพกราฟในทางด้านขวาที่จะผลลัพธ์ที่ได้จากการลดมิติจาก 7 มิติ ให้เหลือเป็น 2 มิติโดยใช้วิธีการ dimentionality reduction เช่น PCA\cite{PCA}\cite{PCAeng} 
เพื่อที่จะแสดงให้เห็นว่า cat กับ kitten อยู่ใกล้กันมากกว่า cat กับ dog

\begin{figure}[H]
    \centering
    \includegraphics[scale=0.2]{7Dto2D}
    \caption{ภาพแสดงการลดมิติของ word embedded แบบ Word2Vec ให้อยู่ในรูป 2 มิติ\cite{sasiwut}}\label{fig:7Dto2D}
\end{figure}



\subsubsection{Skip-gram}

Skip-gram\cite{ichi} เป็นโมเดลที่จะใช้หาความสัมพันธ์ของคำ โดยจะนำคำที่เข้าไปไปหาคำบริบทโดยรอบ โดยการที่โมเดลจะทำการอ่านประโยคเข้าไปและเรียนรู้จากประโยค 
โดยการพิจารณาบริบทโดยรอบของคำ อย่างเช่น Thou shalt not make a machine in the likeness of human a mind โมเดลก็จะทำการรับคำ 
1 คำเข้าไปและพิจารณาคำรอบๆ เริ่มจากว่า not ก็จะพิจารณาสองคำก่อนหน้าและสองคำหลัง (ตามจำนวน window size ในที่นี้กำหนด window size เป็น 2) 
ซึ่งจะได้มา 4 คำ คือ Thou, shalt, make และ a ดังรูปที่ \ref{fig:skipgram} ซึ่งโมเดลก็จะทำการเรียนรู้คำไปเรื่อยๆเพื่อ weight น้ำหนักภายในโมเดล

\begin{figure}[H]
    \centering
    \includegraphics{skipgram}
    \caption{แสดงการทำงานของ Skip-gram \cite{ichi}}\label{fig:skipgram}
\end{figure}

\subsubsection{CBOW}
CBOW\cite{ichi} เป็นโมเดลที่จะใช้หาความสัมพันธ์ของคำแบบจะพิจารณาคำหลายๆคำเพื่อหาคำหนึ่งคำ ดังตัวอย่างในรูปที่ \ref{fig:cbow} โดยการที่โมเดล
จะทำการอ่านประโยคเข้าไปและเรียนรู้จากประโยคเหมือนโมเดล Skip-gram อย่างเช่น Thou shalt not make a machine in the likeness of human a mind 
โมเดลก็จะทำการอ่านคำตามจำนวน window size ถ้า window size เป็น 2 ก็จะเป็นการกำหนดว่าอ่าน input เข้าไป 2 คำและคำถัดจาก 2 คำนั้นในประโยคจะเป็น
ผลลัพธ์ที่ได้จาก input  อย่างเช่นถ้าใส่ thou,shalt เข้าไปก็จะได้คำว่า not ออกมา ซึ่งจะสร้างโมเดลนี้ก็จะต้องใส่ข้อมูลประโยคจำนวนมากเพื่อทำการ weight 
ค่าน้ำหนักภายในโมเดล ให้มีประสิทธิภาพคล้ายกับโมเดล skip-gram

\begin{figure}[H]
    \centering
    \includegraphics{cbow}
    \caption{แสดงการทำงานของ CBOW \cite{ichi}}\label{fig:cbow}
\end{figure}

\section{ภาษาคอมพิวเตอร์และเทคโนโลยี }

\subsection{Open source Computer Vision (OpenCV)}

เป็นซอฟต์แวร์ที่เกี่ยวกับการประมวลผลภาพที่มีการสนับสนุนการพัฒนามาจาก Intel Corporation โดยที่ตัว OpenCV นั้นเป็น Library Open Source  โดยมีจุดประสงค์เพื่อให้นำไปต่อยอดการพัฒนาโปรแกรมในด้าน การรับรู้มองเห็นของคอมพิวเตอร์ (Computer Vision) ให้เข้าใจไม่ว่าจะเป็นภาพนิ่ง (Image) หรือจะเป็นภาพเคลื่อนไหว (Video) โดยภายในโปรเจคนี้ได้นำ OpenCV มาเป็นตัวทำการเตรียมข้อมูลรูปภาพ โดยที่นำรูปภาพที่ได้มาจากสแกนหนังสือ / หนังสือ มาทำการปรับปรุงคุณภาพรูปภาพให้เหมาะสมกับการทำส่วน Optical Character Recognition (OCR) ให้มีความแม่นยำมากยิ่งขึ้นเช่นการลบรูปภาพ การลบสิ่งที่ลบกวน การลบกรอบตาราง การหมุนรูป 

\subsection{Tesseract OCR}

เป็นหนึ่งใน Library ที่เกี่ยวกับการทำ Optical Character Recognition (OCR) ที่ถูกพัฒนาโดย Google โดยเป็น Library Open Source ที่ใช้ในการทำเกี่ยวกับ Text Detection โดยสามารถเรียกใช้งานได้ผ่าน Command line หรือจะเป็นการเรียก API ภายในโปรเกรมก็ทำได้นอกจากนั้น Tesseract เวอร์ชั่น 5.0.0 beta มีการใช้ Convolutional Neural Network (CNN) \cite{keiron} ร่วมกันกับ Long Short-Term Memory (LSTM) เพื่อให้การทำนายผลได้ดีขึ้นโดยเราจะนำตัว Tesseract มาทำเป็น OCR ภายในโปรเจคนี้

\subsection{DeepCut}

เป็น Library ในภาษา Python ที่สร้างมาจาก True Corporation โดยมีลักษณะเด่นที่ใช้ CNN (Convolutional Neural network) \cite{keiron} มาช่วยทำให้ผลลัพธ์ที่ได้ออกมามีความแม่นยำที่ค่อนข้างสูง  ซึ่งโปรเจคของเราต้องการ DeepCut เพื่อที่จะสามารถไปตัดคำจากรูปประโยคภาษาไทยที่มีความซับซ้อน และไม่แบ่งแยกชัดเจนเหมือนภาษาอังกฤษ 

\subsection{ReactJS}

เป็นหนึ่งใน Library หรือจะเรียกว่าเป็น Framework ที่ Facebook  เป็นคนสร้างขึ้นโดยที่มีหน้าที่เป็นการสร้าง UI โดยมีความคิดมากจากรูปแบบ MVC \cite{techterms} (Model View Controller) หรือก็คือเป็นตัวจัดการกับ Model กับ View ของตัวเว็บไซต์ โดยในโปรเจคนี้ได้เลือกใช้ ReactJS เป็น Front End สำหรับการทำ platform Web Application 

\subsection{Python}

Python เป็นภาษาทางโปรแกรมมิ่งซึ่งเป็นภาษาทางคอมพิวเตอร์ระดับสูงที่ออกแบบมาให้ใกล้เคียงกับภาษามนุษย์มากที่สุดเพื่อให้สามารถเข้าใจได้ง่ายมากขึ้น ซึ่งในโปรเจคมีข้อมูลที่ต้องประมวลในแต่ละครั้งมีขนาดใหญ่อาจจะทำให้เกิดความล่าช้าในแต่ละการประมวล ทางผู้จัดทำจึงเลือกใช้ Python เนื่องจากรองรับในส่วนของการทำ Thread รวมถึงนำมาใช้ในการทำ Data preparation ทั้งการเตรียมข้อมูลรูปภาพ และการเตรียมข้อมูลต่างๆหลังจากการทำ OCR นอกจากนี้ยังใช้ในการทำ Web Server อีกด้วย 

\subsubsection{Django}

เป็น REST Framework ที่ใช้ภาษา Python เป็นฐาน โดยในโปรเจคนี้เราจะนำมาสร้าง REST API  เพื่อใช้ในการใช้ Library อย่างเช่น DeepCut หรือ OCR ที่สามารถใช้ร่วมการแบ่ง Multi Thread ได้อย่างมีประสิทธิภาพ และยังสามาถจัดการข้อมูลใน database สำหรับโปรเจคนี้

\subsection{NodeJS}

NodeJS เป็นเสมือนแพลตฟอร์มที่ใช้ภาษา JavaScript ที่มี Library สำหรับใช้จัดการกับฝั่ง Server ซึ่ง NodeJS นั้นมีความยืดหยุ่นสูงที่สำหรับการจัดการ Web Server โดย Library ที่นำมาใช้คือ Express เป็น Web Server ที่เป็น RESTful API ได้
\chapter{การออกแบบและระเบียบวิธีวิจัย}

\section{System Overview}

บทนี้จะกล่าวถึงภาพรวมของระบบโดยแสดงเป็นไดอะแกรมดังรูปที่ \ref{fig:systemoverview} ซึ่งประกอบไปด้วยการออกแบบระบบฐานข้อมูล ระบบการตัดคำ ระบบการประมวลรูปภาพ และการออกแบบ interface สำหรับการใช้งาน

\begin{figure}[H]
    \centering
    \includegraphics{systemoverview}
    \caption{System Overview}\label{fig:systemoverview}
\end{figure}

\section{Feature lists}
\subsection{การแปลงเอกสารเป็นรูปภาพ}

สำหรับการแปลงเอกสารผู้ใช้จะต้องทำการอัปโหลดไฟล์ PDF ของเอกสารเข้าสู่ระบบหลังจากนั้นจะระบบจะทำการแปลงแต่ละหน้าเป็นรูปภาพ JPG เพื่อนำไปใช้ต่อในขั้นต่อไปและนำใช้แสดงภายใน web application

\subsection{Image preparation}

ในส่วนของการจัดการรูปก่อนที่จะทำการ OCR ซึ่งรูปภาพนำมา OCR นั้นมาจากการสแกนทำให้ภาพส่วนใหญ่อยู่ในสภาพดีแต่ก็ยังคงมี noise  และมีความผิดพลาดจากการสแกนเช่น ภาพเอียง หรือตัวหนังสือไม่ชัดเกิดจากการขยับในระหว่างการสแกน หรือมีพื้นหลังสีที่ทำให้ OCR ไม่มีประสิทธิภาพ ดังนั้นจึงต้องมีการทำ Image processing ก่อนที่จะผ่านไปทำ OCR
ซึ่งในการทำ Image processing นั้นทางคณะผู้จัดทำได้ออกแบบไว้ว่าจะทำการแยกระหว่างรูปและตัวหนังสือออกจากกัน โดยการใช้ contour เข้ามาช่วยในการคัดแยกรูปออกจากตัวอักษร โดยดูจากพื้นที่สี่เหลี่ยมที่ได้จาก contour กลับพื้นที่ contour ว่ามีความต่างขนาดและความแตกต่างกันมากเท่าไร หรือใช้ขนาดความกว้างและยาวมาดูว่ามีขนาดเกินเท่าไรถึงจะตัดให้เป็นรูปภาพ
นอกจากนี้ออกแบบการหมุนโดยสร้าง contour บรรทัดและวัดความเอียงของแต่ละบรรทัดว่าเอียงเท่าไรจากนั้นจึงหมุนกลับในองศาตรงข้าม


\subsection{Image to word}

สำหรับการทำแปลงเอกสารเป็นข้อมูลดิจิตอลจะใช้ Tesseract OCR โดยจะใช้รูปภาพที่ผ่านกระบวนการ Image processing และประโยคที่แปลงออกมาได้จัดเก็บไว้ใช้งานต่อไป

\subsection{Text preprocessing}

สำหรับการทำ Text preprocessing จะประกอบไปด้วยการทำ Tokenizer หรือก็คือการตัดคำออกมาจากประโยคโดยการใช้อัลกอริทึม DeepCut และนำคำไปทำ Lemmatization หรือก็คือการลดรูปให้อยู่ในรูปแบบพื้นฐานของคำศัพท์เฉพาะภาษาอังกฤษโดยใช้ library nltk เป็นตัวจัดการก่อนจะนำไปลบ stop word คือการลบคำที่ไม่มีความหมายออกไปโดยใช้กลุ่มข้อมูลของ library pythianlp ก่อนนำไปตรวจสอบคำผิดก่อนโดยใช้อัลกอริทึมของ pythianlp และตรวจเช็คคำเฉพาะที่คณะผู้จัดทำได้กำหนดไว้โดยใช้ Minimum edit distance ก่อนที่จะนำไปใช้งานต่อไป

\subsection{Tag generated}

หลังจากที่นำข้อมูลที่ได้จากการทำ OCR และทำการเตรียมข้อมูลเสร็จเรียบร้อยแล้ว ระบบจะทำการคืนคำแต่ละหน้าให้กับผู้ใช้เพื่อที่จะให้ผู้ใช้สามารถเช็คคำที่ระบบอ่าน และแก้ไขคำเหล่านี้ได้ หลังจากนั้นเมื่อผู้ใช้เช็คคำเสร็จแล้ว ระบบจะนำคำทั้งหมดที่ได้ไปคิดคำนวณเพื่อทำการสร้างคะแนนให้แต่ละคำและทำการจัดลำดับคะแนนให้กับหนังสือเล่มนั้น ๆ โดยใช้การคิดคะแนนด้วยอัลกอริทึม TF-IDF 

\subsection{Search}

ในส่วนของระบบการค้นหานั้นเมื่อผู้ใช้ทำการกรอกคำค้นหาระบบจะทำการนำคำที่ผู้ใช้กรอกมาทำ Text preprocessing อีกครั้งหนึ่งแต่จะไม่ทำในส่วนของการตรวจเช็คคำ ผิด และคำที่ได้จะถูกนำไปเข้าโมเดล Word2Vec เพื่อนำไปค้นหาคำใกล้เคียงของคำค้นหาก่อนที่จะนำคำที่ได้ไปค้นในฐานข้อมูลเพื่อค้นหาหนังสือที่มีความใกล้เคียงกับคำค้นหามากที่สุด เมื่อได้หนังสือมาระบบจะทำการส่งข้อมูลหนังสือกลับไปให้ผู้ใช้

\subsection{Manage Book}

ในการจัดการข้อมูลเอกสารภายในระบบจะแบ่งทั้ง 3 ส่วนนั้นคือ 1. การเพิ่มเอกสารเข้าสู่ระบบ 2.การแก้ไขเอกสารภายในระบบ  3. การลบเอกสารออกจากระบบ 
ส่วนที่ 1. ในการเพิ่มเอกสารเข้าสู่ระบบ ผู้ใช้งานจะต้องอัปโหลดไฟล์เอกสารในรูปแบบ PDF และกรอกรายละเอียดของเอกสารเพื่อเข้าสู่กระบวนการแปลงเอกสารเป็นรูปภาพต่อไปจะเป็นการทำ Image processing ก่อนที่จะนำมาทำการแปลงภาพเป็นตัวอักษรเพื่อที่จะได้ข้อมูลดิจิตอลจากเอกสารที่ผู้ใช้เพิ่มเข้าสู่ระบบหลังจากนั้นจะเป็นการทำ Text preprocessing และให้ผู้ใช้งานได้ตรวจสอบคำอีกนึงครั้งก่อนที่นำคำเหล่านี้ไปผ่านกระบวนการ tag generate เพื่อหาคำสำคัญในเอกสารโดยให้ผู้ใช้งานได้ตรวจสอบแก้ไขหรือเพิ่มเติมก่อนจะสิ้นสุดการเพิ่มเอกสารเข้าสู่ระบบ ในส่วนที่ 2 การแก้ไขเอกสารภายในระบบผู้ใช้งานสามารถค้นหาเอกสารภายในระบบเพื่อนำมาแก้ไขรายละเอียดที่ผู้ใช้งานกรอกเท่านั้นแต่มาสามารถแก้ไขคำที่ถูกแปลงออกมาเป็นดิจิตอลอีกครั้งได้ และส่วนสุดท้ายการลบเอกสารในระบบผู้ใช้งานสามารถลบเอกสารภายในระบบได้โดยการค้นหาเอกสารที่ต้องการและกดลบเอกสารนั้นออกจากระบบโดยเมื่อมีการลบเอกสารออกก็จะลบคำที่มีอยู่ในเอกสารออกไปจากระบบเช่นกัน

\subsection{Login}

ผู้ใช้งานสามารถเข้าสู่ระบบเพื่อใช้งานฟังก์ชั่นต่าง ๆภายในระบบโดยเมื่อผู้ใช้งานทำการเข้าสู่ระบบด้วยชื่อผู้ใช้งานและรหัสผ่านแล้วจะได้รับ “token” เพื่อที่จะใช้สำหรับการยืนยันตัวในการใช้ฟังก์ชั่นต่าง ๆภายในระบบและผู้ใช้งานจะสามารถออกจากระบบได้

\section{System requirements}

\underline{ฝั่งผู้ใช้งาน}

ใช้งานได้บนระบบ web browser 

\begin{itemize}
    \item Google Chrome เวอร์ชั่น 84.0 ขึ้นไป 
    \item Microsoft Edge เวอร์ชั่น 83.0 ขึ้นไป
    \item Firefox เวอร์ชั่น 75.0 ขึ้นไป
\end{itemize}
 
\underline{ฝั่งเซิร์ฟเวอร์}
		
ทางด้าน Hardware

        \begin{itemize}
            \item CPU: Intel or AMD processor with 64-bit โดยที่ต้องมี 2 Core ขึ้นไป
            \item GPU: NVIDIA 1050ti or higher
            \item Disk Storage: 10 GB
            \item RAM: 8GB or higher
        \end{itemize}
        
		ทางด้าน Software แบ่งเป็น 2 ส่วนคือ Python และ JavaScript 
        \begin{enumerate}
            \item Python Backend
            \begin{itemize}
                \item Python เวอร์ชั่น 3.7.5
		        \item Tensorflow เวอร์ชั่น 2.3.1
		        \item DeepCut เวอร์ชั่น 0.7
		        \item Django เวอร์ชั่น 3.1.3
		        \item Djangorestframework เวอร์ชั่น 3.12.2
		        \item Django-cors-headers เวอร์ชั่น 3.5.0
		        \item Pythainlp เวอร์ชั่น 2.2.5
		        \item Pyspellchecker เวอร์ชั่น 0.5.5
		        \item nltk เวอร์ชั่น 3.5.0
		        \item mysqlclient เวอร์ชั่น 2.0.1
		        \item pillow เวอร์ชั่น 8.0.1
		        \item shapely เวอร์ชั่น 1.7.1
		        \item pytesseract เวอร์ชั่น 5.0.0 beta
		        \item opencv-python เวอร์ชั่น 4.4.0.46
		        \item pdf2image เวอร์ชั่น 1.14.0
		        \item scipy เวอร์ชั่น 1.5.4
            \end{itemize}
            \item JavaScript Backend and Frontend
		    \begin{itemize}
                \item nodejs เวอร์ชั่น 12.16.3
		        \item apollo-server-express เวอร์ชั่น 2.19.0
		        \item axios เวอร์ชั่น 0.20.0
		        \item cors เวอร์ชั่น 2.8.5
 		        \item dotenv เวอร์ชั่น 8.2.0
		        \item express เวอร์ชั่น 4.17.1
		        \item graphql เวอร์ชั่น 15.4.0
		        \item jsonwebtoken เวอร์ชั่น 8.5.1
		        \item knex เวอร์ชั่น 0.21.5
		        \item morgan เวอร์ชั่น 1.10.0
		        \item mysql2 เวอร์ชั่น 2.2.1
		        \item password-hash เวอร์ชั่น 1.2.2
		        \item react เวอร์ชั่น 16.13.1
		        \item react-hook-form เวอร์ชั่น 6.3.1
		        \item react-router-dom เวอร์ชั่น 5.2.0
		        \item styled-components เวอร์ชั่น 5.1.1
		        \item props-types เวอร์ชั่น 15.7.2
            \end{itemize}
        \end{enumerate}
		
\section{Database Design}

\begin{figure}[H]
    \centering
    \includegraphics{er}
    \caption{แสดง ER Diagram ของฐานข้อมูล}\label{fig:er}
\end{figure}


\begin{figure}[H]
    \centering
    \includegraphics{er1}
    \caption{แสดง ER Diagram ส่วนของการเก็บความผิดพลาดในการสร้างคีย์เวิร์ดจากเอกสาร}\label{fig:er1}
\end{figure}


\begin{figure}[H]
    \centering
    \includegraphics{er2}
    \caption{แสดง ER Diagram ส่วนของคีย์เวิร์ดและคะแนนความสำคัญในระบบ}\label{fig:er2}
\end{figure}


\begin{figure}[H]
    \centering
    \includegraphics{er3}
    \caption{แสดง ER Diagram ส่วนของการเก็บคำจากแต่ละหน้าที่แปลงมาจากเอกสาร}\label{fig:er3}
\end{figure}


\begin{figure}[H]
    \centering
    \includegraphics{er4}
    \caption{แสดง ER Diagram ส่วนของประวัติของผู้ใช้งานมีการสร้างหรือแก้ไขเอกสาร}\label{fig:er4}
\end{figure}



\begin{figure}[H]
    \centering
    \includegraphics{er5}
    \caption{แสดง ER Diagram ส่วนของการเก็บข้อมูล keyword, relation, type ของเอกสาร}\label{fig:er5}
\end{figure}



\begin{figure}[H]
    \centering
    \includegraphics{er6}
    \caption{แสดง ER Diagram ส่วนของ Creator มีความเกี่ยวข้องกับเอกสารไหนบ้าง}\label{fig:er6}
\end{figure}



\begin{figure}[H]
    \centering
    \includegraphics{er7}
    \caption{แสดง ER Diagram ส่วนของ Creator Organized Name มีความเกี่ยวข้องกับเอกสารไหนบ้าง}\label{fig:er7}
\end{figure}



\begin{figure}[H]
    \centering
    \includegraphics{er8}
    \caption{แสดง ER Diagram ส่วนของ Publisher มีความเกี่ยวข้องกับเอกสารไหนบ้าง}\label{fig:er8}
\end{figure}



\begin{figure}[H]
    \centering
    \includegraphics{er9}
    \caption{แสดง ER Diagram ส่วนของ Contributor มีความเกี่ยวข้องกับเอกสารไหนบ้าง}\label{fig:er9}
\end{figure}



\begin{figure}[H]
    \centering
    \includegraphics{er10}
    \caption{แสดง ER Diagram ส่วนของ Issued Date มีความเกี่ยวข้องกับเอกสารไหนบ้าง}\label{fig:er10}
\end{figure}



\begin{figure}[H]
    \centering
    \includegraphics{er11}
    \caption{แสดง ER Diagram ส่วนของ Knex module ที่ใช้สำหรับ Migration ฐานข้อมูล}\label{fig:er11}
\end{figure}



\begin{figure}[H]
    \centering
    \includegraphics{er12}
    \caption{แสดง ER Diagram ส่วนของการเก็บประวัติการ HTTP Request NodeJS ไปยัง Django}\label{fig:er12}
\end{figure}

\subsection{Database Structure}
รูปที่ 3.2 แสดงฐานข้อมูลของทั้งระบบโดยจะมีหลัก ๆ ทั้งหมดสามส่วน ทางด้านฝั่งขวาของตาราง document จะเป็นตารางที่เก็บข้อมูลเพิ่มเติมจากตาราง document และส่วนทางด้านฝั่งซ้ายของตาราง document สำหรับการเก็บข้อมูลในด้านของการทำระบบการเก็บคำจากเอกสารที่ถูกใส่ลงมาในระบบ ระบบการแปลงคำเป็นคีย์เวิร์ดและคะแนน TF-IDF ที่นำมาใช้สำหรับการค้นหาเอกสาร ระบบจัดการฐานข้อมูลผู้ใช้งาน และการตรวจสอบความผิดพลาดที่มีโอกาสจากการสร้างคีย์เวิร์ด และส่วนสุดท้ายที่เป็นตารางที่ไม่มีการเชื่อมโยงกับตารางใด ๆ จะมีไว้สำหรับการทำระบบฐานข้อมูล และระบบตรวจสอบ HTTP Request ของทาง NodeJS

รูปที่ 3.3 จะเป็นส่วนของการเก็บข้อผิดพลาดที่มาจากระหว่างการสร้างคีย์เวิร์ด และการสร้างคะแนน Term-Frequency  โดยในส่วนนี้มีตาราง document ที่มีความสัมพันธ์ 1 to many กับตาราง django\_log กล่าวก็คือในหนึ่งเอกสารมีได้หลายบันทึกของ Django เนื่องมีโอกาสที่เกิดความผิดพลาดแล้วต้องทำใหม่จนกว่าจะไม่มีความผิดพลาดเกิดขึ้น

รูปที่ 3.4 จะเป็นส่วนของคีย์เวิร์ด และคะแนนเพื่อนำมาใช้สำหรับการค้นหาเอกสารของระบบนี้ โดยจะมีทั้งหมดสามตาราง document, term\_word, score ตาราง document จะเป็นตารางที่เก็บข้อมูลของเอกสารไว้ ส่วนตาราง term\_word จะเป็นการเก็บคีย์เวิร์ด และคะแนน IDF สำหรับการลดความสำคัญของคีย์เวิร์ดนั้น ๆ ไว้ซึ่งทั้งสองตารางนี้จะเป็นความสัมพันธ์แบบ one to many กับตาราง score ที่จะมีคะแนนสำหรับระบบการค้นหาเก็บเอาไว้ ที่มีความสัมพันธ์แบบนี้เนื่องจากในแต่ละคีย์เวิร์ดมีโอกาสพบได้ในหลายเอกสาร และเอกสารเองก็สามารถมีได้หลายคีย์เวิร์ด เนื่องจากแต่ละคีย์เวิร์ดที่อยู่ต่างเอกสารกันจะมีคะแนนไม่เท่ากัน

รูปที่ 3.5 จะเป็นส่วนของการเก็บคำที่แปลงมาจากเอกสารไว้โดยเริ่มที่ตาราง document จะที่สามารถบอกได้ว่าเอกสารไหน ที่จะมีความพันธ์ one to many ไปยังตาราง page\_in\_document ที่จะเป็นตารางที่บอกถึงหน้าต่าง ๆ ในเอกสารนั้น และยังมีความสัมพันธ์ one to many ต่อไปยังตาราง per\_term\_in\_page ที่จะมีคำต่าง ๆ เก็บเอาไว้ ดังนั้นจะเป็นความสัมพันธ์ที่เอกสารนั้นจะสามารถมีได้หลายหน้า แล้วแต่ละหน้าเองก็จะมีคำต่าง ๆ ที่แปลงออกมาถูกเก็บเอาไว้

รูปที่ 3.6 จะเป็นความสัมพันธ์ของบัญชีผู้ใช้กับเอกสาร โดยจะมีตาราง user ที่จะเก็บข้อมูลของผู้ใช้งานที่มีความสัมพันธ์แบบ one to many ไปยังตาราง document ที่จะเก็บต้องเก็บข้อมูลของผู้ใช้ไว้ว่าผู้ใช้คนไหนเป็นคนสร้าง หรือแก้ไขเอกสารนี้ ซึ่งบัญชีผู้ใช้สามารถสร้างหรือแก้ไขเอกสารได้หลายเอกสาร

รูปที่ 3.7 จะเป็นส่วนของข้อมูลของตาราง Document เหมือนกันแต่เนื่องจากข้อมูลมีมากกว่าหนึ่งทำให้ต้องสร้างความสัมพันธ์แบบ one to many กับตาราง dc\_keyword, dc\_relation, dc\_type ซึ่งจะเป็นข้อมูลคีย์เวิร์ด ความสัมพันธ์ และประเภทของเอกสารตามลำดับ

รูปที่ 3.8 จะเป็นส่วนของการเก็บความสัมพันธ์ระหว่าง Creator กับเอกสาร เนื่องจาก Creator สามารถมีได้หลายเอกสารทำให้ตาราง indexing\_creator\_document จะเป็นความสัมพันธ์แบบ one to many กับตาราง document 

รูปที่ 3.9 จะเป็นส่วนของการเก็บความสัมพันธ์ระหว่าง Creator orgname กับเอกสารเนื่อง จากCreator orgname สามารถมีได้หลายเอกสารทำให้ตาราง indexing\_creator\_orgname\_document จะเป็นความสัมพันธ์แบบ one to many กับตาราง document 

รูปที่ 3.10 จะเป็นส่วนของการเก็บความสัมพันธ์ระหว่าง Publisher กับเอกสาร เนื่องจาก Publisher สามารถมีได้หลายเอกสารทำให้ตาราง indexing\_publisher\_document จะเป็นความสัมพันธ์แบบ one to many กับตาราง document 

รูปที่ 3.11 จะเป็นส่วนของการเก็บความสัมพันธ์ระหว่าง Contributor กับเอกสาร เนื่องจาก Contributor สามารถมีได้หลายเอกสารทำให้ตาราง Indexing\_contributor\_document จะเป็นความสัมพันธ์แบบ one to many กับตาราง document 

รูปที่ 3.12 จะเป็นส่วนของการเก็บความสัมพันธ์ระหว่าง Issued Date กับเอกสาร เนื่องจาก Issued Date สามารถมีได้หลายเอกสารทำให้ตาราง indexing\_issued\_date\_document จะเป็นความสัมพันธ์แบบ one to many กับตาราง document 

รูปที่ 3.13 จะเป็นสองตารางที่บันทึกการจัดการฐานข้อมูลของเครื่องมือที่ชื่อว่า Knex ที่จะทำการจัดการสร้างฐานข้อมูล ด้วยคำสั่ง Migration แล้วหลังจากทำคำสั่งเสร็จสิ้นจะเก็บบันทึกไว้

รูปที่ 3.14 จะเป็นตารางสำหรับการเก็บ HTTP Request จาก NodeJS ที่ส่งไปทางฝั่งของ Django ซึ่งจะถูกเก็บข้อมูลไว้ในตารางนี้

\subsection{Database Dictionary}

อธิบายถึงชื่อของคอลัมน์ ความหมายและลักษณะการเก็บข้อมูลภายในฐานข้อมูลโดยที่ตารางมีทั้งหมด 18 ตารางดังนี้



\begin{table}[H]
\caption{ตารางอธิบายความหมายตาราง term\_word}\label{tbl:termword}
\begin{tabular}{|l|l|l|}
\hline
\multicolumn{1}{|c|}{ชื่อคอลัมน์} & \multicolumn{1}{c|}{ความหมาย}             & \multicolumn{1}{c|}{ประเภท}                          \\ \hline
term\_word\_id    & id   สำหรับบ่งบอกคำศัพท์                 & \makecell[l]{INT   (10) PK\\ Auto\_Increment}     \\ \hline
term              & คำศัพท์                                  & VARCHAR   (191)                                                                   \\ \hline
frequency         & จำนวนความถี่ของเอกสารที่มีคำศัพท์นี้อยู่ & INT   (191)                                                                       \\ \hline
score\_idf        & คะแนน   idf ของคำศัพท์นี้                & FLOAT   (255,4)                                                                   \\ \hline
rec\_create\_at   & วันเวลาของการเพิ่มคำศัพท์นี้เข้าสู่ระบบ  & \makecell[l]{DATETIME   (6)\\   current\_timestamp} \\ \hline
rec\_modified\_at & วันเวลาที่อัปเดทข้อมูลของคำศัพท์         & \makecell[l]{DATETIME   (6)\\  current\_timestamp} \\ \hline
\end{tabular}
\end{table}

\begin{table}[H]
\caption{ตารางอธิบายความหมายตาราง user}\label{tbl:user}
\begin{tabular}{|l|l|l|}
\hline
\multicolumn{1}{|c|}{ชื่อคอลัมน์} & \multicolumn{1}{c|}{ความหมาย}             & \multicolumn{1}{c|}{ประเภท}                                     \\ \hline
user\_id                          & id   สำหรับบ่งบอกผู้ใช้งาน                & \makecell[l]{INT   (10) PK\\    Auto\_Increment}     \\ \hline
name                              & ชื่อของผู้ใช้งาน                          & VARCHAR   (50)                                                                    \\ \hline
surname                           & นามสกุลของผู้ใช้งาน                       & VARCHAR   (191)                                                                   \\ \hline
role                              & ตำแหน่งของผู้ใช้งาน                       & VARCHAR   (191)                                                                   \\ \hline
username                          & ชื่อผู้ใช้งานสำหรับทำการ   login          & VARCHAR   (191)                                                                   \\ \hline
password                          & รหัสผ่านผู้ใช้งานสำหรับทำการ   login      & VARCHAR   (191)                                                                   \\ \hline
create\_at                        & วันเวลาของผู้ใช้งานของการเพิ่มเข้าสู่ระบบ & \makecell[l]{DATETIME   (6) \\ current\_timestamp} \\ \hline
active                            & สถานะการระงับบัญชีผู้ใช้งาน               & \makecell[l]{INT   (11) \\ Default 1}             \\ \hline
\end{tabular}
\end{table}

\begin{table}[H]
\caption{ตารางอธิบายความหมายตาราง score}\label{tbl:score}        
\begin{tabular}{|l|l|l|}
\hline
\multicolumn{1}{|c|}{ชื่อคอลัมน์} & \multicolumn{1}{c|}{ความหมาย}                                      & \multicolumn{1}{c|}{ประเภท}                                                   \\ \hline
score\_id             & id   สำหรับบ่งบอกคะแนนของคำศัพท์ &\makecell[l]{INT   (10) PK\\    \\ Auto\_Increment}      \\ \hline
score\_tf             & คะแนน   tf ของคำศัพท์            & FLOAT   (255,4)                                                                     \\ \hline
score\_tf\_idf        & คะแนน   tf-idf ของคำศัพท์        & FLOAT   (255,4)                                                                     \\ \hline
index\_term\_word\_id & id   สำหรับบ่งบอกคำศัพท์         & INT   (10)                                                                          \\ \hline
index\_document\_id   & id สำหรับบ่งบอกเอกสาร            & INT   (10)                                                                          \\ \hline
generate\_by          & คะแนนถูกคำนวณโดยใคร              &\makecell[l]{VARCHAR   (191)\\    \\ Default   ‘default’}\\ \hline
rec\_status           & สถานะการใช้คะแนนนี้              &\makecell[l]{INT   (191)\\    \\ Default 1}              \\ \hline
\end{tabular}
\end{table}

\begin{table}[H]
\caption{ตารางอธิบายความหมายตาราง pre\_term\_in\_page}\label{tbl:preterminpage}        
\begin{tabular}{|l|l|l|}
\hline
\multicolumn{1}{|c|}{ชื่อคอลัมน์} & \multicolumn{1}{c|}{ความหมาย}                                      & \multicolumn{1}{c|}{ประเภท}                                                   \\ \hline
pre\_term\_in\_page\_id           & id   สำหรับบ่งบอกคำศัพท์ชั่วคราวที่รอให้ผู้ใช้งานตรวจสอบ           & \makecell[l]{INT   (10) PK\\Auto\_Increment} \\ \hline
pre\_term                         & คำศัพท์ชั่วคราวที่รอให้ผู้ใช้ตรวจสอบ                               & VARCHAR   (191)                                                               \\ \hline
index\_page\_in\_document\_id     & id   สำหรับบ่งบอกที่อยู่ของคำศัพท์ชั่วคราวที่รอให้ผู้ใช้งานตรวจสอบ & INT (10) FK                                                                   \\ \hline
\end{tabular}
\end{table}

\begin{table}[H]
\caption{ตารางอธิบายความหมายตาราง page\_in\_document}\label{tbl:pageindocument}        
\begin{tabular}{|l|l|l|}
\hline
\multicolumn{1}{|c|}{ชื่อคอลัมน์} & \multicolumn{1}{c|}{ความหมาย}                                      & \multicolumn{1}{c|}{ประเภท}                                                   \\ \hline
page\_in\_document\_id            & id   สำหรับบ่งบอกที่อยู่ของคำศัพท์ชั่วคราวที่รอให้ผู้ใช้งานตรวจสอบ & \makecell[l]{INT   (10) PK\\Auto\_Increment} \\ \hline
page\_index                       & หน้าของเอกสาร                                                      & INT   (191)                                                                   \\ \hline
name                              & ชื่อ   File ของข้อมูล                                              & VARCHAR   (191)                                                               \\ \hline
rec\_status\_confirm              & สถานะการยืนยันโดยผู้ใช้งาน                                         & \makecell[l]{INT   (2)\\Default   2}         \\ \hline
index\_document\_id               & id สำหรับบ่งบอกเอกสาร                                              & INT (10) FK                                                                   \\ \hline
\end{tabular}
\end{table}

\begin{table}[H]
\caption{ตารางอธิบายความหมายตาราง nodejs\_log}\label{tbl:nodejslog}        
\begin{tabular}{|l|l|l|}
\hline
\multicolumn{1}{|c|}{ชื่อคอลัมน์} & \multicolumn{1}{c|}{ความหมาย}                     & \multicolumn{1}{c|}{ประเภท}                                                       \\ \hline
nodejs\_log\_id                   & id สำหรับการจัดเก็บประวัติการทำงานฝั่ง nodejs     & \makecell[l]{INT   (10) PK \\ Auto\_Increment}    \\ \hline
status\_code                      & เก็บสถานะ   HTTP หลังจากที่ส่งไปแล้วว่าได้สถานะใด & INT   (191)                                                                       \\ \hline
header\_date                      & เก็บข้อมูล   header ของ HTTP ที่ส่งไป             & VARCHAR   (191)                                                                   \\ \hline
server                            & ชื่อรูปแบบของเซิฟเวอร์ที่ส่งไป                    & VARCHAR   (191)                                                                   \\ \hline
url                               & ตำแหน่งโดเมนหรือ   IP ที่ส่งไป                    & INT   (10) FK                                                                     \\ \hline
content\_type                     & รูปแบบเนื้อหาที่ส่งไป                             & VARCHAR   (191)                                                                   \\ \hline
rec\_status                       & สถานะที่บอกว่าการส่งเกิดข้อผิดพลาดระหว่างทาง      & INT   (191)                                                                       \\ \hline
rec\_create\_date                 & วันเวลาที่ทำการส่ง   ณ ตอนนั้น                    & \makecell[l]{DATETIME   (6) \\ current\_timestamp}\\ \hline
\end{tabular}
\end{table}

\begin{table}[H]
\caption{ตารางอธิบายความหมายตาราง knex\_migrations\_lock}\label{tbl:knexmigrationslock}        
\begin{tabular}{|l|l|l|}
\hline
\multicolumn{1}{|c|}{ชื่อคอลัมน์} & \multicolumn{1}{c|}{ความหมาย}              & \multicolumn{1}{c|}{ประเภท}                                                   \\ \hline
index                             & id   บ่งบอกลำดับของไฟล์ migration ของ knex & \makecell[l]{INT   (10) PK\\Auto\_Increment} \\ \hline
is\_locked                        & สถานะของไฟล์   migration                   & INT (11)                                                                      \\ \hline
\end{tabular}
\end{table}

\begin{table}[H]
\caption{ตารางอธิบายความหมายตาราง knex\_migrations}\label{tbl:knexmigrations}        
\begin{tabular}{|l|l|l|}
\hline
\multicolumn{1}{|c|}{ชื่อคอลัมน์} & \multicolumn{1}{c|}{ความหมาย}                      & \multicolumn{1}{c|}{ประเภท}                                                   \\ \hline
id                                & id   บ่งบอกลำดับการทำงานของไฟล์ migration ของ knex & \makecell[l]{INT   (10) PK\\Auto\_Increment} \\ \hline
name                              & ชื่อไฟล์   migration ที่ถูกทำงานเรียบร้อย          & VARCHAR   (255)                                                               \\ \hline
batch                             & ลำดับที่                                           & INT   (11)                                                                    \\ \hline
migration\_time                   & เวลาที่ถุกสั่งให้ทำงาน                             & \makecell[l]{TIMESTAMP\\current\_timestamp}  \\ \hline
\end{tabular}
\end{table}

\begin{table}[H]
\caption{ตารางอธิบายความหมายตาราง indexing\_publisher\_document}\label{tbl:indexingpublisherdocument}        
\begin{tabular}{|l|l|l|}
\hline
\multicolumn{1}{|c|}{ชื่อคอลัมน์} & \multicolumn{1}{c|}{ความหมาย}      & \multicolumn{1}{c|}{ประเภท}                                                   \\ \hline
indexing\_publisher\_id           & id สำหรับบ่งบอกสำนักพิมพ์          & \makecell[l]{INT   (10) PK\\Auto\_Increment} \\ \hline
publisher                         & ชื่อสำนักพิมพ์                     & VARCHAR   (191)                                                               \\ \hline
publisher\_email                  & e-mail   ของสำนักพิมพ์             & VARCHAR   (191)                                                               \\ \hline
frequency                         & จำนวนของสำนักพิมพ์นี้ที่ถูกอ้างอิง & INT (191)                                                                     \\ \hline
\end{tabular}
\end{table}

\begin{table}[H]
\caption{ตารางอธิบายความหมายตาราง indexing\_issued\_date\_document}\label{tbl:indexingissueddatedocument}        
\begin{tabular}{|l|l|l|}
\hline
\multicolumn{1}{|c|}{ชื่อคอลัมน์} & \multicolumn{1}{c|}{ความหมาย}             & \multicolumn{1}{c|}{ประเภท}                                                   \\ \hline
indexing\_issued\_date\_id        & id สำหรับบ่งบอกปีที่เขียน                 & \makecell[l]{INT   (10) PK\\Auto\_Increment} \\ \hline
issued\_date                      & วันเวลาของปีที่เขียนเอกสาร                & DATE                                                                          \\ \hline
frequency                         & จำนวนของวันเวลาของปีที่เขียนที่ถูกอ้างอิง & INT (191)                                                                     \\ \hline
\end{tabular}
\end{table}

\begin{table}[H]
\caption{ตารางอธิบายความหมายตาราง indexing\_creator\_orgname\_document}\label{tbl:indexingcreatororgnamedocument}        
\begin{tabular}{|l|l|l|}
\hline
\multicolumn{1}{|c|}{ชื่อคอลัมน์} & \multicolumn{1}{c|}{ความหมาย}                & \multicolumn{1}{c|}{ประเภท}                                                   \\ \hline
indexing\_creator\_orgname\_id    & id สำหรับบ่งบอกชื่อหน่วยงานรับผิดชอบสังกัด   & \makecell[l]{INT   (10) PK\\Auto\_Increment} \\ \hline
creator\_orgname                  & ชื่อหน่วยงานรับผิดชอบสังกัด                  & VARCHAR   (191)                                                               \\ \hline
frequency                         & จำนวนของหน่วยงานรับผิดชอบสังกัดที่ถูกอ้างอิง & INT (191)                                                                     \\ \hline
\end{tabular}
\end{table}

\begin{table}[H]
\caption{ตารางอธิบายความหมายตาราง indexing\_creator\_document}\label{tbl:indexingcreatordocument}
\begin{tabular}{|l|l|l|}
\hline
\multicolumn{1}{|c|}{ชื่อคอลัมน์} & \multicolumn{1}{c|}{ความหมาย}       & \multicolumn{1}{c|}{ประเภท}                                                   \\ \hline
indexing\_creator\_id             & id สำหรับบ่งบอกชื่อผู้เขียนเอกสาร   & \makecell[l]{INT   (10) PK\\Auto\_Increment} \\ \hline
creator                           & ชื่อของผู้เขียนเอกสาร               & VARCHAR   (191)                                                               \\ \hline
frequency                         & จำนวนของผู้เขียนเอกสารที่ถูกอ้างอิง & INT (191)                                                                     \\ \hline
\end{tabular}
\end{table}

\begin{table}[H]
\caption{ตารางอธิบายความหมายตาราง indexing\_contributor\_document}\label{tbl:indexingcontributordocument}
\begin{tabular}{|l|l|l|}
\hline
\multicolumn{1}{|c|}{ชื่อคอลัมน์} & \multicolumn{1}{c|}{ความหมาย}              & \multicolumn{1}{c|}{ประเภท}                                                   \\ \hline
indexing\_contributor\_id         & id สำหรับบ่งบอกชื่อหน่วยข้อมูลผู้ร่วมงาน   & \makecell[l]{INT   (10) PK\\Auto\_Increment} \\ \hline
contributor                       & ชื่อหน่วยข้อมูลผู้ร่วมงาน                  & VARCHAR   (191)                                                               \\ \hline
contributor\_role                 & ตำแหน่งของหน่วยข้อมูลผู้ร่วมงาน            & VARCHAR   (191)                                                               \\ \hline
frequency                         & จำนวนของหน่วยข้อมูลผู้ร่วมงานที่ถูกอ้างอิง & INT (191)                                                                     \\ \hline
\end{tabular}
\end{table}

\begin{table}
\caption{ตารางอธิบายความหมายตาราง document}\label{tbl:document}
\vspace*{-1.25em}
\end{table}
\begin{longtable}[l]{|l|l|l|}
\hline
\multicolumn{1}{|c|}{ชื่อคอลัมน์}      & \multicolumn{1}{c|}{ความหมาย}                    & \multicolumn{1}{c|}{ประเภท}                                                       \\ \hline \endhead
document\_id                           & id สำหรับบ่งบอกเอกสาร                            & \makecell[l]{INT   (10) PK\\Auto\_Increment}     \\ \hline
status\_process\_document              & สถานะการทำงานของเอกสาร                           & INT   (2)                                                                         \\ \hline
name                                   & ชื่อไฟล์   PDF เอกสาร                            & VARCHAR   (191)                                                                   \\ \hline
version                                & ครั้งที่ตีพิมพ์                                  & INT   (255)                                                                       \\ \hline
path                                   & ตำแหน่งไฟล์   PDF ที่ผู้ใช้งานอัปโหลดเข้าสู่ระบบ & TEXT                                                                              \\ \hline
DC\_title                              & ชื่อเอกสาร                                       & VARCHAR   (191)                                                                   \\ \hline
DC\_title\_alternative                 & ชื่อรองของเอกสาร                                 & VARCHAR   (191)                                                                   \\ \hline
DC\_description\_table\_of\_contents   & สาระสำคัญที่มาจากสารบัญ                          & TEXT                                                                              \\ \hline
DC\_description\_summary\_or\_abstract & บทสรุปสาระสำคัญของหนังสือแต่ละเล่ม               & TEXT                                                                              \\ \hline
DC\_description\_note                  & รายละเอียดทั่วไปของเอกสาร                        & TEXT                                                                              \\ \hline
DC\_format                             & รูปแบบข้อมูลที่ถูกจัดเก็บในระบบ                  & VARCHAR   (191)                                                                   \\ \hline
DC\_format\_extent                     & ขนาดของไฟล์เอกสาร                                & VARCHAR   (191)                                                                   \\ \hline
DC\_identifier\_URL                    & แหล่งที่มาของเอกสาร                              & VARCHAR   (191)                                                                   \\ \hline
DC\_identifier\_ISBN                   & เลขมาตราฐานสากลของเอกสาร                         & VARCHAR   (191)                                                                   \\ \hline
DC\_source                             & หน่วยข้อมูลต้นฉบับ                               & VARCHAR   (191)                                                                   \\ \hline
DC\_language                           & ภาษาของเอกสาร                                    & VARCHAR   (191)                                                                   \\ \hline
DC\_coverage\_spatial                  & สถานที่ของเอกสารที่เป็นเจ้าของ                   & VARCHAR   (191)                                                                   \\ \hline
DC\_coverage\_temporal                 & ช่วงเวลาในหน่วยปีของเอกสาร                       & VARCHAR   (191)                                                                   \\ \hline
DC\_rights                             & ระดับการเข้าถึงของข้อมูล                         & VARCHAR   (191)                                                                   \\ \hline
DC\_rights\_access                     & ตำแหน่งที่มีสิทธิ์ในการเข้าถึงข้อมูล             & VARCHAR   (191)                                                                   \\ \hline
thesis\_degree\_name                   & ชื่อเต็มของปริญญา                                & VARCHAR   (191)                                                                   \\ \hline
thesis\_degree\_level                  & ระดับของปริญญา                                   & VARCHAR   (191)                                                                   \\ \hline
thesis\_degree\_discipline             & สาขาวิชา                                         & VARCHAR   (191)                                                                   \\ \hline
thesis\_degree\_grantor                & มหาวิทยาลัย                                      & VARCHAR   (191)                                                                   \\ \hline
rec\_create\_at                        & วันเวลาของเอกสารที่ถูกนำเข้าสู่ระบบ              & \makecell[l]{DATETIME   (6)\\current\_timestamp} \\ \hline
rec\_create\_by                        & id   สำหรับบ่งบอกผู้ใช้งานที่นำเอกสารเข้าสู่ระบบ & INT   (10) FK                                                                     \\ \hline
rec\_modified\_at                      & วันเวลาของเอกสารที่ถูกแก้ไขข้อมูล                & \makecell[l]{DATETIME   (6)\\current\_timestamp} \\ \hline
rec\_modified\_by                      & id   สำหรับบ่งบอกผู้ใช้งานที่แก้ไขเอกสารในระบบ   & INT   (10) FK                                                                     \\ \hline
index\_creator                         & id สำหรับบ่งบอกชื่อผู้เขียนเอกสาร                & INT   (10) FK                                                                     \\ \hline
index\_creator\_orgname                & id สำหรับบ่งบอกชื่อหน่วยงานรับผิดชอบสังกัด       & INT   (10) FK                                                                     \\ \hline
index\_publisher                       & id สำหรับบ่งบอกสำนักพิมพ์                        & INT   (10) FK                                                                     \\ \hline
index\_contributor                     & id สำหรับบ่งบอกชื่อหน่วยข้อมูลผู้ร่วมงาน         & INT   (10) FK                                                                     \\ \hline
index\_issued\_date                    & id สำหรับบ่งบอกปีที่เขียน                        & INT (10) FK                                                                       \\ \hline
\end{longtable}

\begin{table}[H]
\caption{ตารางอธิบายความหมายตาราง django\_log}\label{tbl:djangolog}
\begin{tabular}{|l|l|l|}
\hline
\multicolumn{1}{|c|}{ชื่อคอลัมน์} & \multicolumn{1}{c|}{ความหมาย}                 & \multicolumn{1}{c|}{ประเภท}                                                       \\ \hline
django\_log\_id                   & id สำหรับการจัดเก็บประวัติการทำงานฝั่ง django & \makecell[l]{INT   (10) PK\\Auto\_Increment}    \\ \hline
rec\_status                       & สถานะการทำงานที่เกิดขึ้น                      & INT   (191)                                                                       \\ \hline
rec\_create\_date                 & วันเวลาของการทำงานที่เกิดขึ้น                 & \makecell[l]{DATETIME   (6)\\current\_timestamp}\\ \hline
log\_error                        & ข้อมูลข้อผิดพลาดที่เกิดขึ้น                   & VARCHAR   (191)                                                                   \\ \hline
index\_document                   & id สำหรับบ่งบอกเอกสารที่ทำงาน                 & INT (10) FK                                                                       \\ \hline
\end{tabular}
\end{table}

\begin{table}[H]
\caption{ตารางอธิบายความหมายตาราง dc\_type}\label{tbl:dctype}
\begin{tabular}{|l|l|l|}
\hline
\multicolumn{1}{|c|}{ชื่อคอลัมน์} & \multicolumn{1}{c|}{ความหมาย}  & \multicolumn{1}{c|}{ประเภท}                                                   \\ \hline
DC\_type\_id                      & id สำหรับบ่งบอกประเภทของเอกสาร & \makecell[l]{INT   (10) PK\\Auto\_Increment} \\ \hline
DC\_type                          & ประเภทของเอกสาร                & VARCHAR   (191)                                                               \\ \hline
index\_document\_id               & id สำหรับบ่งบอกเอกสาร          & INT (10)                                                                      \\ \hline
\end{tabular}
\end{table}

\begin{table}[H]
\caption{ตารางอธิบายความหมายตาราง dc\_relation}\label{tbl:dcrelation}
\begin{tabular}{|l|l|l|}
\hline
\multicolumn{1}{|c|}{ชื่อคอลัมน์} & \multicolumn{1}{c|}{ความหมาย}      & \multicolumn{1}{c|}{ประเภท}                                                   \\ \hline
DC\_relation\_id                  & id สำหรับบ่งบอกเอกสารที่เกี่ยวข้อง & \makecell[l]{INT   (10) PK\\Auto\_Increment} \\ \hline
DC\_relation                      & ชื่อเอกสารที่เกี่ยวข้อง            & VARCHAR   (191)                                                               \\ \hline
index\_document\_id               & id สำหรับบ่งบอกเอกสาร              & INT (10)                                                                      \\ \hline
\end{tabular}
\end{table}

\begin{table}[H]
\caption{ตารางอธิบายความหมายตาราง dc\_keyword}\label{tbl:dckeyword}
\begin{tabular}{|l|l|l|}
\hline
\multicolumn{1}{|c|}{ชื่อคอลัมน์} & \multicolumn{1}{c|}{ความหมาย}      & \multicolumn{1}{c|}{ประเภท}                                                   \\ \hline
DC\_keyword\_id                   & id สำหรับบ่งบอก tag           & \makecell[l]{INT   (10) PK\\Auto\_Increment} \\ \hline
DC\_keyword                       & คำศัพท์                       & VARCHAR   (191)                                                               \\ \hline
index\_document\_id               & id สำหรับบ่งบอกเอกสาร              & INT (10)                                                                      \\ \hline
\end{tabular}
\end{table}

\section{UML Design}
\subsection{Use case diagram}

\begin{figure}[H]
    \centering
    \includegraphics{usecasediagram}
    \caption{Use case diagram}\label{fig:usecasediagram}
\end{figure}

\subsection{Sequence diagram}

\subsubsection{Use case Add Document}

Scenario 1: เพิ่มหนังสือ/เอกสารเข้าสู่ระบบ

Goal: เพิ่มข้อมูลของเอกสารเข้าไปอยู่ในระบบ

Precondition: กดไปที่หัวข้อ INSERT BOOK ใน Web Application 

Main success scenario:

\begin{enumerate}
    \item อัพโหลดเอกสาร/หนังสือเลือกหน้าที่จะให้เริ่มต้นการแปลง
    \item กรอกข้อมูลรายละเอียดที่ต้องการลงในระบบ
    \item แสดงสถานะของการเพิ่มข้อมูล
    \item เพิ่มเอกสาร/หนังสือเข้าสู่ระบบ
\end{enumerate}

\begin{figure}[H]
    \centering
    \includegraphics{scene1}
    \caption{แสดง Scenario 1 เพิ่มเอกสารเข้าระบบ}\label{fig:scene1}
\end{figure}

\subsubsection{Use case Manage word in document}

Scenario 2: การตรวจสอบและแก้ไขคำก่อนนำเข้าสู่ระบบ

Goal: ผู้ใช้งานเห็นคำที่จะถูกการแปลงเป็นดิจิตอลแล้วสามารถจัดการคำเหล่านั้นได้

Precondition: อยู่ภายในขั้นตอนการเพิ่มหนังสือ/เอกสารลงในระบบ

Main success scenario:

\begin{enumerate}
    \item ผู้ใช้เข้าไปยังหน้าดูสถานะการเพิ่มเอกสาร
    \item ผู้ใช้เลือกเอกสารที่อยู่ในสถานะตรวจสอบคำ
    \item ระบบแสดงคำทั้งหมดที่ถูกแปลงมาได้จากเอกสารแต่ละหน้า
    \item ผู้ใช้ตรวจสอบ แก้ไขคำที่แสดงขึ้นมา
    \item ยืนยันขั้นตอนการตรวจสอบและแก้ไขคำ
\end{enumerate}
\begin{figure}[H]
    \centering
    \includegraphics{scene2}
    \caption{แสดง Scenario 2 การจัดการคำที่ถูกเก็บได้จากเอกสารในระบบ}\label{fig:scene2}
\end{figure}

\subsubsection{Use case Verify Document to Generate Keyword}

Scenario 3: ยื่นเอกสารว่าพร้อมสำหรับการถูกนำไปสร้างคียเวิร์ด

Goal: เอกสารถูกยืนยันพร้อมกับสร้างคีย์เวิร์ดเพื่อเพิ่มเข้าไปในระบบ

Precondition: ไปยังหน้าสถานะของเอกสารแล้วกดไปยังปุ่มยืนยันเอกสารถูกต้อง

Main success scenario:

\begin{enumerate}
    \item ผู้ใช้เข้าไปยังหน้าดูสถานะการเพิ่มเอกสาร
    \item ระบบแสดงสถานะเอกสารว่าเอกสารไหนอยู่สถานะใดแล้วบ้าง
    \item ผู้ใช้กดยืนยันว่าเอกสารถูกต้อง
    \item ระบบย้ายไปหน้าสถานะเอกสารอีกครั้งเพื่อรอผลการทำงาน
    \item ระบบแสดงการยืนยันเอกสาร และถูกเพื่อคีย์เวิร์ดเสร็จสิ้น
\end{enumerate}
\begin{figure}[H]
    \centering
    \includegraphics{scene3}
    \caption{แสดง Scenario 3 ยืนเอกสารว่าพร้อมสำหรับการถูกนำไปสร้างคีย์เวิร์ด}\label{fig:scene3}
\end{figure}

\subsubsection{Use case Edit Document}

Scenario 4: การแก้ไขรายละเอียดของเอกสาร/หนังสือที่อยู่ภายในระบบ

Goal: รายละเอียดเอกสารถูกแก้ไขตามผู้ใช้งานต้องการ

Precondition: กดไปที่หัวข้อ MANAGE BOOK ใน Web Application

Main success scenario:

\begin{enumerate}
    \item ผู้ใช้ค้นหาเอกสารที่ต้องการแก้ไขรายละเอียด
    \item แสดงผลลัพธ์ในการค้นหาเอกสาร/หนังสือ
    \item เลือกเอกสาร/หนังสือที่ต้องการแก้ไขรายละเอียด
    \item แก้ไขรายละเอียดที่ต้องการ
    \item กดบันทึกข้อมูลลงไปในระบบ
\end{enumerate}
\begin{figure}[H]
    \centering
    \includegraphics{scene4}
    \caption{แสดง Scenario 4 แก้ไขข้อมูลเอกสาร}\label{fig:scene4}
\end{figure}

\subsubsection{Use case Delete Document}

Scenario 5: ลบเอกสาร/หนังสือภายในระบบ

Goal: เอกสาร/หนังสือถูกนำออกจากระบบ

Precondition: กดเลือกหัวข้อ MANAGE BOOK ใน Web Application

Main success scenario:

\begin{enumerate}
    \item ผู้ใช้ทำการค้นหาเอกสารหนังสือที่ต้องการจะลบออกจากระบบ
    \item แสดงผลลัพธ์ในการค้นหาเอกสาร/หนังสือ
    \item กดลบเอกสาร/หนังสือที่ต้องการ
    \item กดยืนยันคำสั่งลบเพื่อบันทึกลงระบบ
\end{enumerate}
\begin{figure}[H]
    \centering
    \includegraphics{scene5}
    \caption{แสดง Scenario 5 ลบเอกสาร}\label{fig:scene5}
\end{figure}

\subsubsection{Use case View Document \& Search Document}

Scenario 6: ดูข้อมูลเอกสาร และการค้นหาเอกสาร

Goal: ผู้ใช้เจอเอกสารที่ต้องการ

Precondition: กดไปที่หัวข้อ SEARCH ใน Web Application

Main success scenario:

\begin{enumerate}
    \item กรอกรายละเอียดข้อมูลที่ต้องการจะค้นหา
    \item แสดงผลลัพธ์ในการค้นหา
    \item ผู้ใช้เลือกเอกสารที่ต้องการที่จะดูข้อมูล
    \item ระบบย้ายไปยังหน้าแสดงข้อมูลเอกสารที่ถูกเลือก
\end{enumerate}
\begin{figure}[H]
    \centering
    \includegraphics{scene6}
    \caption{แสดง Scenario 6 ดูข้อมูลเอกสาร และการค้นหาเอกสาร}\label{fig:scene6}
\end{figure}

\subsubsection{Use case Login}

Scenario 7: ระบบล็อกอิน

Goal: เพื่อเข้าสู่ระบบให้สามารถใช้ฟังก์ชั่นภายใน Web Application เพิ่มเติมได้

Precondition: กดหัวข้อ LOGIN ใน Web Application

Main success scenario:

\begin{enumerate}
    \item ผู้ใช้กรอกชื่อผู้ใช้งานและรหัสผ่าน
    \item กดเข้าสู่ระบบ
    \item เข้าสู่ระบบสำเร็จ ส่งผู้ใช้กลับไปสู่ Homepage
    \item สามารถเข้าใช้งานฟังก์ชั่นของ Web Application ได้
\end{enumerate}
\begin{figure}[H]
    \centering
    \includegraphics{scene7}
    \caption{แสดง Scenario 7 ระบบล็อกอิน}\label{fig:scene7}
\end{figure}

\section{GUI Design}

\subsection{Homepage}
\begin{figure}[H]
    \centering
    \includegraphics[scale=0.3]{hp}
    \caption{ภาพแสดงหน้าหลักของเว็บไซต์}\label{fig:hp}
\end{figure}
หน้าหลักของเว็บไซต์จะเป็นหน้าที่เน้นการค้นหาเป็นหลัก ที่ผู้ใช้สามารถเข้าถึงเมนูการเพิ่มหนังสือ การจัดการ และการเข้าสู่ระบบได้ที่แถบ Navigation ด้านบนของเว็บไซต์ดังรูปที่ \ref{fig:hp}

\subsection{Homepage2}
\begin{figure}[H]
    \centering
    \includegraphics[scale=0.3]{hp2}
    \caption{ภาพแสดงหน้าหลักของเว็บไซต์หลังจากการกดเปิดเมนู}\label{fig:hp2}
\end{figure}
เมื่อกดปุ่มลูกศรที่ด้านล่างของรูป \ref{fig:hp} จะมีเมนูเพิ่มเติมขึ้นมากลายเป็นรูปที่ \ref{fig:hp2} ซึ่งจะแสดงรายละเอียดในแต่ละฟังก์ชั่นเพิ่มเติม

\subsection{Login}
\begin{figure}[H]
    \centering
    \includegraphics[scale=0.3]{login}
    \caption{ภาพแสดงหน้าเข้าสู่ระบบ}\label{fig:scene7}
\end{figure}
ก่อนที่จะทำการเพิ่มหนังสือหรือจัดการกับหนังสือผู้ใช้นั้นจะต้องเข้าสู่ระบบก่อนเสมอ ถ้าเกิดกดเข้าฟังก์ชั่นการเพิ่มหนังสือหรือค้นหาโดยที่ยังไม่ได้เข้าสู่ระบบ ระบบจะบังคับให้ผู้ใช้เข้ามาในหน้าเข้าสู่ระบบดังรูป 3.25 เพื่อทำการเข้าสู่ระบบหรือจะเข้ามาโดยการกด log in ที่ปุ่มขวาบนได้

\subsection{Insert Book(1)}
\begin{figure}[H]
    \centering
    \includegraphics[scale=0.27]{i1}
    \caption{ภาพแสดงขั้นตอนการเพิ่มหนังสือเข้าสู่ระบบขั้นเลือกไฟล์}\label{fig:i1}
\end{figure}
หน้าเพิ่มหนังสือขั้นแรกจะเป็นการเลือกไฟล์เอกสารที่ต้องการโดยที่จะมีส่วนของการเพิ่มไฟล์ที่อยู่รูปของ pdf เพื่อทำ OCR จากนั้นจะสามารถเลือกได้ว่าจะทำการ OCR ตั้งแต่หน้าไหนดังรูปที่ 3.26

\subsection{Insert Book (2) }
\begin{figure}[H]
    \centering
    \includegraphics[scale=0.3]{i2}
    \caption{ภาพแสดงขั้นตอนการเพิ่มหนังสือเข้าสู่ระบบขั้นกรอกข้อมูลขั้นที่ 1}\label{fig:i2}
\end{figure}
หน้าเพิ่มหนังสือขั้นตอนที่ 2 เป็นหน้าที่ต้องใส่ข้อมูลที่จำเป็นของหนังสือ โดยที่จำเป็นต้องใส่จะมีสัญลักษณ์กำกับไว้หรือก็คือชื่อหนังสือดังรูป 3.27 โดยในหน้านี้จะมีกล่องใส่ข้อมูลที่ถูกกรอกบ่อย ๆสำหรับผู้ใช้(เจ้าหน้าที่)

\subsection{Insert Book (3)}
\begin{figure}[H]
    \centering
    \includegraphics[scale=0.3]{i3}
    \caption{ภาพแสดงขั้นตอนการเพิ่มหนังสือเข้าสู่ระบบขั้นกรอกข้อมูลขั้นที่ 2}\label{fig:i3}
\end{figure}
ในขั้นตอนที่ 3 จากรูปที่ 3.28 จะเป็นหน้าที่ใส่ข้อมูลที่ส่วนใหญ่ผู้ใช้จะไม่ค่อยกรอกมากนัก ซึ่งไม่มีกล่องข้อมูลไหนจำเป็นที่ต้องกรอกผู้ใช้สามารถข้ามไปขั้นตอนถัดไปได้เลย

\subsection{Insert Book (4)}
\begin{figure}[H]
    \centering
    \includegraphics[scale=0.3]{i4}
    \caption{ภาพแสดงขั้นตอนการเพิ่มหนังสือเข้าขั้นโหลดข้อมูลเข้าสู่ระบบ}\label{fig:i4}
\end{figure}
หลังจากที่ทำการใส่ข้อมูลออกมาทั้งหมดแล้วมาถึงหน้าที่เป็นหน้าโหลดข้อมูลดังรูป 3.29 ที่ระบบจะทำการ OCR และทำการเตรียมชุดข้อมูลที่ได้จากการ OCR โดยการนำคำมาตัดและเช็คคำผิด เมื่อโหลดข้อมูลเสร็จแล้วระบบจะทำการเปลี่ยนสถานะการโหลดและขึ้นลิ้งเพื่อเข้าสู่ขั้นตอนถัดไปได้

\subsection{Insert Book (5)}
\begin{figure}[H]
    \centering
    \includegraphics[scale=0.3]{i5}
    \caption{ภาพแสดงขั้นตอนการเพิ่มหนังสือเข้าสู่ระบบขั้นแก้ไขคำผิด}\label{fig:i5}
\end{figure}
หลังจากโหลดและเตรียมข้อมูลเรียบร้อยแล้ว ระบบจะทำการแสดงข้อมูลที่ถูกแปลงมาโดยที่ผู้ใช้จะสามารถแก้ไขคำได้ดังรูป 3.30 หรือสามารถข้ามได้เลยเช่นกัน โดยเมื่อคลิกไปที่กล่องข้อความจะขึ้นให้แก้แต่ละคำและเมื่อเปลี่ยนหน้าจะทำการเก็บข้อมูลที่เปลี่ยนไว้ และจะบันทึกการแก้ไขข้อมูลทั้งหมดที่แก้เมื่อข้ามไปขั้นตอนถัดไป

\subsection{Insert Book (6) }
\begin{figure}[H]
    \centering
    \includegraphics[scale=0.3]{i6}
    \caption{ภาพแสดงขั้นตอนการเพิ่มหนังสือเข้าสู่ระบบขั้นแก้ไขและเพิ่มคำสำคัญ}\label{fig:i6}
\end{figure}
หน้าสุดท้ายของการเพิ่มหนังสือจะเป็นหน้าที่ให้ผู้ใช้สามารถจัดการกับ Keyword ได้ดังรูปที่ 3.31 โดยเมื่อผู้ใช้ต้องการใส่คำสำคัญเพิ่มสามารถกด ADD เพื่อเพิ่มคำที่ต้องการใส่ได้ และสามารถลบเมื่อคลิกที่ปุ่มกากบาทที่คำสำคัญที่ระบบทำการสร้างมาให้ เมื่อแก้ไขเสร็จแล้วสามารถกดปุ่ม Finish เพื่อทำการบันทึกข้อมูล

\subsection{Search}
\begin{figure}[H]
    \centering
    \includegraphics[scale=0.3]{search}
    \caption{ภาพแสดงหน้าค้นหาข้อมูล}\label{fig:search}
\end{figure}
หน้าแสดงข้อมูลการค้นหาเมื่อทำการค้นหาข้อมูลจากหน้าแรก (รูปที่ 3.23 หรือ 3.24) จะทำการแสดงข้อมูลหนังสือที่ตรงกับ keyword โดยเรียงคะแนนของหนังสือที่เกี่ยวข้องกับคำค้นหามากที่สุดดังรูปที่ 3.32 เมื่อกดเข้าไปที่รายชื่อหนังสือจะทำการนำทางผู้ใช้ไปยังหน้าดูหนังสือดังรูปที่ 3.33

\subsection{Document View}
\begin{figure}[H]
    \centering
    \includegraphics[scale=0.3]{lookupp}
    \caption{ภาพแสดงหน้าดูหนังสือ}\label{fig:lookupp}
\end{figure}
เมื่อเราค้นหาและเลือกหนังสือ ก็จะมีหน้าหนังสือ (รูปที่ 3.33) ขึ้นมาให้ดูเนื้อหาภายในโดยที่ผู้ใช้สามารถปรับขนาดภาพและสามารถเลือกหน้าที่ต้องการจะเปิดได้และสามารถย้อนหลับไปยังหน้าเสริชได้ที่ปุ่มลูกศรทางด้านซ้ายบน

\subsection{Manage book}
\begin{figure}[H]
    \centering
    \includegraphics[scale=0.3]{man1}
    \caption{ภาพแสดงหน้าการจัดการหนังสือที่เพิ่มเข้าสู่ระบบ}\label{fig:man1}
\end{figure}
ในหน้าของการจัดการหนังสือดังรูปที่ 3.34 จะมีลักษณะคล้ายกับหน้าการค้นหาเพียงแต่ว่าจะมีฟังก์ชั่นสำหรับการแก้ไขเนื้อหนังสือภายในที่ผู้ใช้เคยกรอกไว้ตอน OCR หนังสือมา เมื่อกดปุ่มลบจะมีหน้าต่างแจ้งเตือนเพื่อถามความแน่ใจในการลบเอกสาร หรือกดปุ่ม Edit เพื่อทำการเข้าสู่การแก้ไขข้อมูลของเอกสารนั้นๆดังรูปที่ 3.35 - 3.37

\subsection{Edit Book}
\begin{figure}[H]
    \centering
    \includegraphics[scale=0.3]{e1}
    \caption{ภาพแสดงขั้นตอนการแก้ไขหนังสือขั้นที่ 1}\label{fig:e1}
\end{figure}

\begin{figure}[H]
    \centering
    \includegraphics[scale=0.3]{e2}
    \caption{ภาพแสดงขั้นตอนการแก้ไขหนังสือขั้นที่ 2}\label{fig:e2}
\end{figure}

\begin{figure}[H]
    \centering
    \includegraphics[scale=0.3]{e3}
    \caption{ภาพแสดงขั้นตอนการแก้ไขหนังสือขั้นที่ 3}\label{fig:e3}
\end{figure}

หน้าแก้ไขหนังสือแบ่งออกเป็น 3 ขั้นตอนดังรูป 3.35 - 3.37 ซึ่งจะมีให้แก้ไข ข้อมูลที่เคยกรอกไว้ตอนเพิ่มหนังสือเข้ามา โดยจะมีรูปปกหนังสือและชื่อหนังสือคอยบอกว่ากำลังแก้ไขหนังสือเล่มไหนอยู่ และในทุกหน้าจะมีปุ่มสำหรับบันทึกในทุกหน้าเพื่อที่จะสามารถบันทึกโดยที่ไม่ต้องรอไปหน้าสุดท้ายเพื่อบันทึกข้อมูล

\subsection{Upload Status Page}
\begin{figure}[H]
    \centering
    \includegraphics[scale=0.3]{status}
    \caption{ภาพแสดงหน้าการโหลดข้อมูล}\label{fig:status}
\end{figure}
จากรูป 3.38 สำหรับผู้ใช้ที่ทำการเพิ่มเอกสารเข้าสู่ระบบจะมีหน้าสำหรับโหลดกรณีที่กดออกมาหลังจากผ่านขั้นตอนการเพิ่มหนังสือขั้นตอนที่ 4 จะสามารถเข้ามาดูสถานะและทำการดำเนินการต่อได้โดยไม่ต้องผ่านการเพิ่มหนังสือเข้าสู่ระบบใหม่


\subsection{Evaluate Process Design}

ในส่วนของการประเมินผลการทำงานนั้นจะแบ่งออกเป็น 2 ส่วนคือ ส่วนของการทำ image processing จะช่วยให้การทำ OCR มีประสิทธิภาพมากเท่าไร และส่วนของระบบการค้นหา โดยในส่วนของ OCR จะทำการประเมินจากการเลือกเช็คคำจาก 2 หน้าของแต่ละเอกสารมาเช็คว่าแต่ละหน้ามีคำผิดเท่าไร โดยจะเลือกวัดเอกสารทั้งหมด 5 เล่มแบบสุ่มและเทียบการทำ Image processing ว่าทำแบบไหนได้ผลลัพธ์แบบไหนออกมา

\begin{table}[H]
\caption{ตารางประเมินการทำ OCR}\label{tbl:ocr}
\begin{tabular}{|c|c|c|c|c|}
\hline
\multicolumn{5}{|c|}{ตารางประเมินการทำ OCR}                 \\ \hline
หนังสือ & หน้า & จำนวนคำทั้งหมด & คำที่ผิด(\%) & คำเกิน(คำ) \\ \hline
        &      &                &              &            \\ \hline
\end{tabular}
\end{table}

ระบบการค้นหา จะเช็คโดยให้ผู้ใช้เป็นผู้ประเมินว่าได้รับเอกสารตรงตามที่ต้องการหรือไม่โดยจะให้เจ้าหน้าที่บรรณารักษ์คัดเลือกหนังสือจำนวน 3 เล่มที่คาดหวังว่าจะขึ้นมาเมื่อค้นหาทั้งหมด 10 ครั้ง

\begin{table}[H]
\caption{ตารางประเมินระบบการค้นหา}\label{tbl:searchtest}
\begin{tabular}{|c|l|p{0.40\linewidth}|}
\hline
\multicolumn{3}{|c|}{ตารางประเมินระบบการค้นหา}                                                                                                                                                                                                                                                                                                                                                                                                                                                                                                                                                \\ \hline
คำค้นหา                & \multicolumn{1}{c|}{หนังสือที่คาดหวัง} & \multicolumn{1}{c|}{การค้นหา}                                                                                                                                                                                                                                                                                                                                                                                                                                                                                               \\ \hline
\multicolumn{1}{|l|}{} &                                        & \makecell[l]{คะแนน 5 ระดับ\\ 5   = ค้นหาหนังสือได้ตรงตามที่ต้องการ \\และมีหนังสือที่เกี่ยวข้องกับคำค้นหาขึ้นมาอย่างถูกต้อง\\ 4   = ค้นหาหนังสือได้ถูกต้องตามที่ต้องการ\\บางเล่มและมีหนังสือที่เกี่ยวข้องกับคำค้นหาขึ้นมา\\ 3   = ไม่สามารถค้นหาหนังสือที่ต้องการแต่\\มีหนังสือที่เกี่ยวข้องกับคำค้นหาขึ้นมา\\ 2   = สามารถค้นหาหนังสือที่มีความเกี่ยวข้องกับคำค้นหา \\และมีหนังสือที่ไม่เกี่ยวข้องกับการค้นหาแสดงในผลัพธ์\\ 1 = ไม่มีหนังสือที่เกี่ยวข้องขึ้นมาในผลลัพธ์} \\ \hline
\end{tabular}
\end{table}

\begin{table}[H]
\caption{ตารางประเมิน Design}\label{tbl:design}
\begin{tabular}{|l|l|l|l|l|}
\hline
\multicolumn{5}{|c|}{ตารางประเมิน Design}                                                                                                                                                            \\ \hline
\multicolumn{1}{|c|}{\multirow{2}{*}{เกณฑ์การประเมิน}} & \multicolumn{2}{c|}{ผลลัพธ์}                             & \multicolumn{1}{c|}{\multirow{2}{*}{หมายเหตุ}} & \multicolumn{1}{c|}{คำเกิน(คำ)} \\ \cline{2-3} \cline{5-5} 
\multicolumn{1}{|c|}{}                                 & \multicolumn{1}{c|}{ผ่าน} & \multicolumn{1}{c|}{ไม่ผ่าน} & \multicolumn{1}{c|}{}                          & \multicolumn{1}{c|}{}           \\ \hline
1.หน้าเข้าสู่ระบบ                                      &                           &                              &                                                &                                 \\ \hline
2.Insert Book                                          &                           &                              &                                                &                                 \\ \hline
3.Search                                               &                           &                              &                                                &                                 \\ \hline
4.Manage Book                                          &                           &                              &                                                &                                 \\ \hline
5.View Book                                            &                           &                              &                                                &                                 \\ \hline
6.Home page                                            &                           &                              &                                                &                                 \\ \hline
7.Status Page                                          &                           &                              &                                                &                                 \\ \hline
\end{tabular}
\end{table}

\begin{table}[H]
\caption{ตารางประเมิน test}\label{tbl:test}
\begin{tabular}{|l|l|l|l|}
\hline
\multicolumn{4}{|c|}{ตารางประเมิน Test}                                                                                                                                             \\ \hline
\multicolumn{1}{|c|}{\multirow{2}{*}{เกณฑ์การประเมิน}}                  & \multicolumn{2}{c|}{ผลลัพธ์}                             & \multicolumn{1}{c|}{\multirow{2}{*}{หมายเหตุ}} \\ \cline{2-3}
\multicolumn{1}{|c|}{}                                                  & \multicolumn{1}{c|}{ผ่าน} & \multicolumn{1}{c|}{ไม่ผ่าน} & \multicolumn{1}{c|}{}                          \\ \hline
1. สามารถเข้าสู่ระบบและออกจาก ระบบได้                                   &                           &                              &                                                \\ \hline
2. สามารถเพิ่มเอกสารเข้าสู่ระบบได้                                      &                           &                              &                                                \\ \hline
3. สามารถแก้ไขรายละเอียดเอกสารที่อยู่ในระบบได้                          &                           &                              &                                                \\ \hline
4.   สามารถตรวจสอบและแก้ไขคำที่เพิ่มเข้ามาในระบบในขั้นตอนเพิ่มเอกสารได้ &                           &                              &                                                \\ \hline
5. สามารถลบเอกสารที่อยู่ในระบบได้                                       &                           &                              &                                                \\ \hline
6.   สามารถค้นหาข้อมูลเอกสารภายในระบบได้                                &                           &                              &                                                \\ \hline
7.   สามารถเรียกดูเอกสารที่ต้องการได้                                   &                           &                              &                                                \\ \hline
\end{tabular}
\end{table}
\chapter{ผลการดำเนินงาน}

การดำเนินงานของโปรเจคนี้จะแบ่งออกมาเป็นทั้งหมด 3 ส่วน โดยส่วนแรกคือส่วนของการจัดเก็บข้อมูลเข้าสู่ระบบโดยนำรูปภาพได้ที่ได้รับมาผ่านกระบวนการการเตรียมข้อมูลรูปภาพ 
ก่อนจะนำไปผ่านกระบวนการ OCR และการเตรียมข้อมูลตัวหนังสือ ก่อนจะถูกเก็บข้อมูลในระบบ ส่วนที่สองการค้นหาข้อมูล เป็นการค้นหาแบบ IR (Information retrieval) 
ที่จะนำไปโมเดล Word2Vec เข้ามาช่วยในการค้นหาคำที่มีความสัมพันธ์ใกล้เคียงกับคำค้นหา และนำคะแนน TF-IDF มาใช้เป็นคะแนนในการค้นหา และส่วนสุดท้ายคือส่วนของการทำแพลตฟอร์มเว็ปไซต์
ซึ่งในการประเมินผลการดำเนินงานนั้นเราจะทำการประเมินในส่วนแรก โดยการประเมินความถูกต้องของการทำ OCR จะมีเจ้าหน้าที่บรรณารักษ์กำหนดเกณฑ์ไว้ 
ซึ่งเกณฑ์ที่กำหนดในส่วนของความถูกต้องในการทำ OCR อยู่ที่ 75 \% และความแม่นยำในการค้นหาอยู่ที่ 75 \%

\section{ผลลัพธ์ที่ได้จากการแปลงข้อมูลรูปภาพให้เป็นข้อมูลดิจิทัล}

\subsection{ผลลัพธ์ที่ได้จากประสิทธิภาพของการหมุน}
ผลลัพธ์จากการหมุนภาพตัวหนังสือทั้ง 978 ภาพ มีความคลาดเคลื่อนทั้งหมด 7.98\% ที่ยังไม่สามารถหมุนภาพให้ตรง และทำให้บางภาพแย่ลง เนื่องจากว่าบรรทัดตัวอักษรอาจจะมีสระที่ไม่สามารถทำกร่อนให้กลายเป็นเส้นบรรทัดได้

\begin{figure}[H]
    \centering
    \includegraphics[scale=0.4]{rotateres}
    \caption{ภาพแสดงผลลัพธ์การหมุนรูป}\label{fig:rotateres}
\end{figure}

\subsection{ผลการเปรียบเทียบประสิทธิภาพในการทำ OCR ของ การทำการเตรียมข้อมูลรูปภาพ แต่ละแบบ}

จากการทดสอบประสิทธิภาพของการทำการเตรียมข้อมูลรูปภาพทั้งสองแบบพบว่า การทำการเตรียมข้อมูลรูปภาพ แบบแรกนั้นมีจำนวนคำผิดน้อยกว่า แต่มีจำนวนคำที่ไม่สามารถแปลงเป็นดิจิทัลมากถึง 32.71\% ดังตารางที่ \ref{tbl:imagep1} ซึ่งต่างจากการทำการเตรียมข้อมูลรูปภาพ แบบที่ 2 ที่มีค่าความถูกต้องของคำ 74.74 \% ดังตาราง \ref{tbl:imagep2}

\subsubsection{แบบที่ 1 การใช้การคัดเลือกข้อมูล,การหมุน,การลบรูปภาพ,การลบเส้น และการจัดกลุ่ม}
\begin{table}[H]
    \caption{ตารางประเมินการทำการเตรียมข้อมูลรูปภาพแบบที่ 1 }\label{tbl:imagep1}
    \begin{tabular}{|c|c|c|p{0.1\linewidth}|p{0.1\linewidth}|c|p{0.1\linewidth}|p{0.1\linewidth}|}
        \hline
        หนังสือ                             & หน้า  & จำนวนคำทั้งหมด & จำนวนคำผิดที่ตรวจพบ & เปอร์เซ็นต์คำผิดที่ตรวจพบ(\%)    & จำนวนคำเกิน & จำนวนคำที่ไม่สามารถแปลงเป็นดิจิทัล & เปอร์เซ็นต์คำที่ไม่สามารถแปลงเป็นดิจิทัล(\%)    \\ \hline
        \multirow{2}{*}{กตเวทิตาปี 2542}      & 15    & 4         & \multicolumn{1}{c|}{4  }         & \multicolumn{1}{c|}{100 \%  } & \multicolumn{1}{c|}{0  }    & \multicolumn{1}{c|}{0  }             & \multicolumn{1}{c|}{0 \%    }\\ \cline{2-8} 
                                            & 29    & 252       & \multicolumn{1}{c|}{14 }         & \multicolumn{1}{c|}{5.56 \% }  &\multicolumn{1}{c|}{46}     &\multicolumn{1}{c|}{2 }              &\multicolumn{1}{c|}{0.79 \%} \\ \hline
        \multirow{2}{*}{กตเวทิตาปี 2556}      & 15    & 242       & \multicolumn{1}{c|}{33 }         & \multicolumn{1}{c|}{13.64 \%}  &\multicolumn{1}{c|}{2 }     &\multicolumn{1}{c|}{1 }              &\multicolumn{1}{c|}{0.41 \%} \\ \cline{2-8} 
                                            & 29    & 257       & \multicolumn{1}{c|}{20 }         & \multicolumn{1}{c|}{7.78 \% } & \multicolumn{1}{c|}{3  }    & \multicolumn{1}{c|}{10 }             & \multicolumn{1}{c|}{3.89 \% }\\ \hline
        \multirow{2}{*}{รายงานประจำปี 2544}   & 15    & 47        & \multicolumn{1}{c|}{3  }         & \multicolumn{1}{c|}{6.38 \% } & \multicolumn{1}{c|}{2  }    & \multicolumn{1}{c|}{34 }             & \multicolumn{1}{c|}{72.34 \%} \\ \cline{2-8} 
                                            & 29    & 585       & \multicolumn{1}{c|}{39 }         & \multicolumn{1}{c|}{6.67 \% } & \multicolumn{1}{c|}{3  }    & \multicolumn{1}{c|}{308}             & \multicolumn{1}{c|}{52.65 \%} \\ \hline
        \multirow{2}{*}{รายงานประจำปี 2553}   & 15    & 68        & \multicolumn{1}{c|}{0  }         & \multicolumn{1}{c|}{0 \%    } & \multicolumn{1}{c|}{0  }    & \multicolumn{1}{c|}{68 }             & \multicolumn{1}{c|}{100 \%  }\\ \cline{2-8} 
                                            & 29    & 596       & \multicolumn{1}{c|}{17 }         & \multicolumn{1}{c|}{2.85 \% } & \multicolumn{1}{c|}{8  }    & \multicolumn{1}{c|}{340}             & \multicolumn{1}{c|}{57.05 \%} \\ \hline
        \multirow{2}{*}{รายงานประจำปี 2549}   & 15    & 155       & \multicolumn{1}{c|}{53 }         & \multicolumn{1}{c|}{34.19 \%}  &\multicolumn{1}{c|}{42}     &\multicolumn{1}{c|}{45}              &\multicolumn{1}{c|}{29.03 \%} \\ \cline{2-8} 
                                            & 29    & 304       & \multicolumn{1}{c|}{22 }         & \multicolumn{1}{c|}{7.24 \% } & \multicolumn{1}{c|}{20 }    & \multicolumn{1}{c|}{13 }             & \multicolumn{1}{c|}{4.28 \%}\\ \hline
        \multicolumn{1}{|l|}{}              & total & 2510      & \multicolumn{1}{c|}{205}         & \multicolumn{1}{c|}{8.17 \% } & \multicolumn{1}{c|}{126}    & \multicolumn{1}{c|}{821}             & \multicolumn{1}{c|}{32.71 \%} \\ \hline
        \end{tabular}
        \end{table}

\subsubsection{แบบที่ 2 ใช้การลบพื้นหลัง}

\begin{table}[H]
    \caption{ตารางประเมินการทำการเตรียมข้อมูลรูปภาพแบบที่ 2}\label{tbl:imagep2}
    \begin{tabular}{|c|c|c|p{0.1\linewidth}|p{0.1\linewidth}|c|p{0.1\linewidth}|p{0.1\linewidth}|}
            \hline
            หนังสือ                             & หน้า                       & จำนวนคำทั้งหมด & จำนวนคำผิดที่ตรวจพบ & เปอร์เซ็นต์คำผิดที่ตรวจพบ(\%)    & จำนวนคำเกิน & จำนวนคำที่ไม่สามารถแปลงเป็นดิจิทัล & เปอร์เซ็นต์คำที่ไม่สามารถแปลงเป็นดิจิทัล(\%)    \\ \hline
            \multirow{2}{*}{กตเวทิตาปี 2542}      & 15                         & \multicolumn{1}{c|}{4   }      & \multicolumn{1}{c|}{4  }         & \multicolumn{1}{c|}{100\%  } & \multicolumn{1}{c|}{0 }     & \multicolumn{1}{c|}{0  }             & \multicolumn{1}{c|}{0\%    } \\ \cline{2-8} 
                                                & 29                         & \multicolumn{1}{c|}{252 }      & \multicolumn{1}{c|}{30 }         & \multicolumn{1}{c|}{11.9\% } & \multicolumn{1}{c|}{6 }     & \multicolumn{1}{c|}{9  }             & \multicolumn{1}{c|}{3.57\% } \\ \hline
            \multirow{2}{*}{กตเวทิตาปี 2556}      & 15                         & \multicolumn{1}{c|}{242 }      & \multicolumn{1}{c|}{42 }         & \multicolumn{1}{c|}{17.36\%} & \multicolumn{1}{c|}{2 }     & \multicolumn{1}{c|}{48 }             & \multicolumn{1}{c|}{19.83\%} \\ \cline{2-8} 
                                                & 29                         & \multicolumn{1}{c|}{257 }      & \multicolumn{1}{c|}{54 }         & \multicolumn{1}{c|}{21.01\%} & \multicolumn{1}{c|}{2 }     & \multicolumn{1}{c|}{62 }             & \multicolumn{1}{c|}{24.12\%} \\ \hline
            \multirow{2}{*}{รายงานประจำปี 2544}   & 15                         & \multicolumn{1}{c|}{47  }      & \multicolumn{1}{c|}{27 }         & \multicolumn{1}{c|}{57.45\%} & \multicolumn{1}{c|}{5 }     & \multicolumn{1}{c|}{5  }             & \multicolumn{1}{c|}{10.64\%} \\ \cline{2-8} 
                                                & 29                         & \multicolumn{1}{c|}{585 }      & \multicolumn{1}{c|}{101}         & \multicolumn{1}{c|}{17.26\%} & \multicolumn{1}{c|}{23}     & \multicolumn{1}{c|}{0  }             & \multicolumn{1}{c|}{0\%    } \\ \hline
            \multirow{2}{*}{รายงานประจำปี 2553}   & 15                         & \multicolumn{1}{c|}{68  }      & \multicolumn{1}{c|}{30 }         & \multicolumn{1}{c|}{44.12\%} & \multicolumn{1}{c|}{7 }     & \multicolumn{1}{c|}{0  }             & \multicolumn{1}{c|}{0\%    } \\ \cline{2-8} 
                                                & 29                         & \multicolumn{1}{c|}{596 }      & \multicolumn{1}{c|}{85 }         & \multicolumn{1}{c|}{14.26\%} & \multicolumn{1}{c|}{30}     & \multicolumn{1}{c|}{0  }             & \multicolumn{1}{c|}{0\%    } \\ \hline
            \multirow{2}{*}{รายงานประจำปี 2549}   & 15                         & \multicolumn{1}{c|}{155 }      & \multicolumn{1}{c|}{57 }         & \multicolumn{1}{c|}{36.77\%} & \multicolumn{1}{c|}{14}     & \multicolumn{1}{c|}{4  }             & \multicolumn{1}{c|}{2.58\% } \\ \cline{2-8} 
                                                & 29                         & \multicolumn{1}{c|}{304 }      & \multicolumn{1}{c|}{76 }         & \multicolumn{1}{c|}{25\%   } & \multicolumn{1}{c|}{7 }     & \multicolumn{1}{c|}{0  }             & \multicolumn{1}{c|}{0\%    } \\ \hline
            \multicolumn{1}{|l|}{}              & \multicolumn{1}{c|}{total} & \multicolumn{1}{c|}{2510}      & \multicolumn{1}{c|}{506}         & \multicolumn{1}{c|}{20.16\%} & \multicolumn{1}{c|}{96}     & \multicolumn{1}{c|}{128}             & \multicolumn{1}{c|}{5.1\%  } \\ \hline
            \end{tabular}
            \end{table}

จากผลลัพธ์ตารางที่ \ref{tbl:imagep1} และ \ref{tbl:imagep2} ทำให้ผู้จัดทำเลือกการทำ image processing แบบที่ 2 มาใช้ในการเตรียมรูปภาพก่อนนำไป OCR
ถึงแม้ว่าจะมีจำนวนคำผิดมากกว่าในแบบที่ 1 แต่จำนวนคำที่ไม่ถูกอ่านในการทำ image processing แบบที่ 1 มีมากถึง 32.71 \% ซึ่งจะทำให้ผู้ใช้งานจะมีภาระในการตรวจสอบคำมากกว่าแบบที่ 2 


\subsection{ผลการเปรียบเทียบข้อมูล 2 ชุดสำหรับการทำ OCR}

โดยตอนที่เลือกข้อมูลที่ใช้การทำ OCR ทางผู้จัดได้พบว่ามีชุดข้อมูลที่ทาง Tesseract ได้ปล่อยออกมาในเว็บไซต์หลักซึ่งเป็นชุดเทรนในปี 2016 เป็นชุดที่ 1
และมีข้อมูลชุดที่ได้มีการอ้างอิงมาว่าเป็นชุดข้อมูลที่ดีในปี 2019 ที่สุดที่ได้รับการประเมินจาก Google เป็นชุดที่ 2
ซึ่งทำให้ผู้จัดทำนำชุดข้อมูลทั้งสองชุดนี้มาทำการเปรียบเทียบว่าชุดไหนมีประสิทธภาพมากกว่ากัน
จากการเปรียบข้อมูลทั้งสองชุดระหว่าง ชุดข้อมูลปี 2016 (ชุดที่ 1)ดังตารางที่ \ref{tbl:bigdata} ชุดข้อมูลปี 2019 (ชุดที่ 2) ดังตารางที่ \ref{tbl:smalldata} กับ
พบว่าประสิทธิภาพของข้อมูลชุดที่ 1 มีความถูกต้องอยู่ที่ 76.61 \% ซึ่งมีจำนวนคำผิดสูงกว่าประมาณ 2\% ความถูกต้องอยู่ที่ 
เมื่อเทียบกับข้อมูลชุดที่ 2 ที่มีความถูกต้องอยู่ที่ 77.41 \% แต่ว่ามีจำนวนคำเกินที่ต่างกันเป็นเท่าตัว 
และมีจำนวนคำที่ไม่สามารถแปลงเป็นดิจิทัลมากกว่า 28 คำ

\begin{table}[H]
    \caption{ตารางประเมินข้อมูลชุดที่ 1}\label{tbl:bigdata}
    \begin{tabular}{|c|c|c|p{0.1\linewidth}|p{0.1\linewidth}|c|p{0.1\linewidth}|p{0.1\linewidth}|}
            \hline
            หนังสือ                             & หน้า  & จำนวนคำทั้งหมด & จำนวนคำผิดที่ตรวจพบ & เปอร์เซ็นต์คำผิดที่ตรวจพบ(\%)    & จำนวนคำเกิน & จำนวนคำที่ไม่สามารถแปลงเป็นดิจิทัล & เปอร์เซ็นต์คำที่ไม่สามารถแปลงเป็นดิจิทัล(\%)    \\ \hline
            \multirow{2}{*}{กตเวทิตาปี 2542}      & 15    & 4         & \multicolumn{1}{c|}{2  }         & \multicolumn{1}{c|}{50\%   } & \multicolumn{1}{c|}{0 }     & \multicolumn{1}{c|}{2  }             & \multicolumn{1}{c|}{50\%   } \\ \cline{2-8} 
                                                & 29    & 252       & \multicolumn{1}{c|}{34 }         & \multicolumn{1}{c|}{13.49\%} & \multicolumn{1}{c|}{12}     & \multicolumn{1}{c|}{4  }             & \multicolumn{1}{c|}{1.59\% } \\ \hline
            \multirow{2}{*}{กตเวทิตาปี 2556}      & 15    & 242       & \multicolumn{1}{c|}{37 }         & \multicolumn{1}{c|}{15.29\%} & \multicolumn{1}{c|}{0 }     & \multicolumn{1}{c|}{49 }             & \multicolumn{1}{c|}{20.25\%} \\ \cline{2-8} 
                                                & 29    & 257       & \multicolumn{1}{c|}{47 }         & \multicolumn{1}{c|}{18.29\%} & \multicolumn{1}{c|}{2 }     & \multicolumn{1}{c|}{45 }             & \multicolumn{1}{c|}{17.51\%} \\ \hline
            \multirow{2}{*}{รายงานประจำปี 2544}   & 15    & 47        & \multicolumn{1}{c|}{40 }         & \multicolumn{1}{c|}{85.11\%} & \multicolumn{1}{c|}{0 }     & \multicolumn{1}{c|}{4  }             & \multicolumn{1}{c|}{8.51\% } \\ \cline{2-8} 
                                                & 29    & 585       & \multicolumn{1}{c|}{78 }         & \multicolumn{1}{c|}{13.33\%} & \multicolumn{1}{c|}{11}     & \multicolumn{1}{c|}{15 }             & \multicolumn{1}{c|}{2.56\% } \\ \hline
            \multirow{2}{*}{รายงานประจำปี 2553}   & 15    & 68        & \multicolumn{1}{c|}{44 }         & \multicolumn{1}{c|}{64.71\%} & \multicolumn{1}{c|}{0 }     & \multicolumn{1}{c|}{0  }             & \multicolumn{1}{c|}{0\%    } \\ \cline{2-8} 
                                                & 29    & 596       & \multicolumn{1}{c|}{76 }         & \multicolumn{1}{c|}{12.75\%} & \multicolumn{1}{c|}{9 }     & \multicolumn{1}{c|}{12 }             & \multicolumn{1}{c|}{2.01\% } \\ \hline
            \multirow{2}{*}{รายงานประจำปี 2549}   & 15    & 155       & \multicolumn{1}{c|}{44 }         & \multicolumn{1}{c|}{28.39\%} & \multicolumn{1}{c|}{15}     & \multicolumn{1}{c|}{1  }             & \multicolumn{1}{c|}{0.65\% } \\ \cline{2-8} 
                                                & 29    & 304       & \multicolumn{1}{c|}{53 }         & \multicolumn{1}{c|}{17.43\%} & \multicolumn{1}{c|}{34}     & \multicolumn{1}{c|}{0  }             & \multicolumn{1}{c|}{0\%    } \\ \hline
                                                & total & 2510      & \multicolumn{1}{c|}{455}         & \multicolumn{1}{c|}{18.13\%} & \multicolumn{1}{c|}{83}     & \multicolumn{1}{c|}{132}             & \multicolumn{1}{c|}{5.26\% } \\ \hline
            \end{tabular}
            \end{table}

\begin{table}[H]
    \caption{ตารางประเมินข้อมูลชุดที่ 2}\label{tbl:smalldata}
    \begin{tabular}{|c|c|c|p{0.1\linewidth}|p{0.1\linewidth}|c|p{0.1\linewidth}|p{0.1\linewidth}|}
        \hline
        หนังสือ                             & หน้า  & จำนวนคำทั้งหมด & จำนวนคำผิดที่ตรวจพบ & เปอร์เซ็นต์คำผิดที่ตรวจพบ(\%)    & จำนวนคำเกิน & จำนวนคำที่ไม่สามารถแปลงเป็นดิจิทัล & เปอร์เซ็นต์คำที่ไม่สามารถแปลงเป็นดิจิทัล(\%)    \\ \hline
        \multirow{2}{*}{กตเวทิตาปี 2542}      & 15    & 4         & \multicolumn{1}{c|}{4  }         & \multicolumn{1}{c|}{100\%  } & \multicolumn{1}{c|}{0  }    & \multicolumn{1}{c|}{0   }            & \multicolumn{1}{c|}{0\%    } \\ \cline{2-8} 
                                            & 29    & 252       & \multicolumn{1}{c|}{40 }         & \multicolumn{1}{c|}{15.87\%} & \multicolumn{1}{c|}{20 }    & \multicolumn{1}{c|}{10  }            & \multicolumn{1}{c|}{3.97\% } \\ \hline
        \multirow{2}{*}{กตเวทิตาปี 2556}      & 15    & 242       & \multicolumn{1}{c|}{46 }         & \multicolumn{1}{c|}{19.01\%} & \multicolumn{1}{c|}{11 }    & \multicolumn{1}{c|}{44  }            & \multicolumn{1}{c|}{18.18\%} \\ \cline{2-8} 
                                            & 29    & 257       & \multicolumn{1}{c|}{32 }         & \multicolumn{1}{c|}{12.45\%} & \multicolumn{1}{c|}{2  }    & \multicolumn{1}{c|}{62  }            & \multicolumn{1}{c|}{24.12\%} \\ \hline
        \multirow{2}{*}{รายงานประจำปี 2544}   & 15    & 47        & \multicolumn{1}{c|}{26 }         & \multicolumn{1}{c|}{55.32\%} & \multicolumn{1}{c|}{0  }    & \multicolumn{1}{c|}{4   }            & \multicolumn{1}{c|}{8.51\% } \\ \cline{2-8} 
                                            & 29    & 585       & \multicolumn{1}{c|}{63 }         & \multicolumn{1}{c|}{10.77\%} & \multicolumn{1}{c|}{7  }    & \multicolumn{1}{c|}{28  }            & \multicolumn{1}{c|}{4.79\% } \\ \hline
        \multirow{2}{*}{รายงานประจำปี 2553}   & 15    & 68        & \multicolumn{1}{c|}{36 }         & \multicolumn{1}{c|}{52.94\%} & \multicolumn{1}{c|}{9  }    & \multicolumn{1}{c|}{2   }            & \multicolumn{1}{c|}{2.94\% } \\ \cline{2-8} 
                                            & 29    & 596       & \multicolumn{1}{c|}{65 }         & \multicolumn{1}{c|}{10.91\%} & \multicolumn{1}{c|}{60 }    & \multicolumn{1}{c|}{2   }            & \multicolumn{1}{c|}{0.34\% } \\ \hline
        \multirow{2}{*}{รายงานประจำปี 2549}   & 15    & 155       & \multicolumn{1}{c|}{43 }         & \multicolumn{1}{c|}{27.74\%} & \multicolumn{1}{c|}{30 }    & \multicolumn{1}{c|}{8   }            & \multicolumn{1}{c|}{5.16\% } \\ \cline{2-8} 
                                            & 29    & 304       & \multicolumn{1}{c|}{52 }         & \multicolumn{1}{c|}{17.11\%} & \multicolumn{1}{c|}{34 }    & \multicolumn{1}{c|}{0   }            & \multicolumn{1}{c|}{0\%    } \\ \hline
                                            & total & 2510      & \multicolumn{1}{c|}{407}         & \multicolumn{1}{c|}{16.22\%} & \multicolumn{1}{c|}{173}    & \multicolumn{1}{c|}{160 }            & \multicolumn{1}{c|}{6.37\% } \\ \hline
        \end{tabular}
        \end{table}


จากผลลัพธ์ตารางที่ \ref{tbl:bigdata} และ \ref{tbl:smalldata} ผู้จัดทำเลือกใช้ข้อมูลชุดที่ 1 เนื่องจากมีการแปลงข้อมูลเป็นดิจิทัลได้ครอบคลุมกว่า และมีจำนวนคำเกินน้อยกว่า เมื่อเทียบกับข้อมูลชุดที่ 2

\subsection{ประสิทธิภาพการแก้ไขคำผิด}
จากการเปรียบข้อมูลที่ไม่ถูกแก้ไขคำผิดในตารางที่ \ref{tbl:correction} กับข้อมูลที่ผ่านระบบการแก้ไขคำผิดในตารางที่ \ref{tbl:bigdata}
พบว่าการใช้ระบบการแก้คำผิดทำให้คำผิดที่เกิดขึ้นนั้นน้อยประมาณ 2\% ทำให้เปอร์เซ็นต์ความถูกต้องอยู่ที่ 76.61 \% ดังตารางที่ \ref{tbl:bigdata}

\begin{table}[H]
    \caption{ตารางประเมินข้อมูลชุดที่ 1 ที่ไม่ผ่านการแก้คำผิด}\label{tbl:correction}
    \begin{tabular}{|c|c|c|p{0.1\linewidth}|p{0.1\linewidth}|c|p{0.1\linewidth}|p{0.1\linewidth}|}
        \hline
        หนังสือ                             & หน้า                       & จำนวนคำทั้งหมด & จำนวนคำผิดที่ตรวจพบ & เปอร์เซ็นต์คำผิดที่ตรวจพบ(\%)    & จำนวนคำเกิน & จำนวนคำที่ไม่สามารถแปลงเป็นดิจิทัล & เปอร์เซ็นต์คำที่ไม่สามารถแปลงเป็นดิจิทัล(\%)    \\ \hline
        \multirow{2}{*}{กตเวทิตาปี 2542}    & 15                           & \multicolumn{1}{c|}{4}         & \multicolumn{1}{c|}{4}          & \multicolumn{1}{c|}{100\%}   & 0      & \multicolumn{1}{c|}{0}               & \multicolumn{1}{c|}{0\%}     \\ \cline{2-8} 
                                            & 29                         & \multicolumn{1}{c|}{252 }      & \multicolumn{1}{c|}{30 }         & \multicolumn{1}{c|}{11.9\% } & \multicolumn{1}{c|}{6 }     & \multicolumn{1}{c|}{9  }             & \multicolumn{1}{c|}{3.57\% } \\ \hline
        \multirow{2}{*}{กตเวทิตาปี 2556}    & 15                           & \multicolumn{1}{c|}{242 }      & \multicolumn{1}{c|}{42 }         & \multicolumn{1}{c|}{17.36\%} & \multicolumn{1}{c|}{2 }     & \multicolumn{1}{c|}{48 }             & \multicolumn{1}{c|}{19.83\%} \\ \cline{2-8} 
                                            & 29                         & \multicolumn{1}{c|}{257 }      & \multicolumn{1}{c|}{54 }         & \multicolumn{1}{c|}{21.01\%} & \multicolumn{1}{c|}{2 }     & \multicolumn{1}{c|}{62 }             & \multicolumn{1}{c|}{24.12\%} \\ \hline
        \multirow{2}{*}{รายงานประจำปี 2544} & 15                           & \multicolumn{1}{c|}{47  }      & \multicolumn{1}{c|}{27 }         & \multicolumn{1}{c|}{57.45\%} & \multicolumn{1}{c|}{5 }     & \multicolumn{1}{c|}{5  }             & \multicolumn{1}{c|}{10.64\%} \\ \cline{2-8} 
                                            & 29                         & \multicolumn{1}{c|}{585 }      & \multicolumn{1}{c|}{101}         & \multicolumn{1}{c|}{17.26\%} & \multicolumn{1}{c|}{23}     & \multicolumn{1}{c|}{0  }             & \multicolumn{1}{c|}{0\%    } \\ \hline
        \multirow{2}{*}{รายงานประจำปี 2553} & 15                           & \multicolumn{1}{c|}{68  }      & \multicolumn{1}{c|}{30 }         & \multicolumn{1}{c|}{44.12\%} & \multicolumn{1}{c|}{7 }     & \multicolumn{1}{c|}{0  }             & \multicolumn{1}{c|}{0\%    } \\ \cline{2-8} 
                                            & 29                         & \multicolumn{1}{c|}{596 }      & \multicolumn{1}{c|}{85 }         & \multicolumn{1}{c|}{14.26\%} & \multicolumn{1}{c|}{30}     & \multicolumn{1}{c|}{0  }             & \multicolumn{1}{c|}{0\%    } \\ \hline
        \multirow{2}{*}{รายงานประจำปี 2549} & 15                           & \multicolumn{1}{c|}{155 }      & \multicolumn{1}{c|}{57 }         & \multicolumn{1}{c|}{36.77\%} & \multicolumn{1}{c|}{14}     & \multicolumn{1}{c|}{4  }             & \multicolumn{1}{c|}{2.58\% } \\ \cline{2-8} 
                                            & 29                         & \multicolumn{1}{c|}{304 }      & \multicolumn{1}{c|}{76 }         & \multicolumn{1}{c|}{25\%   } & \multicolumn{1}{c|}{7 }     & \multicolumn{1}{c|}{0  }             & \multicolumn{1}{c|}{0\%    } \\ \hline
        \multicolumn{1}{|l|}{}              & \multicolumn{1}{l|}{total} & \multicolumn{1}{c|}{2510}      & \multicolumn{1}{c|}{506}         & \multicolumn{1}{c|}{20.16\%} & \multicolumn{1}{c|}{96}     & \multicolumn{1}{c|}{128}             & \multicolumn{1}{c|}{5.1\%  } \\ \hline
        \end{tabular}
        \end{table}

\section{ผลลัพธ์จากการค้นหา}
ในการประเมินการค้นหา ผู้จัดทำได้ทำการประเมินการค้นหาโดยการใช้ Confusion matrix 
เพื่อหา Precision และ Recall ของการทำงาน การค้นหาในครั้งนี้มีคำค้นหาทั้งหมด 
16 คำ หาจากหนังสือ ทั้งหมด 44 เล่ม โดยที่แต่ละคำบรรณารักษ์จะเป็นคนกำหนดว่าผลลัพธ์ที่ออกควรเป็นเล่มไหน 

\begin{table}[H]
    \caption{ตารางแสดงการทำ Confusion matrix}\label{tbl:evasearch}
    \begin{tabular}{|c|c|c|}
    \hline
             & TRUE & FALSE \\ \hline
    POSITION & TP   & FP   \\ \hline
    NEGATIVE & FN   & TN   \\ \hline
    \end{tabular}
    \end{table}

การทดสอบการค้นหาในครั้งนี้ไม่มีการช่วยแก้คำจากเจ้าหน้าที่
บรรณารักษ์แต่เป็นการทดสอบโดยใช้ระบบทั้งหมด 
ได้ผลลัพธ์ออกมาเป็นดังตาราง \ref{tbl:evasearch1}

\begin{table}[H]
    \caption{ตารางแสดงผลการค้นหาจากชุดข้อมูล 44 เล่มที่ไม่ผ่านการแก้ไขคำผิดจากมนุษย์}\label{tbl:evasearch1}
    \begin{tabular}{|c|c|c|}
    \hline
             & TRUE & FALSE \\ \hline
    POSITION & 13   & 166   \\ \hline
    NEGATIVE & 57   & 458   \\ \hline
    \end{tabular}
    \end{table}

    หลังจากคำนวนค่าจากตารางได้ค่า Recall 18.57 \% ค่า Precision อยู่ที่ 7.26 \% และค่า Accuracy อยู่ที่ 67.86 \%  
    จากผลลัพธ์ดังกล่าวทำให้ผู้จัดทำได้ลองทำการทดสอบรอบที่ 2 เปรียบเทียบข้อมูลที่ได้รับการแก้คำจากผู้ใช้ 
    โดยจะใช้ชุดข้อมูลทั้งหมด 6 เล่ม และใช้คำค้นหาเดิม เปรียบเทียบระหว่างข้อมูล 6 ชุดที่ได้รับการแก้ไขคำ
    ในโมเดล Word2Vec และข้อมูล 6 เล่มที่ไม่ผ่านการแก้ไขคำในระบบ Word2Vec ซึ่งได้ผลลัพธ์ดังตารางที่ 
    \ref{tbl:evasearch2} และ ตารางที่ \ref{tbl:evasearch3}

\begin{table}[H]
    \caption{ตารางแสดงผลการค้นหาจากชุดข้อมูล 6 เล่มที่ไม่ผ่านการแก้ไขคำผิดจากมนุษย์}\label{tbl:evasearch2}
    \begin{tabular}{|c|c|c|}
    \hline
                & TRUE & FALSE \\ \hline
    POSITION & 9   & 28   \\ \hline
    NEGATIVE & 8   & 51   \\ \hline
    \end{tabular}
    \end{table}

\begin{table}[H]
    \caption{ตารางแสดงผลการค้นหาจากชุดข้อมูล 6 เล่มที่ผ่านการแก้ไขคำผิดจากมนุษย์}\label{tbl:evasearch3}
    \begin{tabular}{|c|c|c|}
    \hline
                & TRUE & FALSE \\ \hline
    POSITION & 15   & 35   \\ \hline
    NEGATIVE & 2   & 44   \\ \hline
    \end{tabular}
    \end{table}

\begin{figure}[H]
    \centering
    \includegraphics[scale=0.7]{editcompare}
    \caption{ภาพแสดงผลการเปรียบเทียบการใช้โมเดลที่ผ่านการแก้ไขคำผิด และไม่ผ่านการแก้ไขคำผิด}\label{fig:editcompare}
\end{figure}
    
    จากผลลัพพธ์ในตารางที่ \ref{tbl:evasearch2} ได้ค่า Precision อยู่ที่ 24.32 \% ค่า Recall 52.94 \% และมี 
    accuracy 62.5 \% หลังจากการแก้ไขคำและทำการทดลองมีค่า Precision อยู่ที่ 30 \% ค่า Recall 88.24 \% และ
    ค่า accuracy อยู่ที่ 61.46 \% จะเห็นได้ว่ามีค่า Precision และ Recall สูงขึ้น แต่มีค่า accuracy ต่ำลง เมื่อทาง
    ผู้จัดทำได้ตรวจสอบระบบการค้นหาพบว่าระบบการค้นหาค่าความสัมพันธ์ Word2Vec ยังไม่ดีนัก เนื่องจาก 
    Word2Vec ของชุดข้อมูล 6 เล่มนั้นมีจำนวน corpus หรือชุดข้อมูลที่นำไปใช้น้อยเกินไปค่าคำเหมือนที่ได้จาก 
    Word2Vec มีค่าใกล้เคียงกัน ดังภาพที่ \ref{fig:scoreword2vec} ซึ่ง ทำให้ได้คำที่ไม่เกี่ยวข้องเข้ามาใช้ในการค้นหา อย่างเลข 6 
    หรือ 2009 ที่ถูกดึงมาใช้ทั้งๆที่ไม่มีความเกี่ยวข้อง ส่งผลให้มีหนังสือเล่มอื่นติดมาด้วย ซึ่งสำหรับคำค้นหาคำนี้ถ้า
    เทียบในระบบใหญ่แล้ว ถ้าคำนี้ไม่ถูกอ่านผิดระบบใหญ่จะสามารถแยกความสำคัญได้ดีมากกว่าระบบเล็ก
    
\begin{figure}[H]
    \centering
    \includegraphics[scale=0.5]{scoreword2vec}
    \caption{ภาพแสดงคะแนนการค้นหาคำเหมือนจากโมเดล word2vec}\label{fig:scoreword2vec}
\end{figure}

ซึ่งถ้าเปรียบเทียบค่า Recall ในแต่ละโมเดลการแก้ไขคำผิด และโมเดลที่ไม่ได้แก้ไขคำผิด จะพบว่าผลลัพธ์การแก้ไขคำผิดจะช่วย
ให้ได้หนังสือที่ครอบคลุมกับผลลัพธ์มากกว่าไม่แก้คำผิด แต่ด้วยความสามารถของโมเดล 
Word2Vec ทำให้ได้หนังสือที่ไม่เกี่ยวข้องเพิ่มมาด้วยเช่นกัน ส่งผลให้มีค่าความแม่นยำ 
หรือค่า accuracy ลดลง ดังนั้นทางผู้จัดทำได้ลองทำการเปรียบเทียบการค้นหาโดยใช้ 
Word2Vec และไม่ใช้ Word2Vec เข้ามาช่วย ได้ผลลัพธ์ออกมาดังภาพที่ \ref{fig:word2vecCompare}

\begin{figure}[H]
    \centering
    \includegraphics[scale=0.7]{word2vecCompare}
    \caption{ภาพแสดงผลการเปรียบเทียบการใช้ word2vec และไม่ใช้ word2vec}\label{fig:word2vecCompare}
\end{figure}

การค้นหาที่ใช้ Word2Vec จะเป็นการค้นหาที่นำคำค้นหาเข้าโมเดล Word2Vec หลังจากนั้นนำคำที่ได้ไปค้นหาคะแนน TF-IDF เพื่อนำไปหาหนังสือที่มีคะแนนความสัมพันธ์กับคำค้นหาโดยเรียงลำดับจากมากไปน้อย 
ส่วนการค้นหาที่ไม่ใช้ Word2Vec จะเป็นการค้นหาโดยการนำคำค้นหาไปหาคะแนน TF-IDF โดยไม่ผ่านโมเดล Word2Vec
จากภาพที่ \ref{fig:word2vecCompare}  จะเห็นได้ว่า การใช้ Word2Vec ทำให้ประสิทธิภาพในการค้นหาลดลง ถึงแม้ผลลัพธ์ความครอบคลุมของหนังสือที่ถูกต้อง (recall) จะมีค่าเท่ากัน 
แต่ก็มีจำนวนหนังสือที่ไม่เกี่ยวข้องเข้ามาเยอะกว่าเมื่อเทียบการค้นหาแบบใช้ TF-IDF เพียงอย่างเดียว



\subsection{ผลการเปรียบเทียบประสิทธิภาพเวลาในการค้นหา}
ทางผู้จัดทำได้ทำการทดสอบการค้นหาโดยกำหนด คำ 3 คำค้น ให้กับเจ้าหน้าที่บรรณารักษ์และบุคคลธรรมดา 
2 คน โดยกำหนดขอบเขตในการค้นหาหนังสือ 6 เล่ม ซึ่งบุคคลภายนอกที่ใช้ระบบค้นหาของโปรเจคนี้ คนที่ 1 
สามารถระบุหนังสือที่มีเนื้อตรงกับคำค้นหาได้ภายใน 1 นาที ในการค้นหา 3 คำค้นหา และเจอหน้าที่มีคำค้นหา 
3 คำภายใน 11 นาที และคนที่ 2 สามารถระบุหนังสือที่มีเนื้อตรงกับคำค้นหาได้ภายใน 1 นาทีในการค้นหา 3 
คำค้นหา และเจอหน้าที่มีคำค้นหา 3 คำภายใน 9 นาที เมื่อเปรียบเทียบกับเจ้าหน้าที่บรรณารักษ์ที่ค้นหาด้วยวิธีการปกติ 
ซึ่งใช้เวลาไปประมาณ 6 นาที ในการระบุหนังสือที่มีเนื้อตรงกับคำค้นหาเนื่องจากต้องสุ่มหยิบเล่มหนังสือและใช้เวลาอีก 
5 นาทีในการระบุหน้าที่มีคำค้นหาทั้ง 3 คำ ดังนั้นการใช้เว็บไซต์ของโปรเจคนี้จะช่วยลดเวลาในการค้นหาหรือระบุเล่มได้ถึง 50 \% 

\section{ผลลัพธ์จากการดำเนินงานในส่วนของการทำเว็บไซต์}

\subsection{การประเมินความพึงพอใจของบรรณารักษ์ต่อการออกแบบ UX/UI}
\begin{table}[H]
\caption{ตารางประเมินความพึงพอใจการออกแบบ UX/UI}\label{tbl:uxuieva}
\begin{tabular}{|p{0.15\linewidth}|p{0.15\linewidth}|p{0.15\linewidth}|p{0.15\linewidth}|p{0.15\linewidth}|c|}
\hline
                        & \multicolumn{1}{c|}{4}                                                                                                 & \multicolumn{1}{c|}{3}                                                                                   & \multicolumn{1}{c|}{2}                                                                                        & \multicolumn{1}{c|}{1}                                                                        & คะแนนที่ได้ \\ \hline
ความสมบูรณ์ของข้อมูล    & ข้อมูลมีความสมบูรณ์   ชัดเจนทำให้เข้าใจความหมายที่ต้องการจะสื่อได้เป็นอย่างดี                                            &  มีข้อมูลที่ชัดเจน   และแม่นยำในบางครั้ง และสามารถแสดงความหมายที่ต้องการจะสื่อได้บ้าง                     &ข้อมูลมีความแม่นยำ   และชัดเจนบ้าง                                                                            & มีข้อมูลที่ไม่ชัดเจน   ไม่ครบ สื่อความหมายได้ไม่ดี                                            & 3           \\ \hline
การออกแบบ           & มีการออกแบบที่เน้นความสำคัญและจัดวางองค์ประกอบ   สี เสียง และการเคลื่อนไหว(animation) ได้อย่างเหมาะสม                            &  มีการจัดหน้า   และองค์ประกอบทำให้เห็นใจความสำคัญของเนื้อหา มีการใช้การเคลื่อนไหว(animation) บ้าง                    & การวางหน้าและการจัดองค์ประกอบมีความไม่เหมาะสม   มีการใช้การเคลื่อนไหว(animation) เข้ามาช่วยบ้าง                               & การวางหน้าและการจัดองค์ประกอบมีความไม่เหมาะสมและไม่มีการใช้  การเคลื่อนไหว(animation) เข้ามาช่วยในการใช้งาน & 4           \\ \hline
การใช้งาน            &  ผู้ใช้สามารถใช้งานปุ่มหรือย้ายไปยังหน้าต่างๆได้อย่างง่ายดาย   แต่มีลิ้งค์(Link) ที่พาไปผิดหน้าอย่างมากหนึ่งลิ้งค์(Link) หรือไม่มีเลย                &  ผู้ใช้สามารถใช้งานปุ่มหรือย้ายไปยังหน้าต่างๆได้อย่างง่ายดาย   แต่มีลิ้งค์(Link) ที่พาไปผิดหน้าอย่างมากสองลิ้งค์(Link)            &ผู้ใช้มีความสับสนในการใช้ปุ่ม   หรือการย้ายไปยังหน้าต่างๆ บางครั้ง และมีลิ้งค์(Link) ที่พาไปผิดหน้าอย่างมากสามลิ้งค์(Link)                      & ผู้ใช้เกิดความสับสนในปุ่มหรือลิ้งค์(Link) ที่ย้ายไปหน้าต่างๆ                                         & 4           \\ \hline
การใช้ภาษา           & มีการใช้คำผิดหรือภาษาที่ไม่เหมาะสมอย่างมาก   1 จุด                                                               &  มีการใช้คำผิดหรือภาษาที่ไม่เหมาะสมอย่างมาก   2 จุด                                                &มีการใช้คำผิดหรือภาษาที่ไม่เหมาะสมอย่างมาก 3 จุด                                                                 & มีการใช้คำผิดหรือภาษาที่ไม่เหมาะสมมากกว่า 4 จุด                                               & 4           \\ \hline
\end{tabular}
\end{table}

\subsection{หน้าหลัก}
\begin{figure}[H]
    \centering
    \includegraphics[scale=0.2]{web1}
    \caption{ภาพแสดงหน้าเว็บหลัก}\label{fig:web1}
\end{figure}

\subsection{การเข้าสู่ระบบเว็บไซต์}
\begin{figure}[H]
    \centering
    \includegraphics[scale=0.2]{weblogin}
    \caption{ภาพแสดงหน้าเข้าสู่ระบบ}\label{fig:weblogin}
\end{figure}
การเข้าสู่ระบบในเว็บไซต์เราได้ใช้ JSON Web Token (JWT) ในการดูสิทธิ์การเข้าใช้ระบบโดยที่เมื่อผู้ใช้งานเข้าสู่ระบบด้วยรหัสผู้ใช้งานและรหัสผ่านที่ถูกต้องNode JS ก็จะคืน Token ที่ถูกเข้ารหัสไว้กลับไปให้ทางเครื่องผู้ใช้งานเก็บใน local storage เพื่อที่จะเป็นการบ่งบอกสิทธิ์การใช้งาน API ที่เหลือทั้งหมดไม่ว่าจะเป็นการค้นหาข้อมูล เพิ่มข้อมูลหนังสือ แก้ไขข้อมูลหนังสือ หรือลบข้อมูลหนังสือออกจากระบบ ถ้าผู้ใช้งานไม่ได้ส่ง Token มาด้วยหรือ Token นั้นมีการดัดแปลงแก้ไขระบบจะทำการลบ Token ภายในเครื่องทึ้งและทำการออกจากระบบโดยทันที

\subsection{การเพิ่มหนังสือเข้าสู่ระบบฐานข้อมูล}
เนื่องจากการเพิ่มหนังสือเข้าสู่ระบบมีขั้นตอนจำนวนมากและใช้เวลานานจึงแบ่งการรอประมวลผลเป็นส่วนของการเพิ่มข้อมูลของหนังสือ ส่วนของการแก้ไขและตรวจสอบคำก่อนนำเข้าสู่ระบบ ส่วนของการตรวจสอบแก้ไขแท็ก ซึ่งผู้ใช้งานไม่จำเป็นต้องรอภายในหน้าเพิ่มหนังสือสามารถไปทำงานฟังก์ชั่นอื่นได้ตามปกติและเมื่อเสร็จกระบวนการเหล่านี้เสร็จจะสามารถกลับมาดำเนินการเพิ่มข้อมูลต่อได้โดยการกดที่หน้าแสดงสถานะ และกลับเข้าสู่กระบวนการเพิ่มข้อมูลหนังสือ
\subsubsection{เพิ่มข้อมูลของหนังสือ}
\begin{figure}[H]
    \centering
    \includegraphics[scale=0.2]{webinsert}
    \caption{ภาพแสดงขั้นตอนการเพิ่มหนังสือขั้นตอนการเพิ่มไฟล์}\label{fig:webinsert}
\end{figure}
\begin{figure}[H]
    \centering
    \includegraphics[scale=0.2]{webinsert2}
    \caption{ภาพแสดงขั้นตอนการเพิ่มหนังสือเข้าสู่ระบบขั้นกรอกข้อมูลขั้นที่ 1}\label{fig:webinsert2}
\end{figure}
\begin{figure}[H]
    \centering
    \includegraphics[scale=0.2]{webinsert3}
    \caption{ภาพแสดงขั้นตอนการเพิ่มหนังสือเข้าสู่ระบบขั้นกรอกข้อมูลขั้นที่ 2}\label{fig:webinsert3}
\end{figure}

\begin{figure}[H]
    \centering
    \includegraphics[scale=0.2]{webinsert4}
    \caption{ภาพแสดงขั้นตอนการเพิ่มหนังสือเข้าสู่ระบบขั้นการเตรียมข้อมูล}\label{fig:webdel}
\end{figure}
ในส่วนนี้จะเป็นการใช้ผู้ใช้งานทำการเลือกไฟล์ PDF และกรอกข้อมูลของหนังสือโดยที่เมื่อผู้ใช้งานยืนยันข้อมูลเรียบร้อยแล้วระบบก็จะทำการเพิ่มไฟล์ PDF เพื่อนำไปทำกระบวนการเปลี่ยน PDF เป็นรูปภาพและทำการ OCR และการเตรียมข้อมูลตัวหนังสือ  เพื่อทำการแปลงข้อมูลออกมาให้ผู้ใช้งาน

\subsubsection{การแก้ไขและตรวจสอบคำก่อนนำเข้าสู่ระบบ}
\begin{figure}[H]
    \centering
    \includegraphics[scale=0.2]{webcorrect}
    \caption{ภาพแสดงขั้นตอนการเพิ่มหนังสือเข้าสู่ระบบขั้นการแก้ไขคำผิด}\label{fig:webcorrect}
\end{figure}

\begin{figure}[H]
    \centering
    \includegraphics[scale=0.2]{websubmit}
    \caption{ภาพแสดงหน้าต่างยืนยันการแก้คำ}\label{fig:websubmit}
\end{figure}

\begin{figure}[H]
    \centering
    \includegraphics[scale=0.2]{webcheckout}
    \caption{ภาพแสดงขั้นตอนการเพิ่มหนังสือเข้าสู่ระบบขั้นการสร้างคำสำคัญ}\label{fig:webcheckout}
\end{figure}
ในส่วนนี้จะเป็นผลลัพธ์การดำเนินการของการเพิ่มข้อมูลหนังสือ จะมีคำของแต่ละหน้าพร้อมรูปภาพประกอบเพื่อให้ผู้ใช้งานได้ตรวจสอบคำเพิ่มและแก้ไขคำได้อย่างอิสระก่อนจะนำคำเหล่านี้เข้าสู่ระบบและในส่วนนี้ถ้ายืนยันการแก้ไขแล้วจะไม่สามารถมาแก้ไขคำในหนังสือเล่มนี้ในระบบได้อีกโดยถ้ายืนยันแล้วระบบจะทำการเพิ่มคำเหล่านี้เข้าสู่ระบบและทำการคำนวนคะแนน TF-IDF ของคำเหล่านี้ก่อนจะสร้างแท็ก ของหนังสือเล่มนี้ให้อัตโนมัติ

\subsubsection{การตรวจสอบแก้ไขแท็ก}
\begin{figure}[H]
    \centering
    \includegraphics[scale=0.2]{webtag}
    \caption{ภาพแสดงขั้นตอนการเพิ่มหนังสือเข้าสู่ระบบขั้นการแก้ไขคำสำคัญ}\label{fig:webtag}
\end{figure}
ในส่วนนี้จะเป็นผลลัพธ์ของการแก้ไขและตรวจสอบคำก่อนนำเข้าสู่ระบบโดยผู้ใช้งานได้แท็ก ที่ทางระบบทำขึ้นอัตโนมัติเพื่อให้ผู้ใช้งานได้ตรวจสอบเพิ่มลดแท็ก ก่อนจะยืนยันเพิ่มเข้าสู่ระบบ
\subsection{การแสดงสถานะการเพิ่มหนังสือ}
\begin{figure}[H]
    \centering
    \includegraphics[scale=0.2]{webstatus}
    \caption{ภาพแสดงสถานะของการเพิ่มข้อมูลเข้าสู่ระบบ}\label{fig:webstatus}
\end{figure}

\subsection{การแสดงการค้นหาหนังสือ}
\begin{figure}[H]
    \centering
    \includegraphics[scale=0.2]{websearch}
    \caption{ภาพแสดงหน้าการค้นหา}\label{fig:websearch}
\end{figure}

\subsection{การแสดงข้อมูลหนังสือ}
\begin{figure}[H]
    \centering
    \includegraphics[scale=0.2]{webview}
    \caption{ภาพแสดงหน้าแสดงหนังสือ}\label{fig:webview}
\end{figure}
จะเป็นการแสดงข้อมูลของหนังสือที่อยู่ภายในระบบที่ผู้ใช้งานกรอกเข้ามาในระบบพร้อมทั้งแสดง PDF ที่ถูกอัพโหลดขึ้นมา

\begin{figure}[H]
    \centering
    \includegraphics[scale=0.2]{webview2}
    \caption{ภาพแสดงข้อมูลของหนังสือ}\label{fig:webview2}
\end{figure}

\subsection{การแสดงการแก้ไขข้อมูลของหนังสือ}

\begin{figure}[H]
    \centering
    \includegraphics[scale=0.2]{webman}
    \caption{ภาพแสดงหน้าการค้นหาในหน้าการจัดการหนังสือ}\label{fig:webman}
\end{figure}

\begin{figure}[H]
    \centering
    \includegraphics[scale=0.2]{webdel}
    \caption{ภาพแสดงหน้าการลบหนังสือ}\label{fig:webdel}
\end{figure}

\begin{figure}[H]
    \centering
    \includegraphics[scale=0.2]{webman1}
    \caption{ภาพแสดงหน้าการแก้ไขข้อมูลขั้นที่ 1}\label{fig:webman1}
\end{figure}

\begin{figure}[H]
    \centering
    \includegraphics[scale=0.2]{webman3}
    \caption{ภาพแสดงหน้าการแก้ไขข้อมูลขั้นที่ 2}\label{fig:webman2}
\end{figure}

\begin{figure}[H]
    \centering
    \includegraphics[scale=0.2]{webman4}
    \caption{ภาพแสดงหน้าการแก้ไขคำสำคัญ}\label{fig:webman4}
\end{figure}

การแก้ไขข้อมูลจะแก้ได้ต่อเมื่อเพิ่มข้อมูลหนังสือเสร็จสิ้นแล้วโดยที่จะสามารถแก้ไขข้อมูลในส่วนของข้อมูลหนังสือและแท็ก ได้เหมือนกันกับการเพิ่มหนังสือโดยเมื่อแก้ไขเสร็จสิ้นแล้วยืนยันระบบจะทำการบันทึกข้อมูลใหม่ให้ทันที
\chapter{สรุปผล}

\section{ผลการดำเนินงาน}
ในภาคการเรียนที่ 1/2563 ทางคณะผู้จัดทำได้ทำการนำเสนอหัวข้อโปรเจค ศึกษาและรวบรวมข้อมูลต่างๆ เก็บ Requirment จากบรรณารักษ์ และได้ออกแบบเว็บไซต์ User Interface, โครงสร้างฐานข้อมูลและวิธีต่างๆในการสร้างเว็บไซต์ที่จะทำการแปลงรูปภาพให้อยู่ในรูปแบบดิจิทัล 
และทำการศึกษา เรียนรู้และออกแบบวิธีการเตรียมข้อมูลรูปภาพก่อนที่จะนำไปทำการ OCR สร้างระบบเตรียมข้อมูลตัวหนังสือ อย่างการตัดคำ และการแก้ไขคำผิด เพื่อเตรียมข้อมูลที่ได้จากการทำ OCR ให้อยู่ในรูปแบบที่เหมาะสมสำหรับระบบการค้นหา และสร้างระบบค้นหาคำสำคัญด้วยหลักการของ TF-IDF เพื่อใช้สร้างแท็ก ของตัวหนังสือ

ณ เวลาปัจจุบันในภาคเรียนที่ 2/2563 ทางคณะผู้จัดทำได้วางแผนที่จะสร้างเว็บไซต์ จัดทำระบบการค้นหา และโมเดล Word2Vec ทำการประเมินระบบการออกแบบ User Interface การเตรียมข้อมูลรูปภาพ ระบบการแก้คำผิด และระบบการค้นหา

\begin{table}[H]
\caption{ตารางสรุปผลลัพธ์การดำเนินงาน}\label{tbl:milestone}
\begin{tabular}{|p{0.5\linewidth}|l|l|l|}
\hline
\multicolumn{1}{|c|}{แผนการดำเนินการ}                                                & \multicolumn{1}{c|}{ยังไม่ดำเนินการ} & \multicolumn{1}{c|}{กำลังดำเนินการ} & \multicolumn{1}{c|}{เสร็จสิ้น} \\ \hline
ศึกษาค้นคว้าหาปัญหาของโครงการ                                                        &                                      &                                     & \cellcolor[HTML]{92D050}       \\ \hline
เสนอหัวข้อโปรเจค                                                                     &                                      &                                     & \cellcolor[HTML]{92D050}       \\ \hline
ศึกษาและหาข้อมูลเกี่ยวกับเทคโนโลยีที่ใช้ในโปรเจค                                     &                                      &                                     & \cellcolor[HTML]{92D050}       \\ \hline
ประเมินความเป็นไปได้และกำหนดขอบเขตของโปรเจค                                          &                                      &                                     & \cellcolor[HTML]{92D050}       \\ \hline
จัดเก็บ requirement จากกลุ่มผู้ใช้งาน                                                &                                      &                                     & \cellcolor[HTML]{92D050}       \\ \hline
นำเสนอโครงงานครั้งที่ 1                                                              &                                      &                                     & \cellcolor[HTML]{92D050}       \\ \hline
ออกแบบ UX/UI                                                                         &                                      &                                     & \cellcolor[HTML]{92D050}       \\ \hline
การแปลงรูปภาพให้อยู่ในรูปแบบดิจิทัล                                                        &                                      &                                     & \cellcolor[HTML]{92D050}       \\ \hline
นำข้อมูลที่เก็บไว้มาทำการตัดแบ่งคำภาษาไทยและทำการสร้างแท็ก โดยใช้หลักการของ TF-IDF &                                      &                                     & \cellcolor[HTML]{92D050}       \\ \hline
จัดทำระบบการค้นหา                                                                    &                                      & \cellcolor[HTML]{FFC000}            &                                \\ \hline
จัดทำเว็บไซต์แพลตฟอร์ม                                                               &                                      &                                     & \cellcolor[HTML]{92D050}       \\ \hline
ทดสอบระบบ                                                                            &                                      & \cellcolor[HTML]{FFC000}            &                                \\ \hline
ปรับปรุงแก้ไข                                                                        &                                      & \cellcolor[HTML]{FFC000}            &                                \\ \hline
นำเสนอโปรเจค                                                                         & \cellcolor[HTML]{FF0000}                & \cellcolor[HTML]{FFFFFF}               &                                \\ \hline
\end{tabular}
\end{table}

\section{ปัญหาที่พบและการแก้ไข}
\subsection{ปัญหาหน้าสีอ่านยาก}
เนื่องจากหนังสือแต่ละเล่มมีลักษณะที่แตกต่างกันในเรื่องของสีของกระดาษและตัวอักษร ลักษณะการสแกนรูปภาพจึงทำให้การนำประโยคข้อความที่ถูกตัดออกมาทำ OCR แล้วเกิดความผิดผลาดเยอะ

\underline{การแก้ไข}

	ทำการแก้ไขกระบวนการการเตรียมข้อมูลรูปภาพ จากการพยายามข้ามหน้าสีเป็นพัฒนาการลบพื้นหลังในรูปแบบใหม่
\subsection{ปัญหาการหมุนไม่ตรง}
เนื่องจากภาษาไทยมีสระวรรณยุกต์ด้านบนต่อกันสูงสุดต่อกันถึง 2 ชั้นนอกจากนั้นยังมีสระวรรณยุกต์ด้านล่างทำให้บางทีไม่สามารถแยกข้อความแต่ละบรรทัดออกมาได้อย่างสมบูรณ์จึงทำให้มีการหมุนที่ผิดพลาดเกิดขึ้น

\underline{การแก้ไข}

	ทำการแก้ไขกระบวนการการเตรียมข้อมูลรูปภาพ ปรับเปลี่ยนวิธีการหมุนเป็นการหา arctan ที่จุดขอบด้านบนที่ทำให้การหมุนมีข้อผิดพลาดน้อยลง
\subsection{ปัญหาการแก้ไขคำผิด}
เนื่องจากการแก้ไขคำผิดยังไม่สามารถแก้คำผิดรูปแบบคำเฉพาะได้อย่างเช่นชื่อคน หรือชื่อสถานที่ หรืออาจจะคิดว่าคำเฉพาะนั้นผิดพลาดและทำการแก้ไขให้อัตโนมัติทำให้คำที่ได้รับออกมาเกิดข้อผิดพลาดขึ้น

\underline{การแก้ไข}

	ให้ผู้ใช้งานได้ตรวจสอบและแก้ไขได้เองก่อจะเพิ่มข้อมูลเข้าสู่ระบบและเพิ่มข้อมูลคำเฉพาะบางส่วนลงไป
\subsection{ปัญหาเรื่องระยะเวลาในการเพิ่มข้อมูลหนังสือ}
เนื่องจากการเพิ่มข้อมูลมีขั้นตอนจำนวนมากและใช้เวลานานผู้จัดทำจึงออกแบบโครงสร้างเป็นระบบ thread เพื่อที่จะให้เซิฟเวอร์ตอบกลับไปยังผู้ใช้งานและเปิด thread ในการทำงานไม่ว่าจะเป็น OCR หรือการทำการเตรียมข้อมูลตัวหนังสือ เพื่อเพิ่มความเร็วในการทำงานแต่ก็ไม่สามารถลดเวลาการทำงานลงได้

\underline{การแก้ไข}

	ทำแก้ไขโดยการ spawn process แทนขึ้นมาเนื่องจาก python นั้นเวลาเปิด thread ยังคงใช้ core เดียวในการประมวลลัพธ์จึงต้องเปลี่ยนมาเป็นการ spawn process ที่แยกการใช้ core ของหน่วยประมวลผลทำให้สามารถลดเวลาในการเพิ่มข้อมูลหนังสือได้
\section{ข้อจำกัดและข้อเสนอแนะ}
\begin{enumerate}
    \item การแก้ไขคำเฉพาะยังไม่สามารถทำได้ถึงแม้จะมีการเพิ่มคำศัพท์เฉพาะลงไปก็ตามแต่ก็ยังต้องให้มนุษย์เป็นผู้ตรวจสอบอีกรอบเพื่อความถูกต้อง
    \item เนื่องจากลักษณะหนังสือแต่และเล่มแตกต่างกันทำให้การเตรียมข้อมูลรูปภาพที่อาจจะไม่ได้ส่งผลลัพธ์ที่ดีที่สุดให้กับหนังสือทุกประเภททำให้ประสิทธิ์ภาพในหนังสือบางเล่มน้อยกว่าหรือมากกว่าอีกเล่มได้
    \item ลักษณะแสงของการสแกนหนังสือเนื่องจากไฟล์ที่ได้รับมาไม่มีการควบคุมในการสแกนหนังสือเข้ามาจึงทำให้ไฟล์มีความสว่างที่ไม่เท่ากันขนาดและการจัดวางที่ไม่เหมือนกันทำให้ประสิทธิ์ภาพของแต่ละเล่มอาจจะไม่เท่าเดิม
\end{enumerate}


%%%%%%%%%%%%%%%%%%%%%%%%%%%%%%%%%%%%%%%%%%%%%%%%%%%%%%%%%%%%%%%
%%%%%%%%%%%%%%%%%%%% Bibliography %%%%%%%%%%%%%%%%%%%%%%%%%%%%%
%%%%%%%%%%%%%%%%%%%%%%%%%%%%%%%%%%%%%%%%%%%%%%%%%%%%%%%%%%%%%%%

%%%% Comment this in your report to show only references you have
%%%% cited. Otherwise, all the references below will be shown.
% \nocite{*}
%% Use the kmutt.bst for bibtex bibliography style 
%% You must have cpe.bib and string.bib in your current directory.
%% You may go to file .bbl to manually edit the bib items.
\bibliographystyle{kmutt}
\bibliography{string,cpe}

\end{document}
