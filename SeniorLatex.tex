%%%%%% Run at command line, run
%%%%%% xelatex grad-sample.tex 
%%%%%% for a few times to generate the output pdf file
\documentclass[12pt,oneside,openright,a4paper]{cpe-thai-project}
\usepackage{enumitem}
\usepackage{multirow}
\usepackage[table,xcdraw]{xcolor}
\usepackage{float}

\defaultfontfeatures{Mapping=tex-text,Scale=1.23,LetterSpace=0.0}
\setmainfont[Scale=1.23,LetterSpace=0,WordSpace=1.0,FakeStretch=1.0]{TH Sarabun New}
%\setmathfont(Digits)[Scale=1.0,LetterSpace=0,FakeStretch=1.0]{Times New Roman}

%%%%%%%%%%%%%%%%%%%%%%%%%%%%%%%%%%%%%%%%%%%%%%%%%%%%%%%%%%%%%%%%%%%
% Customize below to suit your needs 
% The ones that are optional can be left blank. 
%%%%%%%%%%%%%%%%%%%%%%%%%%%%%%%%%%%%%%%%%%%%%%%%%%%%%%%%%%%%%%%%%%%
% First line of title
\def\disstitleone{Project No. 67}   
% Second line of title
\def\disstitletwo{ระบบจัดเก็บและจัดการเอกสารภายในหอบรรณสารสนเทศ }  
% Three line of title
\def\disstitlethree{(KMUTT Archives Management Platform) }   

% Your first name and lastname
\def\dissauthor{Mr.Akarapon Boonsermsakul}   % 1st member
%%% Put other group member names here ..
\def\dissauthortwo{Ms.Thanaporn Pitianusorn}   % 2nd member (optional)
\def\dissauthorthree{Mr.Annop Kongsombatcharoen}   % 3rd member (optional)


% The degree that you're persuing..
\def\dissdegree{Bachelor of Engineering} % Name of the degree
\def\dissdegreeabrev{B.Eng} % Abbreviation of the degree
\def\dissyear{2020}                   % Year of submission
\def\thaidissyear{2563}               % Year of submission (B.E.)

%%%%%%%%%%%%%%%%%%%%%%%%%%%%%%%%%%%%%%%%%%%%
% Your project and independent study committee..
%%%%%%%%%%%%%%%%%%%%%%%%%%%%%%%%%%%%%%%%%%%%
\def\dissadvisor{Asst.Prof. Suthathip Manee, Ph.D.}  % Advisor
%%% Leave it empty if you have no Co-advisor
\def\disscoadvisor{}  % Co-advisor
\def\disscommitteetwo{Dr.Prapong Prechaprapranwong, Ph.D.}  % 3rd committee member (optional)
\def\disscommitteethree{Asst.Prof.Sanan Srakaew}   % 4th committee member (optional) 
\def\disscommitteefour{Asst.Prof.Surapont Toomnark}    % 5th committee member (optional) 

\def\worktype{Project} %%  Project or Independent study
\def\disscredit{3}   %% 3 credits or 6 credits


\def\fieldofstudy{Computer Engineering} 
\def\department{Computer Engineering} 
\def\faculty{Engineering}

\def\thaifieldofstudy{วิศวกรรมคอมพิวเตอร์} 
\def\thaidepartment{วิศวกรรมคอมพิวเตอร์} 
\def\thaifaculty{วิศวกรรมศาสตร์}
 
\def\appendixnames{Appendix} %%% Appendices or Appendix

\def\thaiworktype{ปริญญานิพนธ์} %  Project or research project % 
\def\thaidisstitleone{ระบบจัดเก็บและจัดการเอกสารภายในหอบรรณสารสนเทศ}
\def\thaidisstitletwo{KMUTT Archives Management Platform}
\def\thaidissauthor{นายอัครพล บุญเสริมศักดิ์กุล}
\def\thaidissauthortwo{นางสาวธนพร ปิติอนุสรณ์} %Optional
\def\thaidissauthorthree{นายอรรณพ กองสมบัติเจริญ} %Optional

\def\thaidissadvisor{ผศ.ดร.สุธาทิพย์ มณีวงศ์วัฒนา}
%% Leave this empty if you have no co-advisor
\def\thaidissdegree{วิศวกรรมศาสตรบัณฑิต}

% Change the line spacing here...
\linespread{1.15}

%%%%%%%%%%%%%%%%%%%%%%%%%%%%%%%%%%%%%%%%%%%%%%%%%%%%%%%%%%%%%%%%
% End of personal customization.  Do not modify from this part 
% to \begin{document} unless you know what you are doing...
%%%%%%%%%%%%%%%%%%%%%%%%%%%%%%%%%%%%%%%%%%%%%%%%%%%%%%%%%%%%%%%%


%%%%%%%%%%%% Dissertation style %%%%%%%%%%%
%\linespread{1.6} % Double-spaced  
%%\oddsidemargin    0.5in
%%\evensidemargin   0.5in
%%%%%%%%%%%%%%%%%%%%%%%%%%%%%%%%%%%%%%%%%%%
%\renewcommand{\subfigtopskip}{10pt}
%\renewcommand{\subfigbottomskip}{-5pt} 
%\renewcommand{\subfigcapskip}{-6pt} %vertical space between caption
%                                    %and figure.
%\renewcommand{\subfigcapmargin}{0pt}

\renewcommand{\topfraction}{0.85}
\renewcommand{\textfraction}{0.1}

\newtheorem{theorem}{Theorem}
\newtheorem{lemma}{Lemma}
\newtheorem{corollary}{Corollary}

\def\QED{\mbox{\rule[0pt]{1.5ex}{1.5ex}}}
\def\proof{\noindent\hspace{2em}{\itshape Proof: }}
\def\endproof{\hspace*{\fill}~\QED\par\endtrivlist\unskip}
%\newenvironment{proof}{{\sc Proof:}}{~\hfill \blacksquare}
%% The hyperref package redefines the \appendix. This one 
%% is from the dissertation.cls
%\def\appendix#1{\iffirstappendix \appendixcover \firstappendixfalse \fi \chapter{#1}}
%\renewcommand{\arraystretch}{0.8}
%%%%%%%%%%%%%%%%%%%%%%%%%%%%%%%%%%%%%%%%%%%%%%%%%%%%%%%%%%%%%%%%
%%%%%%%%%%%%%%%%%%%%%%%%%%%%%%%%%%%%%%%%%%%%%%%%%%%%%%%%%%%%%%%%
\begin{document}

\makesignaturepage 

%%%%%%%%%%%%%%%%%%%%%%%%%%%%%%%%%%%%%%%%%%%%%%%%%%%%%%%%%%%%%%
%%%%%%%%%%%%%%%%%%%%%% English abstract %%%%%%%%%%%%%%%%%%%%%%%
%%%%%%%%%%%%%%%%%%%%%%%%%%%%%%%%%%%%%%%%%%%%%%%%%%%%%%%%%%%%%%
\abstract

KMUTT's library have collected the archive of valued documents. 
Because these document have not transformed into digital form, 
there is vital problem in searching for information in these 
document for librarian and patrons. In this project, we developed 
web platform to digitize these document into digital format 
and implement the search function that facilitate the librarian 
and patron to search for information. The platform consists of 
2 components. The first part is importing documents and 
digitization. In this step, we applied image processing 
techniques such as Morphology Transformation to preprocess
the images of documents and transform the images to full text 
data by using Tesseract. After getting the text files, we tokenize 
the text into words by using the Deepcut library and find the 
significant words of the document by using the TF-IDF algorithm. 
In the second part, we start by getting the input from the user 
and use the word2Vec model to find a similar word. And take 
input and similar words to get the TF-IDF score that we 
generate at first to find the best document for the input word.  

\begin{flushleft}
\begin{tabular*}{\textwidth}{@{}lp{0.8\textwidth}}
\textbf{Keywords}: & Natural language processing / RESTful Service / Optical character recognition / Image Processing / Information retrieval / Term Frequency-Inverse Document Frequency / Word2Vec / Word Embedded 
\end{tabular*}
\end{flushleft}
\endabstract

%%%%%%%%%%%%%%%%%%%%%%%%%%%%%%%%%%%%%%%%%%%%%%%%%%%%%%%%%%%%%%
%%%%%%%%%% Thai abstract here %%%%%%%%%%%%%%%%%%%%%%%%%%%%%%%%%
%%%%%%%%%%%%%%%%%%%%%%%%%%%%%%%%%%%%%%%%%%%%%%%%%%%%%%%%%%%%%%
{\newfontfamily\thaifont{TH Sarabun New:script=thai}[Scale=1.3]
\XeTeXlinebreaklocale "th_TH"	
\thaifont
\thaiabstract

การจะสืบค้นข้อมูลจากเอกสารหรือชั้นหนังสือที่มีการรวบรวมข้อมูลไว้ตั้งแต่อดีตนั้นเป็น
ปัญหาอย่างหนึ่งของเจ้าหน้าที่บรรณารักษ์ที่ต้องทำการดูแลเอกสารเหล่านี้ เนื่องจาก
การที่ยังไม่มีการเก็บหนังสือและเอกสารให้อยู่ในรูปแบบของข้อมูลดิจิตอลทำให้ต้อง
สืบค้นโดยการค้นหาเอกสารและหนังสือแต่ละเล่มโดยการดูจากเนื้อหาสารบัญเพื่อให้
ได้หนังสือที่ตรงกับข้อมูลที่ต้องการมากที่สุด ซึ่งการที่ค้นหาจากหน้าสารบัญของ
หนังสือแต่ละเล่มก็จะทำให้การค้นหาเป็นไปอย่างล่าช้า และบางครั้งการดูเพียง
แค่สารบัญก็อาจจะทำให้ได้หนังสือที่ไม่ตรงกับความต้องการของผู้ที่เข้ามายืมหนังสือ 
ในโครงการนี้เราได้ทำการพัฒนาการระบบจัดเก็บและค้นหาเอกสารอิเล็กทรินิกส์ 
โดยแบ่งออกเป็น 2 ขั้นตอนคือ การนำเข้าข้อมูล  และการสร้างระบบค้นหา 
โดยขั้นตอนการนำเข้าข้อมูล เราจะเริ่มจากการทำ image processing 
เพื่อเตรียมข้อมูลรูปภาพที่ได้มา ก่อนจะนำไปผ่านกระบวนการ OCR 
เพื่อแปลงรูปภาพเหล่านี้ให้อยู่ในรูปของข้อมูลดิจิตอล โดยการเก็บข้อมูลในรูปแบบของ 
Information Retrieval เพื่อช่วยให้ความเร็วการค้นหามีประสิทธิภาพมากยิ่งขึ้น 
และนำข้อมูลมาทำการตัดคำ และเช็คคำผิด จากนั้นจะนำมาหาคำสำคัญของหนังสือหรือเอกสารนั้น ๆ
โดยการใช้การหาคะแนนแบบ TF-IDF ส่วนการสร้างระบบการค้นหาจะเริ่มจากรับคำค้นหามาจากผู้
ใช้และทำการนำคำที่ได้ไปเข้าโมเดล word2Vec เพื่อหาคำที่ใกล้เคียง 
จากนั้นนำคำใกล้เคียงและคำค้นหาไปดึงคะแนน TF-IDF ที่เก็บไว้เพื่อค้นหาว่า
มีเอกสารหรือหนังสือเล่มไหนที่มีคะแนนที่ตรงและใกล้เคียงกับคำค้นหามากที่สุด

\begin{flushleft}
\begin{tabular*}{\textwidth}{@{}lp{0.8\textwidth}}
 & \\

\textbf{คำสำคัญ}: & Natural language processing / RESTful Service / Optical character recognition / Image Processing / Information retrieval / Term Frequency-Inverse Document Frequency / Word2Vec / Word Embedded 
\end{tabular*}
\end{flushleft}
\endabstract
}

%%%%%%%%%%%%%%%%%%%%%%%%%%%%%%%%%%%%%%%%%%%%%%%%%%%%%%%%%%%%
%%%%%%%%%%%%%%%%%%%%%%% Acknowledgments %%%%%%%%%%%%%%%%%%%%
%%%%%%%%%%%%%%%%%%%%%%%%%%%%%%%%%%%%%%%%%%%%%%%%%%%%%%%%%%%%
\preface
ขอขอบคุณนางสาวอารยา ศรีบัวบาน เจ้าหน้าที่หอบรรณสารสนเทศและ ผศ.ดร.สุธาทิพย์ มณีวงศ์วัฒนา อาจารย์ที่ปรึกษารวมทั้งเจ้าหน้าที่ภายในหอสมุดมหาวิทยาลัยเทคโนโลยีพระจอมเกล้าธนบุรีที่เสียสละเวลาให้ความรู้ความเข้าใจ ทั้งในเรื่องการเก็บข้อมูลและคอยแนะนำวิธีการจัดการกับปัญหาต่างๆที่เกิดขึ้น นำมาสู่การทำหัวข้อปริญญานิพนธ์ฉบับนี้ให้สำเร็จตามที่ต้องการ 

%%%%%%%%%%%%%%%%%%%%%%%%%%%%%%%%%%%%%%%%%%%%%%%%%%%%%%%%%%%%%
%%%%%%%%%%%%%%%% ToC, List of figures/tables %%%%%%%%%%%%%%%%
%%%%%%%%%%%%%%%%%%%%%%%%%%%%%%%%%%%%%%%%%%%%%%%%%%%%%%%%%%%%%
% The three commands below automatically generate the table 
% of content, list of tables and list of figures
\tableofcontents                    
\listoftables
\listoffigures                      

%%%%%%%%%%%%%%%%%%%%%%%%%%%%%%%%%%%%%%%%%%%%%%%%%%%%%%%%%%%%%%
%%%%%%%%%%%%%%%%%%%%% List of symbols page %%%%%%%%%%%%%%%%%%%
%%%%%%%%%%%%%%%%%%%%%%%%%%%%%%%%%%%%%%%%%%%%%%%%%%%%%%%%%%%%%%
% You have to add this manually..
\listofsymbols
\begin{flushleft}
\begin{tabular}{@{}p{0.07\textwidth}p{0.7\textwidth}p{0.1\textwidth}}
\textbf{SYMBOL}  & & \textbf{UNIT} \\[0.2cm]
$\alpha$ & Test variable\hfill & m$^2$ \\
$\lambda$ & Interarival rate\hfill &  jobs/second\\
$\mu$ & Service rate\hfill & jobs/second\\
\end{tabular}
\end{flushleft}
%%%%%%%%%%%%%%%%%%%%%%%%%%%%%%%%%%%%%%%%%%%%%%%%%%%%%%%%%%%%%%
%%%%%%%%%%%%%%%%%%%%% List of vocabs & terms %%%%%%%%%%%%%%%%%
%%%%%%%%%%%%%%%%%%%%%%%%%%%%%%%%%%%%%%%%%%%%%%%%%%%%%%%%%%%%%%
% You also have to add this manually..
\listofvocab
\begin{flushleft}
\begin{tabular}{@{}p{1in}@{=\extracolsep{0.5in}}l}
ABC & Adaptive Bandwidth Control \\
MANET & Mobile Ad Hoc Network 
\end{tabular}
\end{flushleft}

%\setlength{\parskip}{1.2mm}

%%%%%%%%%%%%%%%%%%%%%%%%%%%%%%%%%%%%%%%%%%%%%%%%%%%%%%%%%%%%%%%
%%%%%%%%%%%%%%%%%%%%%%%% Main body %%%%%%%%%%%%%%%%%%%%%%%%%%%%
%%%%%%%%%%%%%%%%%%%%%%%%%%%%%%%%%%%%%%%%%%%%%%%%%%%%%%%%%%%%%%%
\chapter{บทนำ}

\section{คำสำคัญ}

Natural language processing, RESTful Service, Optical character recognition, Image Processing, Information retrieval, Term Frequency-Inverse Document Frequency, Word2Vec, Word Embedded 

\section{ความสำคัญของปัญหา}

นับตั้งแต่การก่อตั้งหอสมุดมหาวิทยาลัยเทคโนโลยีพระจอมเกล้าธนบุรี ได้มีการเก็บรวบรวมองค์ความรู้จากประสบการณ์การทำงานของคณะอาจารย์ ผู้เชี่ยวชาญในทางด้านศาสตร์ต่าง ๆ ในรูปแบบลายมือและสื่อสิ่งพิมพ์ไม่ว่าจะเป็น หนังสือ หนังสือ รวมถึงบันทึกเหตุการณ์ในอดีตในรูปของจดหมายเหตุเพื่อส่งต่อประวัติศาสตร์ความรู้ไปยังคนรุ่นหลังโดยมีการจัดเก็บอยู่ภายในหอจดหมายเหตุที่มีเจ้าหน้าที่บรรณารักษ์เป็นผู้ดูแล และเนื่องจากการที่ หนังสือ หนังสือยังไม่ได้มีการจัดเก็บในรูปแบบดิจิทัลทำให้เมื่อมีบุคคลภายนอกที่ต้องการข้อมูลเพื่อนำไปทำกิจกรรมต่าง ๆ ไม่ว่าจะเป็นการทำวิจัย รายงาน หรือหาข้อมูลเพื่อประกอบการประชุมก็ตามแต่ ก็จำเป็นที่จะต้องมาติดต่อเจ้าหน้าที่บรรณารักษ์ผู้ดูแลเพื่อที่จะให้เจ้าหน้าที่บรรณารักษ์ทำการค้นหาหนังสือที่มีเนื้อหาตามที่เราต้องการ ซึ่งการค้นหาข้อมูลที่ต้องการนั้นเจ้าหน้าที่จะต้องทำการค้นหาด้วยระบบมือทำให้การค้นหาข้อมูลดำเนินการไปอย่างล่าช้า นอกจากนั้นวิธีการหาข้อมูลของเจ้าหน้าที่บรรณารักษ์จะเลือกตรวจสอบข้อมูลของหนังสือจากการดูสารบัญทำให้ข้อมูลที่ได้รับมาอาจจะตกหล่นจากข้อมูลเล่มอื่นได้ 

เพื่ออำนวยความสะดวกให้กับบรรณารักษ์ในการสืบค้นข้อมูลและทำให้การบริการในการสืบค้นหนังสือต่าง ๆ และให้บุคคลภายนอกสามารถทำการค้นหาข้อมูลได้ด้วยตนเองครบถ้วนทางคณะผู้จัดทำโครงการจึงได้พัฒนาระบบการจัดเก็บหนังสือและระบบการค้นหาโดยการใช้เครื่องมือในการทำ OCR เพื่อแปลงหนังสือให้อยู่ในรูปแบบของหนังสือ ดิจิทัล และหาคำสำคัญในการสร้างแท็ก ด้วยวิธี Term Frequency - Inverse Document Frequency เพื่อเพิ่มประสิทธิ์ภาพให้กับการค้นหา 

\section{ประเภทของโครงงาน}

นำเสนอความต้องการของผู้มีส่วนได้ส่วนเสียเฉพาะกลุ่ม 

\section{วิธีการที่นำเสนอ}

ระบบการค้นหาหนังสือ มีขั้นตอนการทำงานดังนี้

\begin{enumerate}
    \item นำหนังสือมาแปลงเป็นรูปภาพในรูปแบบสแกน
    \item นำรูปภาพเข้าสู่ระบบโดยใช้การรับส่งข้อมูลแบบ RESTful API ในระบุประเภทของการใช้งาน
    \item นำรูปภาพผ่านกระบวนการเตรียมข้อมูลรูปภาพ โดยใช้ OpenCV ในการลบส่วนอื่น ๆที่ไม่ใช่ข้อความออกและตัดเฉพาะข้อความเพื่อนำไปใช้ในขั้นต่อไป
    \item นำรูปที่ผ่านการเตรียมข้อมูลรูปภาพ มาเข้าสู่ระบบ OCR เพื่อแปลงข้อมูลจากรูปภาพมาเป็นข้อความในระบบดิจิทัล
    \item นำข้อมูลที่เก็บไว้มาทำการตัดแบ่งคำภาษาไทยและแก้คำผิด
    \item ค้นหาคำสำคัญโดยใช้วิธี TF-IDF เพื่อนำมาใช้ในการสร้างแท็ก 
    \item นำข้อมูลที่ถูกแปลงเก็บและข้อมูลเกี่ยวกับแท็ก ลงในดาต้าเบส 
    \item ทำระบบค้นหาในรูปแบบโคซายซิมิลาริตี้(Cosine Similarity) 
    \item ทำระบบหาคำใกล้เคียงโดยใช้วิธี Word2Vec
    \item ทำแพลตฟอร์มเว็ปไซต์เพื่อเป็น User Interface ให้กับผู้ใช้งานได้ใช้งานสำหรับการใช้งานในการค้นหาข้อมูลและเพิ่มข้อมูลหนังสือลงไปในฐานข้อมูลเพิ่ม
\end{enumerate}
\section{วัตถุประสงค์}
\begin{enumerate}
    \item สร้างระบบแปลงข้อมูลหนังสือให้อยู่ในรูปแบบดิจิทัล
    \item สร้าง web platform เพื่อทำการค้นหาหนังสือจากคำค้น และพัฒนาเครื่องมือสนับสนุนการทำงานของบรรณารักษ์ประจำหอบรรณสารสนเทศ
    \item สร้างระบบการค้นหาโดยการใช้วิธีการ อินฟอเมชันรีทีฟวอล ซึ่งวัดความใกล้เคียงกันระหว่างคำค้นและข้อมูลในฐานข้อมูลโดยวิธีโคซายซิมิลาริตี้(Cosine Similarity)
    \item เพิ่มประสิทธิ์ภาพในการเข้าถึงข้อมูลในรูปแบบดิจิทัล
    \item เรียนรู้เรื่องการเตรียมข้อมูลรูปภาพ
\end{enumerate}
\section{ขอบเขตของงานวิจัย}
\begin{enumerate}
    \item ระบบแปลงข้อมูลจากหนังสือและหนังสือเก่า รองรับเฉพาะหนังสือที่เป็นตัวอักษรแบบพิมพ์ และรองรับไฟล์หนังสือเฉพาะ PDF เท่านั้น
    \item ทำระบบตัดคำ Stop word ภาษาไทยโดยอ้างอิงมาจาก pythainip และภาษาอังกฤษจาก nltk
    \item ทำระบบค้นหาแบบโคซายซิมิลาริตี้(Cosine Similarity) ในระบบ Information retrieval
    \item ข้อมูลหนังสือที่นำมาใช้คือหนังสือจำพวก งานแสดงกตเวทิตาจิต หนังสือรายงานประจำปี ตั้งแต่ปีพุทธศักราช 2527 ถึง 2560 รวมประมาณ 43 เล่ม จากหอจดหมายเหตุมหาวิทยาลัยเทคโนโลยีพระจอมเกล้าธนบุรี
    \item ทำ platform เว็บไซต์ในรูปแบบ responsive แต่ไม่รองรับขนาดมือถือ รองรับเฉพาะคอมพิวเตอร์หรือโน๊ตบุ๊ค
    \item การแปลงสิ่งพิมพ์เป็นดิจิทัลใช้ Tesseract ในการแปลงหนังสือและหนังสือเป็นรูปแบบดิจิทัล
    \item การตัดคำภาษาไทยทางคณะผู้จัดทำ จะใช้ freeware เช่น DeepCut มาใช้ในส่วนของการตัดคำภาษาไทย
\end{enumerate}
\section{เนื้อหาทางวิศวกรรมที่เป็นต้นฉบับ}
\begin{itemize}
    \item การเตรียมข้อมูลรูปภาพ สำหรับการเตรียมภาพก่อนนำไปทำ OCR 
    
    โปรเจคของเราทำเกี่ยวกับการทำ OCR เพื่ออ่านภาพให้กลายเป็น text แต่ถึงแม้ว่าภาพที่ได้มาจะจากการสแกนหรือการถ่ายรูป แต่ถึงอย่างนั้น OCR ที่ใช้ก็ยังคงมีข้อจำกัดในเรื่องของคุณภาพของภาพที่ใช้ ถ้าเกิดว่าภาพที่ใช้เอียง หรือมี noise จะทำให้การอ่านมีประสิทธิภาพน้อยลง นอกจากนี้การตัดภาพแยกย่อหน้าแต่ละย่อหน้าทำให้การอ่านมีความถูกต้องมากยิ่งขึ้น

    \item การพัฒนาเว็บไซต์สำหรับการค้นหาหนังสือในหอจดหมายเหตุ
    
    เว็บไซต์ของเราจะใช้ ReactJS, NodeJs, python  ในการพัฒนาเว็บไซต์เป็น Interface ให้กับผู้ใช้งาน สำหรับการใช้งานระบบการค้นหาหนังสือ รวมถึงการอัปโหลดหนังสือเพื่อแปลงหนังสือเข้าสู่ระบบดิจิทัลและ API ต่าง ๆ
    
    \item คัดเลือกคำสำคัญออกมาเพื่อสร้างแท็ก 
    
    สำหรับแบ่งแยกหมวดหมู่ของหนังสือโดยใช้ หลักการของ TF-IDF ในการค้นหาคำสำคัญของหนังสือเพื่อนำมาสร้างแท็ก และใช้สำหรับการค้นหาข้อมูล
    
    \item ทำระบบค้นหาโดยใช้คำที่มีความหมายใกล้เคียง
    
    สำหรับการค้นหาเราจะนำคะแนน TF-IDF มาใช้เป็นคะแนนเพื่อใช้ในการค้นหาแบบโคซายซิมิลาริตี้(Cosine Similarity) และค้นหาคำใกล้เคียง (Query Expansion) เพื่อทำให้การค้นหาเจอผลลัพธ์ที่ต้องการเพิ่มมากขึ้น
    
\end{itemize}
\section{การแยกย่อยงาน และร่างแผนการดำเนินงาน}
\begin{enumerate}
    \item 	ศึกษาและค้นคว้าปัญหาของโครงการ
    \item 	เสนอหัวข้อโปรเจค 
    \item 	ค้นหาข้อมูลเกี่ยวกับเทคโนโลยีที่ใช้ในโปรเจค
    \item 	ประเมินความเป็นไปได้และกำหนดขอบเขตของโปรเจค 
    \item 	จัดเก็บ requirement จากกลุ่มผู้ใช้งาน
    \begin{enumerate}[label*=\arabic*.]
        \item   ติดต่อเจ้าหน้าที่ของหอสมุด
        \item   เก็บข้อมูลที่ต้องการแปลงเข้าสู่ระบบดิจิทัล
    \end{enumerate}
    \item 	นำเสนอโครงการครั้งที่ 1 
    \item 	ออกแบบ UX/UI
    \item 	การแปลงรูปภาพให้อยู่ในรูปแบบดิจิทัล
    \begin{enumerate}[label*=\arabic*.]
        \item  นำหนังสือมาแปลงเป็นรูปภาพในรูปแบบสแกน 
        \item  ศึกษาการใช้งาน OpenCV
        \item  สร้างระบบการเตรียมข้อมูลรูปภาพ เพื่อทำการปรับแต่งรูปภาพและทำการปรับแต่งจนได้ระบบที่รองรับกับ Data ที่มี
        \item  นำรูปที่ผ่านการเตรียมข้อมูลรูปภาพ มาเข้าสู่ระบบ OCR เพื่อแปลงข้อมูลจากรูปภาพมาเป็นข้อความในระบบดิจิทัล 
    \end{enumerate}
    \item 	นำข้อมูลที่เก็บไว้มาทำการตัดแบ่งคำภาษาไทยและหาคำสำคัญโดยใช้ TF-IDF 
    \begin{enumerate}[label*=\arabic*.]
        \item	ทำการตัดแบ่งคำ (Tokenization)
        \item	ลบ stop word ออกจากข้อมูล 
    \end{enumerate}
    \item	ทำระบบค้นหา 
    \begin{enumerate}[label*=\arabic*.]
       \item	ทำระบบค้นหาโดยใช้หลักการโคซายซิมิลาริตี้(Cosine Similarity)
       \item	ทำการค้นหาด้วยคำใกล้เคียงโดยใช้ Word2Vec 
    \end{enumerate}    
    \item   จัดทำเว็บไซต์แพลตฟอร์ม
    \item   ทดสอบระบบ
    \item   ปรับปรุงแก้ไข
    \item   นำเสนอโปรเจค
    \end{enumerate}

\section{ตารางการดำเนินงาน}

\renewcommand{\arraystretch}{1.5}
\begin{table}[H]
    \caption{ตารางการดำเนินงาน ภาคการศึกษาที่ 1/2563}\label{tbl:work1}
    \begin{tabular}{|l|p{0.20\linewidth}|l|l|l|l|l|l|l|l|l|l|l|l|l|l|l|l|l|l|l|l|}
    \hline
    \multicolumn{22}{|c|}{ตารางการดำเนินงาน ภาคการศึกษาที่ 1/2563}                                                                                                                                                                                                                                                                                                                                                                                                                                                                                                                                                                                                                                                \\ \hline
                       &                 & \multicolumn{4}{c|}{สิงหาคม}                                                                                                                                   & \multicolumn{4}{c|}{กันยายน}                                                                                     & \multicolumn{4}{c|}{ตุลาคม}                                                                                     & \multicolumn{4}{c|}{พฤศจิกายน}                                                                                     & \multicolumn{4}{c|}{ธันวาคม}                                                                                     \\ \cline{3-22} 
    \multirow{-2}{*}{ที่} & \multicolumn{1}{c|}{\multirow{-2}{*}{หัวข้อ}} & \multicolumn{1}{c|}{1}   & \multicolumn{1}{c|}{2}                          & \multicolumn{1}{c|}{3}                          & \multicolumn{1}{c|}{4}   & \multicolumn{1}{c|}{1}   & \multicolumn{1}{c|}{2}   & \multicolumn{1}{c|}{3}   & \multicolumn{1}{c|}{4}   & \multicolumn{1}{c|}{1}   & \multicolumn{1}{c|}{2}   & \multicolumn{1}{c|}{3}   & \multicolumn{1}{c|}{4}   & \multicolumn{1}{c|}{1}   & \multicolumn{1}{c|}{2}   & \multicolumn{1}{c|}{3}   & \multicolumn{1}{c|}{4}   & \multicolumn{1}{c|}{1}   & \multicolumn{1}{c|}{2}   & \multicolumn{1}{c|}{3}   & \multicolumn{1}{c|}{4}   \\ \hline
    1                  & ศึกษาค้นคว้าหาปัญหาของโครงการ                                        & \cellcolor[HTML]{656565} & \cellcolor[HTML]{656565}                        & \cellcolor[HTML]{656565}                        &                          &                          &                          &                          &                          &                          &                          &                          &                          &                          &                          &                          &                          &                          &                          &                          &                          \\ \hline
    2                  & เสนอหัวข้อโปรเจค                                        &                          & \cellcolor[HTML]{656565}{\color[HTML]{656565} } & \cellcolor[HTML]{656565}{\color[HTML]{656565} } &                          &                          &                          &                          &                          &                          &                          &                          &                          &                          &                          &                          &                          &                          &                          &                          &                          \\ \hline
    3                  & ศึกษาและหาข้อมูลเกี่ยวกับเทคโนโลยีที่ใช้ในโปรเจค                                        &                          &                                                 & \cellcolor[HTML]{656565}                        & \cellcolor[HTML]{656565} &                          &                          &                          &                          &                          &                          &                          &                          &                          &                          &                          &                          &                          &                          &                          &                          \\ \hline
    4                  & ประเมินความเป็นไปได้และกำหนดขอบเขตของโปรเจค                                       &                          &                                                 & \cellcolor[HTML]{656565}                        & \cellcolor[HTML]{656565} &                          &                          &                          &                          &                          &                          &                          &                          &                          &                          &                          &                          &                          &                          &                          &                          \\ \hline
    5                  & จัดเก็บ requirement จากกลุ่มผู้ใช้งาน                                        &                          &                                                 &                                                 & \cellcolor[HTML]{656565} & \cellcolor[HTML]{656565} &                          &                          &                          &                          &                          &                          &                          &                          &                          &                          &                          &                          &                          &                          &                          \\ \hline
    6                  & นำเสนอโครงงานครั้งที่ 1                                        &                          &                                                 &                                                 &                          & \cellcolor[HTML]{656565} &                          &                          &                          &                          &                          &                          &                          &                          &                          &                          &                          &                          &                          &                          &                          \\ \hline
    7                  & ออกแบบ UX/UI                                        &                          &                                                 &                                                 &                          &                          & \cellcolor[HTML]{656565} & \cellcolor[HTML]{656565} &                          &                          &                          &                          &                          &                          &                          &                          &                          &                          &                          &                          &                          \\ \hline
    8                  & แปลงรูปภาพให้อยู่ในรูปแบบดิจิทัล                                       &                          &                                                 &                                                 &                          &                          & \cellcolor[HTML]{656565} & \cellcolor[HTML]{656565} & \cellcolor[HTML]{656565} & \cellcolor[HTML]{656565} & \cellcolor[HTML]{656565} & \cellcolor[HTML]{656565} & \cellcolor[HTML]{656565} & \cellcolor[HTML]{656565} & \cellcolor[HTML]{656565} & \cellcolor[HTML]{656565} &                          &                          &                          &                          &                          \\ \hline
    9                  & นำข้อมูลที่เก็บไว้มาทำการตัดแบ่งคำภาษาไทยและทำการสร้างแท็ก โดยใช้หลักการของ TF-IDF                                        &                          &                                                 &                                                 &                          &                          &                          & \cellcolor[HTML]{656565} & \cellcolor[HTML]{656565} & \cellcolor[HTML]{656565} & \cellcolor[HTML]{656565} & \cellcolor[HTML]{656565} & \cellcolor[HTML]{656565} & \cellcolor[HTML]{656565} & \cellcolor[HTML]{656565} & \cellcolor[HTML]{656565} & \cellcolor[HTML]{656565} & \cellcolor[HTML]{656565} &                          &                          &                          \\ \hline
    10                 & นำเสนอโปรเจค                                        &                          &                                                 &                                                 &                          &                          &                          &                          &                          &                          &                          &                          &                          &                          &                          &                          &                          & \cellcolor[HTML]{656565} &                          &                          &                          \\ \hline
    11                 & จัดทำระบบการค้นหา                                        &                          &                                                 &                                                 &                          &                          &                          &                          &                          &                          &                          &                          &                          &                          &                          &                          &                          &                          & \cellcolor[HTML]{656565} & \cellcolor[HTML]{656565} & \cellcolor[HTML]{656565} \\ \hline
    \end{tabular}
    \end{table}

\begin{table}[H]
\caption{ตารางการดำเนินงาน ภาคการศึกษาที่ 2/2563}\label{tbl:work2}
\begin{tabular}{|l|p{0.35\linewidth}|l|l|l|l|l|l|l|l|l|l|l|l|l|l|l|l|}
\hline
\multicolumn{18}{|c|}{ตารางการดำเนินงาน ภาคการศึกษาที่ 2/2563}                                                                                                                                                                                                                                                                                                                                                                                                                                                                 \\ \hline
                   &                    & \multicolumn{4}{c|}{มกราคม}                                                                                     & \multicolumn{4}{c|}{กุมภาพันธ์}                                                                                     & \multicolumn{4}{c|}{มีนาคม}                                                                                     & \multicolumn{4}{c|}{เมษายน}                                                                                     \\ \cline{3-18} 
\multirow{-2}{*}{ที่} & \multicolumn{1}{c|}{\multirow{-2}{*}{หัวข้อ}} & 1                        & 2                        & 3                        & 4                        & 1                        & 2                        & 3                        & 4                        & 1                        & 2                        & 3                        & 4                        & 1                        & 2                        & 3                        & 4                        \\ \hline
1                  & จัดทำระบบการค้นหา                   & \cellcolor[HTML]{656565} & \cellcolor[HTML]{656565} & \cellcolor[HTML]{656565} & \cellcolor[HTML]{656565} & \cellcolor[HTML]{656565} & \cellcolor[HTML]{656565} & \cellcolor[HTML]{656565} &                          &                          &                          &                          &                          &                          &                          &                          &                          \\ \hline
2                  & จัดทำเว็บไซต์                   & \cellcolor[HTML]{656565} & \cellcolor[HTML]{656565} & \cellcolor[HTML]{656565} & \cellcolor[HTML]{656565} & \cellcolor[HTML]{656565} & \cellcolor[HTML]{656565} & \cellcolor[HTML]{656565} & \cellcolor[HTML]{656565} &                          &                          &                          &                          &                          &                          &                          &                          \\ \hline
3                  & ทดสอบระบบ                   &                          &                          &                          &                          & \cellcolor[HTML]{656565} & \cellcolor[HTML]{656565} & \cellcolor[HTML]{656565} & \cellcolor[HTML]{656565} & \cellcolor[HTML]{656565} & \cellcolor[HTML]{656565} & \cellcolor[HTML]{656565} & \cellcolor[HTML]{656565} & \cellcolor[HTML]{656565} & \cellcolor[HTML]{656565} &                          &                          \\ \hline
4                  & ปรับปรุงแก้ไข                   &                          &                          &                          &                          & \cellcolor[HTML]{656565} & \cellcolor[HTML]{656565} & \cellcolor[HTML]{656565} & \cellcolor[HTML]{656565} & \cellcolor[HTML]{656565} & \cellcolor[HTML]{656565} & \cellcolor[HTML]{656565} & \cellcolor[HTML]{656565} & \cellcolor[HTML]{656565} & \cellcolor[HTML]{656565} &                          &                          \\ \hline
5                  & นำเสนอโปรเจค                   &                          &                          &                          &                          &                          &                          &                          &                          &                          &                          &                          &                          &                          &                          & \cellcolor[HTML]{656565} & \cellcolor[HTML]{656565} \\ \hline
\end{tabular}
\end{table}

\subsection{ผลการดำเนินงานในภาคการศึกษาที่ 1}
\begin{itemize}
    \item ทำระบบเตรียมข้อมูลรูปภาพ สำหรับการเตรียมรูปภาพสำหรับการแปลงข้อมูลเป็นดิจิทัล
    \item ทำ API ในการตัดคำและจัดการ stop word สำหรับการเตรียมการเตรียมข้อมูลตัวหนังสือ
    \item ทำระบบ Term Frequency-Inverse Document Frequency สำหรับการค้นหาคำสำคัญเพื่อสร้างแท็ก 
    \item ทำส่วนของการทำการค้นหาข้อมูลเบื้องต้น
\end{itemize}

\subsection{ผลการดำเนินงานในภาคการศึกษาที่ 2}
\begin{itemize}
    \item ทำระบบค้นหาให้เสร็จสิ้น
    \item ปรับปรุงระบบค้นหาให้ตอบโจทย์มากยิ่งขึ้น
    \item ทำเว็บไซต์ทั้งฝั่ง frontend และ backend
\end{itemize}


\end{document}
